% (find-LATEX "2022-2-C2-edos-lineares.tex")
% (defun c () (interactive) (find-LATEXsh "lualatex -record 2022-2-C2-edos-lineares.tex" :end))
% (defun C () (interactive) (find-LATEXsh "lualatex 2022-2-C2-edos-lineares.tex" "Success!!!"))
% (defun D () (interactive) (find-pdf-page      "~/LATEX/2022-2-C2-edos-lineares.pdf"))
% (defun d () (interactive) (find-pdftools-page "~/LATEX/2022-2-C2-edos-lineares.pdf"))
% (defun e () (interactive) (find-LATEX "2022-2-C2-edos-lineares.tex"))
% (defun o () (interactive) (find-LATEX "2022-2-C2-edos-lineares.tex"))
% (defun u () (interactive) (find-latex-upload-links "2022-2-C2-edos-lineares"))
% (defun v () (interactive) (find-2a '(e) '(d)))
% (defun d0 () (interactive) (find-ebuffer "2022-2-C2-edos-lineares.pdf"))
% (defun cv () (interactive) (C) (ee-kill-this-buffer) (v) (g))
%          (code-eec-LATEX "2022-2-C2-edos-lineares")
% (find-pdf-page   "~/LATEX/2022-2-C2-edos-lineares.pdf")
% (find-sh0 "cp -v  ~/LATEX/2022-2-C2-edos-lineares.pdf /tmp/")
% (find-sh0 "cp -v  ~/LATEX/2022-2-C2-edos-lineares.pdf /tmp/pen/")
%     (find-xournalpp "/tmp/2022-2-C2-edos-lineares.pdf")
%   file:///home/edrx/LATEX/2022-2-C2-edos-lineares.pdf
%               file:///tmp/2022-2-C2-edos-lineares.pdf
%           file:///tmp/pen/2022-2-C2-edos-lineares.pdf
% http://angg.twu.net/LATEX/2022-2-C2-edos-lineares.pdf
% (find-LATEX "2019.mk")
% (find-sh0 "cd ~/LUA/; cp -v Pict2e1.lua Pict2e1-1.lua Piecewise1.lua ~/LATEX/")
% (find-sh0 "cd ~/LUA/; cp -v Pict2e1.lua Pict2e1-1.lua Pict3D1.lua ~/LATEX/")
% (find-sh0 "cd ~/LUA/; cp -v C2Subst1.lua C2Formulas1.lua ~/LATEX/")
% (find-CN-aula-links "2022-2-C2-edos-lineares" "2" "c2m222edols" "c2el")

% «.defs»			(to "defs")
% «.title»			(to "title")
% «.links»			(to "links")
%   «.links-dem-complicada»	(to "links-dem-complicada")
%   «.links-complexos»		(to "links-complexos")
% «.maxima»			(to "maxima")
%
% «.djvuize»			(to "djvuize")



% <videos>
% Video (not yet):
% (find-ssr-links     "c2m222edols" "2022-2-C2-edos-lineares")
% (code-eevvideo      "c2m222edols" "2022-2-C2-edos-lineares")
% (code-eevlinksvideo "c2m222edols" "2022-2-C2-edos-lineares")
% (find-c2m222edolsvideo "0:00")

\documentclass[oneside,12pt]{article}
\usepackage[colorlinks,citecolor=DarkRed,urlcolor=DarkRed]{hyperref} % (find-es "tex" "hyperref")
\usepackage{amsmath}
\usepackage{amsfonts}
\usepackage{amssymb}
\usepackage{pict2e}
\usepackage[x11names,svgnames]{xcolor} % (find-es "tex" "xcolor")
\usepackage{colorweb}                  % (find-es "tex" "colorweb")
%\usepackage{tikz}
%
% (find-dn6 "preamble6.lua" "preamble0")
%\usepackage{proof}   % For derivation trees ("%:" lines)
%\input diagxy        % For 2D diagrams ("%D" lines)
%\xyoption{curve}     % For the ".curve=" feature in 2D diagrams
%
\usepackage{edrx21}               % (find-LATEX "edrx21.sty")
\input edrxaccents.tex            % (find-LATEX "edrxaccents.tex")
\input edrx21chars.tex            % (find-LATEX "edrx21chars.tex")
\input edrxheadfoot.tex           % (find-LATEX "edrxheadfoot.tex")
\input edrxgac2.tex               % (find-LATEX "edrxgac2.tex")
%\usepackage{emaxima}              % (find-LATEX "emaxima.sty")
%
%\usepackage[backend=biber,
%   style=alphabetic]{biblatex}            % (find-es "tex" "biber")
%\addbibresource{catsem-slides.bib}        % (find-LATEX "catsem-slides.bib")
%
% (find-es "tex" "geometry")
\usepackage[a6paper, landscape,
            top=1.5cm, bottom=.25cm, left=1cm, right=1cm, includefoot
           ]{geometry}
%
\begin{document}

\catcode`\^^J=10
\directlua{dofile "dednat6load.lua"}  % (find-LATEX "dednat6load.lua")
%L dofile "Piecewise1.lua"           -- (find-LATEX "Piecewise1.lua")
%L dofile "QVis1.lua"                -- (find-LATEX "QVis1.lua")
%L dofile "Pict3D1.lua"              -- (find-LATEX "Pict3D1.lua")
%L dofile "C2Formulas1.lua"          -- (find-LATEX "C2Formulas1.lua")
%L Pict2e.__index.suffix = "%"
\pu
\def\pictgridstyle{\color{GrayPale}\linethickness{0.3pt}}
\def\pictaxesstyle{\linethickness{0.5pt}}
\def\pictnaxesstyle{\color{GrayPale}\linethickness{0.5pt}}
\celllower=2.5pt

% «defs»  (to ".defs")
% (find-LATEX "edrx21defs.tex" "colors")
% (find-LATEX "edrx21.sty")

\def\u#1{\par{\footnotesize \url{#1}}}

\def\drafturl{http://angg.twu.net/LATEX/2022-2-C2.pdf}
\def\drafturl{http://angg.twu.net/2022.2-C2.html}
\def\draftfooter{\tiny \href{\drafturl}{\jobname{}} \ColorBrown{\shorttoday{} \hours}}



%  _____ _ _   _                               
% |_   _(_) |_| | ___   _ __   __ _  __ _  ___ 
%   | | | | __| |/ _ \ | '_ \ / _` |/ _` |/ _ \
%   | | | | |_| |  __/ | |_) | (_| | (_| |  __/
%   |_| |_|\__|_|\___| | .__/ \__,_|\__, |\___|
%                      |_|          |___/      
%
% «title»  (to ".title")
% (c2m222edolsp 1 "title")
% (c2m222edolsa   "title")

\thispagestyle{empty}

\begin{center}

\vspace*{1.2cm}

{\bf \Large Cálculo 2 - 2022.2}

\bsk

Aula 25: EDOs lineares

com coeficientes constantes

\bsk

Eduardo Ochs - RCN/PURO/UFF

\url{http://angg.twu.net/2022.2-C2.html}

\end{center}

\newpage

% «links»  (to ".links")
% (c2m222edolsp 2 "links")
% (c2m222edolsa   "links")
% (find-books "__analysis/__analysis.el" "trench" "2.1 Linear First Order")
% (find-books "__analysis/__analysis.el" "trench" "5.1 Homogeneous Linear")
% (find-books "__analysis/__analysis.el" "boyce-diprima" "3 Second-Order Linear")
% (find-books "__analysis/__analysis.el" "marsden-weinstein" "12.7. Second-Order Linear")
% (find-books "__analysis/__analysis.el" "marsden-weinstein" "12.7. Second-Order Linear")
% (find-books "__analysis/__analysis.el" "thomas" "16 Second-Order")
% (c2m221dvsp 1 "title")
% (c2m221dvsa   "title")
% (c2m212introp 12 "EDOs-chutar-testar")
% (c2m212introa    "EDOs-chutar-testar")
% (find-angg ".emacs" "c2q192" "f''+7f+10f =")

{\bf Links:}

\scalebox{0.6}{\def\colwidth{14cm}\firstcol{

EDOLCCs no ``Diffy Qs'' do Jiri Lebl:

{\scriptsize

% (find-books "__analysis/__analysis.el" "lebl")
% (find-books "__analysis/__analysis.el" "lebl" "2.1 Second order linear")
% (find-books "__analysis/__analysis.el" "lebl" "2.2 Constant coefficient")
% (find-diffyqspage 84 "2.2 Constant coefficient second order linear ODEs")
%    https://www.jirka.org/diffyqs/diffyqs.pdf#page=84
\url{https://www.jirka.org/diffyqs/diffyqs.pdf\#page=84}

}

\msk

EDOLCCs nas notas da Cristiane Hernández:

{\scriptsize

% (find-books "__analysis/__analysis.el" "hernandez")
% (find-books "__analysis/__analysis.el" "hernandez" "24 EDOs homogêneas")
% (find-books "__analysis/__analysis.el" "hernandez" "com coeficientes constantes")
% (find-hernandezpage (+ 10 219) "26 EDO lineares homogêneas de grau n com coeficientes constantes")
%    http://angg.twu.net/2015.1-C2/CALCULOIIA_EAD_Versao_Final_correcao_aulas_25_a_30.pdf#page=229
\url{http://angg.twu.net/2015.1-C2/CALCULOIIA_EAD_Versao_Final_correcao_aulas_25_a_30.pdf\#page=229}

}

\msk

Questões sobre EDOLCCs nas provas de 2022.1:

{\scriptsize

% (c2m221p2p 5 "edo-2a-ordem")
% (c2m221p2a   "edo-2a-ordem")
%    http://angg.twu.net/LATEX/2022-1-C2-P2.pdf#page=5
\url{http://angg.twu.net/LATEX/2022-1-C2-P2.pdf\#page=5}

% (c2m221vrp 4 "questao-2")
% (c2m221vra   "questao-2")
%    http://angg.twu.net/LATEX/2022-1-C2-VR.pdf#page=4
\url{http://angg.twu.net/LATEX/2022-1-C2-VR.pdf\#page=4}

% (c2m221vsap 2 "questao-1")
% (c2m221vsaa   "questao-1")
%    http://angg.twu.net/LATEX/2022-1-C2-VSA.pdf#page=2
\url{http://angg.twu.net/LATEX/2022-1-C2-VSA.pdf\#page=2}

}


\msk

% «links-dem-complicada»  (to ".links-dem-complicada")
%
``Uma demonstração complicada:''

{\scriptsize

% (c2m221dfip 5 "demonstracao-complicada")
% (c2m221dfia   "demonstracao-complicada")
%    http://angg.twu.net/LATEX/2022-1-C2-der-fun-inv.pdf#page=5
\url{http://angg.twu.net/LATEX/2022-1-C2-der-fun-inv.pdf\#page=5}

}

\msk

Nessas aulas eu escrevi muitas coisas no quadro, e ainda

não digitei elas. Você pode acessar as fotos dos quadros aqui:

% (find-angg ".emacs" "c2q222")
% (find-c2q222page 41 "nov16: EDOLCCs")
% (find-c2q222page 43 "nov17: EDOLCCs (2)")

{\scriptsize

%    http://angg.twu.net/2022.2-C2/C2-quadros.pdf#page=41
\url{http://angg.twu.net/2022.2-C2/C2-quadros.pdf\#page=41}

}

\msk

Material de 2019.2:

{\scriptsize

% (find-angg ".emacs" "c2q192")
% (c2q192 93 "20191101 gde aula 19: f''+7f+10f = 0 via Álgebra Linear")
% (c2q192 95 "20191101 peq aula 19: f''+7f+10f = 0 via Álgebra Linear, aviso MT")
% (c2q192 98 "20191106 peq aula 20: f''+af'+bf = 0 complexo, valor inicial, polígono")
% (c2q192 101 "20191107 gde aula 20: f''+af'+bf = 0 complexo, valor inicial")
% (c2q192 103 "20191108 gde aula 21: f''+af'+bf = 0 complexo, poligono")
%    http://angg.twu.net/2019.2-C2/2019.2-C2.pdf#page=93
\url{http://angg.twu.net/2019.2-C2/2019.2-C2.pdf\#page=93}

% (c2m192tudop 10 "title")
% (c2m192tudoa    "title")
%    http://angg.twu.net/LATEX/2019-2-C2-tudo.pdf#page=10
\url{http://angg.twu.net/LATEX/2019-2-C2-tudo.pdf\#page=10}

}

\msk

Sobre o `$·$' em $42·f$ e em $f·g$, veja os slides 15 até 21 daqui:

{\scriptsize

% (find-books "__cats/__cats.el" "bauer-dawn")
% (find-bauerdawnpage 15 "If f is linear then")
\url{http://math.andrej.com/asset/data/the-dawn-of-formalized-mathematics.pdf}

\url{http://math.andrej.com/2021/06/24/the-dawn-of-formalized-mathematics/}

\url{https://vimeo.com/567049015}

}

\msk

O Leithold define $f+g$, $f-g$ e $f·g$ na página 36 (cap.1)

e define a notação $\left. \frac{dy}{dx} \right]_{x=x_0}$ na p.145 (sec.3.1).

% (find-books "__analysis/__analysis.el" "leithold" "(+ 17  36)" "f+g, f-g, f*g")
% (find-books "__analysis/__analysis.el" "leithold" "(+ 17 145)" "at")

\msk

% «links-complexos»  (to ".links-complexos")
% (c2m222edolsp 2 "links-complexos")
% (c2m222edolsa   "links-complexos")

Capítulo sobre números complexos no ``GA1'' do Acker:

{\scriptsize

% (find-books "__analysis/__analysis.el" "acker")
% (find-books "__analysis/__analysis.el" "acker" "23 Números complexos")
%    http://angg.twu.net/acker/acker__ga_livro1_2019.pdf#page=141
\url{http://angg.twu.net/acker/acker__ga_livro1_2019.pdf\#page=141}

% (c3m222taylorp 1 "title")
% (c3m222taylora   "title")

}

}\anothercol{
}}


\newpage


\scalebox{0.6}{\def\colwidth{9cm}\firstcol{

$\begin{array}{rcl}
 (f+g)(x) &=& f(x)+g(x) \\
      f+g &=& λx.f(x)+g(x) \\
 \\
 (k·f)(x) &=& k·f(x) \\
      k·f &=& λx.k·f(x) \\
 \\
 (f·g)(x) &=& f(x)·g(x) \\
      f·g &=& λx.f(x)·g(x) \\
 \\
 (Df)(x) &=& \ddx f(x) \\
      Df &=& λx.\ddx f(x) \\
       D &=& λf.λx.\ddx f(x) \\
 \\
 D(g+h) &=& (λf.λx.\ddx f(x))(g+h) \\
        &=& λx.\ddx (g+h)(x) \\
        &=& λx.\ddx (g(x)+h(x)) \\
        &=& λx.\ddx g(x) + \ddx h(x)) \\
        &=& (λx.\ddx g(x)) + (λx.\ddx h(x))) \\
        &=& Dg + Dh \\
 \\
 D(k·g) &=& (λf.λx.\ddx f(x))(k·g) \\
        &=& λx.\ddx (k·g)(x) \\
        &=& λx.\ddx (k·g(x)) \\
        &=& λx.k·\ddx g(x) \\
        &=& k·(λx.\ddx g(x)) \\
        &=& k·Dg \\
 \end{array}
$

}\anothercol{

$\begin{array}{rcl}
 D(λx.x^2) &=& (λf.λx.\ddx f(x))(λx.x^2) \\
           &=& λx.\ddx (λx.x^2)(x) \\
           &=& λx.\ddx x^2 \\
           &=& λx.2x \\
 \\
            k·f &=& λx.k·f \\
  k·(λx.e^{kx}) &=& λx.k·(λx.e^{kx})(x) \\
                &=& λx.k·e^{kx} \\
  k·(λx.e^{kx}) &=& λx.k·e^{kx} \\
 42·(λx.e^{kx}) &=& λx.42e^{42x} \\
 \\
 D(λx.e^{42x}) &=& (λf.λx.\ddx f(x))(λx.e^{42x}) \\
           &=& λx.\ddx (λx.e^{42x})(x) \\
           &=& λx.\ddx e^{42x} \\
           &=& λx.42e^{42x} \\
           &=& 42·(λx.e^{kx}) \\
 \\
 \end{array}
$

}}

\newpage


\scalebox{0.6}{\def\colwidth{9cm}\firstcol{

$\begin{array}[t]{rcl}
                     e^{iθ}  &=& \cos  θ + i\senθ \\
                     \sen -θ &=& -\sen θ \\
                     \cos -θ &=& -\cos θ \\
                     e^{-iθ} &=& \cos -θ + i\sen-θ \\
                             &=& \cos -θ - i\sen θ \\
                             &=& \cos  θ - i\sen θ \\
              e^{iθ}+e^{-iθ} &=& 2\cos θ \\
              e^{iθ}-e^{-iθ} &=& 2i\sen θ \\
              \cos θ & \eqnp{9}& \frac1 2  (e^{iθ} + e^{-iθ}) \\
              \sen θ &\eqnp{10}& \frac1{2i}(e^{iθ} - e^{-iθ}) \\
             \cos kθ &\eqnp{11}& \frac1 2  (e^{ikθ} + e^{-ikθ}) \\
             \sen kθ &\eqnp{12}& \frac1{2i}(e^{ikθ} - e^{-ikθ}) \\
                 e^{(α+βi)θ} &=& e^{αθ} e^{βiθ} \\
                             &=& e^{αθ} (\cos βθ + i \sen βθ) \\
                             &=& e^{αθ} \cos βθ + i e^α \sen βθ \\
                 e^{(α-βi)θ} &=& e^{αθ} e^{-βiθ} \\
                             &=& e^{αθ} (\cos(-βθ) + i \sen(-βθ)) \\
                             &=& e^{αθ} (\cos  βθ  - i \sen  βθ) \\
   e^{(α+βi)θ} + e^{(α-βi)θ} &=& e^{αθ} (e^{βiθ} + e^{-βiθ}) \\
                             &=& 2 \, e^{αθ} \cos βθ \\
   e^{(α+βi)θ} - e^{(α-βi)θ} &=& e^α (e^{βiθ} - e^{-βiθ}) \\
                             &=& 2i \, e^{αθ} \sen βθ \\
                 e^α \cos βθ &=& \frac12 e^{(α+βi)θ} + \frac12 e^{(α-βi)θ} \\
                 e^α \sen βθ &=& \frac1{2i} e^{(α+βi)θ} - \frac1{2i} e^{(α-βi)θ} \\
 \end{array}
$

}\anothercol{

$\begin{array}[t]{rcl}
                           E &=& e^{iθ} \\
                         c_α &=& \cos αθ \\
                         s_α &=& \sen αθ \\
                         E^α &=& (e^{iθ})^α \\
                             &=& e^{iαθ} \\
                             &=& c_α + is_α \\
                      E^{-α} &=& e^{-iαθ} \\
                             &=& c_{-α} + is_{-α} \\
                             &=& c_α - is_α \\
                         c_α &=& \frac1 2  (E^α + E^{-α}) \\
                         s_α &=& \frac1{2i}(E^α - E^{-α}) \\
                    E^{α+iβ} &=&  \\
   e^{(α+βi)θ} + e^{(α-βi)θ} &=& e^α (e^{βiθ} + e^{-βiθ}) \\
                             &=& 2 \, e^α \cos βθ \\
   e^{(α+βi)θ} - e^{(α-βi)θ} &=& e^α (e^{βiθ} - e^{-βiθ}) \\
                             &=& 2i \, e^α \sen βθ \\
                 e^α \cos βθ &=& \frac12 e^{(α+βi)θ} + \frac12 e^{(α-βi)θ} \\
                 e^α \sen βθ &=& \frac1{2i} e^{(α+βi)θ} - \frac1{2i} e^{(α-βi)θ} \\
 \end{array}
$

}}

\bsk

% $\ga{EDOLG=}$



% Lambda em Haskell
% Notação at
% Funções são conjuntos

% (lam181p 7 "function-graphs")
% (lam181a   "function-graphs")

\newpage

% «maxima»  (to ".maxima")
% (c2m222edolsp 4 "maxima")
% (c2m222edolsa   "maxima")
% (find-es "maxima" "EDOLCCs")


\GenericWarning{Success:}{Success!!!}  % Used by `M-x cv'

\end{document}

%  ____  _             _         
% |  _ \(_)_   ___   _(_)_______ 
% | | | | \ \ / / | | | |_  / _ \
% | |_| | |\ V /| |_| | |/ /  __/
% |____// | \_/  \__,_|_/___\___|
%     |__/                       
%
% «djvuize»  (to ".djvuize")
% (find-LATEXgrep "grep --color -nH --null -e djvuize 2020-1*.tex")

 (eepitch-shell)
 (eepitch-kill)
 (eepitch-shell)
# (find-fline "~/2022.2-C2/")
# (find-fline "~/LATEX/2022-2-C2/")
# (find-fline "~/bin/djvuize")

cd /tmp/
for i in *.jpg; do echo f $(basename $i .jpg); done

f () { rm -v $1.pdf;  textcleaner -f 50 -o  5 $1.jpg $1.png; djvuize $1.pdf; xpdf $1.pdf }
f () { rm -v $1.pdf;  textcleaner -f 50 -o 10 $1.jpg $1.png; djvuize $1.pdf; xpdf $1.pdf }
f () { rm -v $1.pdf;  textcleaner -f 50 -o 20 $1.jpg $1.png; djvuize $1.pdf; xpdf $1.pdf }

f () { rm -fv $1.png $1.pdf; djvuize $1.pdf }
f () { rm -fv $1.png $1.pdf; djvuize WHITEBOARDOPTS="-m 1.0 -f 15" $1.pdf; xpdf $1.pdf }
f () { rm -fv $1.png $1.pdf; djvuize WHITEBOARDOPTS="-m 1.0 -f 30" $1.pdf; xpdf $1.pdf }
f () { rm -fv $1.png $1.pdf; djvuize WHITEBOARDOPTS="-m 1.0 -f 45" $1.pdf; xpdf $1.pdf }
f () { rm -fv $1.png $1.pdf; djvuize WHITEBOARDOPTS="-m 0.5" $1.pdf; xpdf $1.pdf }
f () { rm -fv $1.png $1.pdf; djvuize WHITEBOARDOPTS="-m 0.25" $1.pdf; xpdf $1.pdf }
f () { cp -fv $1.png $1.pdf       ~/2022.2-C2/
       cp -fv        $1.pdf ~/LATEX/2022-2-C2/
       cat <<%%%
% (find-latexscan-links "C2" "$1")
%%%
}

f 20201213_area_em_funcao_de_theta
f 20201213_area_em_funcao_de_x
f 20201213_area_fatias_pizza



%  __  __       _        
% |  \/  | __ _| | _____ 
% | |\/| |/ _` | |/ / _ \
% | |  | | (_| |   <  __/
% |_|  |_|\__,_|_|\_\___|
%                        
% <make>

 (eepitch-shell)
 (eepitch-kill)
 (eepitch-shell)
# (find-LATEXfile "2019planar-has-1.mk")
make -f 2019.mk STEM=2022-2-C2-edos-lineares veryclean
make -f 2019.mk STEM=2022-2-C2-edos-lineares pdf

% Local Variables:
% coding: utf-8-unix
% ee-tla: "c2el"
% ee-tla: "c2m222edols"
% End:
