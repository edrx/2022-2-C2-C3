% (find-LATEX "2022-2-C2-VS.tex")
% (defun c () (interactive) (find-LATEXsh "lualatex -record 2022-2-C2-VS.tex" :end))
% (defun C () (interactive) (find-LATEXsh "lualatex 2022-2-C2-VS.tex" "Success!!!"))
% (defun D () (interactive) (find-pdf-page      "~/LATEX/2022-2-C2-VS.pdf"))
% (defun d () (interactive) (find-pdftools-page "~/LATEX/2022-2-C2-VS.pdf"))
% (defun e () (interactive) (find-LATEX "2022-2-C2-VS.tex"))
% (defun o () (interactive) (find-LATEX "2022-1-C2-VSA.tex"))
% (defun o () (interactive) (find-LATEX "2022-2-C2-P2.tex"))
% (defun o () (interactive) (find-LATEX "2022-2-C2-VR.tex"))
% (defun u () (interactive) (find-latex-upload-links "2022-2-C2-VS"))
% (defun v () (interactive) (find-2a '(e) '(d)))
% (defun d0 () (interactive) (find-ebuffer "2022-2-C2-VS.pdf"))
% (defun cv () (interactive) (C) (ee-kill-this-buffer) (v) (g))
%          (code-eec-LATEX "2022-2-C2-VS")
% (find-pdf-page   "~/LATEX/2022-2-C2-VS.pdf")
% (find-sh0 "cp -v  ~/LATEX/2022-2-C2-VS.pdf /tmp/")
% (find-sh0 "cp -v  ~/LATEX/2022-2-C2-VS.pdf /tmp/pen/")
%     (find-xournalpp "/tmp/2022-2-C2-VS.pdf")
%   file:///home/edrx/LATEX/2022-2-C2-VS.pdf
%               file:///tmp/2022-2-C2-VS.pdf
%           file:///tmp/pen/2022-2-C2-VS.pdf
% http://angg.twu.net/LATEX/2022-2-C2-VS.pdf
% (find-LATEX "2019.mk")
% (find-sh0 "cd ~/LUA/; cp -v Pict2e1.lua Pict2e1-1.lua Piecewise1.lua ~/LATEX/")
% (find-sh0 "cd ~/LUA/; cp -v Pict2e1.lua Pict2e1-1.lua Pict3D1.lua ~/LATEX/")
% (find-sh0 "cd ~/LUA/; cp -v C2Subst1.lua C2Formulas1.lua ~/LATEX/")
% (find-CN-aula-links "2022-2-C2-VS" "2" "c2m222vs" "c2vs")

% «.defs»		(to "defs")
% «.defs-T-and-B»	(to "defs-T-and-B")
% «.title»		(to "title")
% «.links»		(to "links")
% «.questao-1»		(to "questao-1")
% «.questao-1-maxima»	(to "questao-1-maxima")
% «.questao-2»		(to "questao-2")
% «.questao-2-maxima»	(to "questao-2-maxima")
% «.anexo-L»		(to "anexo-L")
% «.anexo-R»		(to "anexo-R")
% «.anexo»		(to "anexo")
%
% «.djvuize»		(to "djvuize")



% <videos>
% Video (not yet):
% (find-ssr-links     "c2m222vs" "2022-2-C2-VS")
% (code-eevvideo      "c2m222vs" "2022-2-C2-VS")
% (code-eevlinksvideo "c2m222vs" "2022-2-C2-VS")
% (find-c2m222vsvideo "0:00")

\documentclass[oneside,12pt]{article}
\usepackage[colorlinks,citecolor=DarkRed,urlcolor=DarkRed]{hyperref} % (find-es "tex" "hyperref")
\usepackage{amsmath}
\usepackage{amsfonts}
\usepackage{amssymb}
\usepackage{pict2e}
\usepackage[x11names,svgnames]{xcolor} % (find-es "tex" "xcolor")
\usepackage{colorweb}                  % (find-es "tex" "colorweb")
%\usepackage{tikz}
%
% (find-dn6 "preamble6.lua" "preamble0")
%\usepackage{proof}   % For derivation trees ("%:" lines)
%\input diagxy        % For 2D diagrams ("%D" lines)
%\xyoption{curve}     % For the ".curve=" feature in 2D diagrams
%
\usepackage{edrx21}               % (find-LATEX "edrx21.sty")
\input edrxaccents.tex            % (find-LATEX "edrxaccents.tex")
\input edrx21chars.tex            % (find-LATEX "edrx21chars.tex")
\input edrxheadfoot.tex           % (find-LATEX "edrxheadfoot.tex")
\input edrxgac2.tex               % (find-LATEX "edrxgac2.tex")
%\usepackage{emaxima}              % (find-LATEX "emaxima.sty")
%
% (find-es "tex" "geometry")
\usepackage[a6paper, landscape,
            top=1.5cm, bottom=.25cm, left=1cm, right=1cm, includefoot
           ]{geometry}
%
\begin{document}

\catcode`\^^J=10
\directlua{dofile "dednat6load.lua"}  % (find-LATEX "dednat6load.lua")
%%L dofile "Piecewise1.lua"           -- (find-LATEX "Piecewise1.lua")
%%L dofile "QVis1.lua"                -- (find-LATEX "QVis1.lua")
%%L dofile "Pict3D1.lua"              -- (find-LATEX "Pict3D1.lua")
%L dofile "C2Formulas1.lua"          -- (find-LATEX "C2Formulas1.lua")
%L dofile "2022-1-C2-P2.lua"         -- (find-LATEX "2022-1-C2-P2.lua")
%L Pict2e.__index.suffix = "%"
\pu
\def\pictgridstyle{\color{GrayPale}\linethickness{0.3pt}}
\def\pictaxesstyle{\linethickness{0.5pt}}
\def\pictnaxesstyle{\color{GrayPale}\linethickness{0.5pt}}
\celllower=2.5pt

% «defs»  (to ".defs")
% (find-LATEX "edrx21defs.tex" "colors")
% (find-LATEX "edrx21.sty")

\def\u#1{\par{\footnotesize \url{#1}}}

\def\drafturl{http://angg.twu.net/LATEX/2022-2-C2.pdf}
\def\drafturl{http://angg.twu.net/2022.2-C2.html}
\def\draftfooter{\tiny \href{\drafturl}{\jobname{}} \ColorBrown{\shorttoday{} \hours}}

\sa{[M]}{\CFname{M}{}}
\sa{[F]}{\CFname{F}{}}
\sa{[S]}{\CFname{S}{}}

\def\Smile{\emoji{slightly-smiling-face}}

% «defs-T-and-B»  (to ".defs-T-and-B")
\long\def\ColorOrange#1{{\color{orange!90!black}#1}}
\def\T(Total: #1 pts){{\bf(Total: #1)}}
\def\T(Total: #1 pts){{\bf(Total: #1 pts)}}
\def\T(Total: #1 pts){\ColorRed{\bf(Total: #1 pts)}}
\def\B       (#1 pts){\ColorOrange{\bf(#1 pts)}}



%  _____ _ _   _                               
% |_   _(_) |_| | ___   _ __   __ _  __ _  ___ 
%   | | | | __| |/ _ \ | '_ \ / _` |/ _` |/ _ \
%   | | | | |_| |  __/ | |_) | (_| | (_| |  __/
%   |_| |_|\__|_|\___| | .__/ \__,_|\__, |\___|
%                      |_|          |___/      
%
% «title»  (to ".title")
% (c2m222vsp 1 "title")
% (c2m222vsa   "title")

\thispagestyle{empty}

\begin{center}

\vspace*{1.2cm}

{\bf \Large Cálculo 2 - 2022.2}

\bsk

Prova suplementar (VS)

\bsk

Eduardo Ochs - RCN/PURO/UFF

\url{http://angg.twu.net/2022.2-C2.html}

\end{center}

\newpage

% «links»  (to ".links")

\newpage

%   ___                  _                _ 
%  / _ \ _   _  ___  ___| |_ __ _  ___   / |
% | | | | | | |/ _ \/ __| __/ _` |/ _ \  | |
% | |_| | |_| |  __/\__ \ || (_| | (_) | | |
%  \__\_\\__,_|\___||___/\__\__,_|\___/  |_|
%                                           
% «questao-1»  (to ".questao-1")
% (c2m222vsp 2 "questao-1")
% (c2m222vsa   "questao-1")
%
%L namedang("EDOVSintro", "", [[
%L    \begin{array}{rcl}
%L      \ga{[M]} &=& <EDOVSG> \\ \\[-5pt]
%L      \ga{[F]} &=& <EDOVSP> \\
%L    \end{array}
%L ]])
%L namedang("metodo", "", [[ <EDOVSG> ]])
%L metodo:sa("FOO"):output()
\pu

\scalebox{0.6}{\def\colwidth{9.5cm}\firstcol{

    {\bf \large Questão 1}

    \T(Total: 6.0 pts)
    
    \ssk

    A primeira EDO com variáveis separáveis (``EDOVS'') que nós vimos
    no curso foi $\frac{dy}{dx}=-\frac{x}{y}$. As soluções
    particulares dela eram pedaços das curvas de nível de $x^2+y^2=C$
    --- e esses pedaços eram ou semicírculos acima do eixo $x$ ou
    semicírculos abaixo do eixo $y$.

    \msk

    Na P2 eu pus uma questão sobre uma EDOVS um pouco mais complicada
    do que essa dos semicírculos, e muitas pessoas fizeram erros de
    conta horríveis {\sl que eu acho que foram causados por
      desorganização na hora de fazer as contas}... por exemplo,
    várias pessoas escreveram ``$H(x)=\sqrt{x}$'' num lugar das contas
    e ``$H(x)=\sqrt{x+3}$'' em outro --- o que OBVIAMENTE é um erro
    conceitual GRAVÍSSIMO, né, {\sl a menos que você explique em
      português direitinho que $H(x)$ vai ser $\sqrt{x}$ em um
      contexto e $\sqrt{x+3}$ em outro contexto separado...}

    \msk


}\anothercol{

  \vspace*{0.85cm}

  Seja $(*)$ esta EDOVS:
  % 
  $$\frac{dy}{dx} = \frac{4x^3}{4(y+3)^3} \qquad (*)
  $$

  Encontre as duas soluções gerais da EDO $(*)$ --- uma ``positiva'' e
  outra ``negativa'' ---, encontre as soluções particulares que passam
  pelos pontos $(1,-1)$ e $(1,-5)$, e teste tudo.

  \msk

  {\sl Nesta questão eu vou avaliar principalmente se você sabe usar
    direito os truques do anexo da página 4.} Ou seja, não vai bastar
  você usar o ``método'' para resolver EDOVSs, que é esse aqui:
  %
  $$\ga{[M]} \;=\; \scalebox{0.85}{$\ga{FOO}$}$$

}}


% «questao-1-maxima»  (to ".questao-1-maxima")
% (c2m222vsp 2 "questao-1-maxima")
% (c2m222vsa   "questao-1-maxima")
% (setq eepitch-preprocess-regexp "^")
% (setq eepitch-preprocess-regexp "^%T ")
%
%T  (eepitch-maxima)
%T  (eepitch-kill)
%T  (eepitch-maxima)
%T define(G(x), x^4);
%T define(H(y), (y+3)^4);
%T 
%T define(g(x), diff(G(x),x));
%T define(h(y), diff(H(y),y));
%T edo : dydx = g(x) / h(y);
%T 
%T eq_a   : x = H(y);
%T          solve(eq_a, y);
%T 
%T getsol(n,C31,x1,y1) := (
%T   eq_b   : solve(eq_a, y)[n],
%T   define(Hinv(x), rhs(eq_b)),
%T   eq_c   : y = Hinv(G(x) + C3),
%T   eq_d   : y = Hinv(G(x) + C31),
%T   define(f  (x), rhs(eq_c)),
%T   define(f1 (x), rhs(eq_d))
%T   );
%T 
%T getsol(4,15, 1,-1);
%T f(x);  
%T f1(x);
%T f1(1);
%T 
%T getsol(3,15, 1,-5);
%T f(x);  
%T f1(x);
%T f1(1);


\newpage

%   ___                  _                ____  
%  / _ \ _   _  ___  ___| |_ __ _  ___   |___ \ 
% | | | | | | |/ _ \/ __| __/ _` |/ _ \    __) |
% | |_| | |_| |  __/\__ \ || (_| | (_) |  / __/ 
%  \__\_\\__,_|\___||___/\__\__,_|\___/  |_____|
%                                               
% «questao-2»  (to ".questao-2")
% (c2m222vsp 3 "questao-2")
% (c2m222vsa   "questao-2")

\scalebox{0.7}{\def\colwidth{9cm}\firstcol{

{\bf \large Questão 2}

\T(Total: 4.0 pts)

\ssk

Calcule a integral abaixo usando pelo menos três mudanças de
variáveis.
%
$$\intx{ \frac{8e^{4x}\ln(e^{4x}+2)}{e^{4x}+2} }
$$

Obs: no curso nós vimos que qualquer integral que pode ser resolvida
por uma sequência de mudanças de variáveis também pode ser resolvida
por uma mudança de variável só, mas aqui é pra usar pelo menos três!

% «questao-2-maxima»  (to ".questao-2-maxima")
% (c2m222vsp 3 "questao-2-maxima")
% (c2m222vsa   "questao-2-maxima")
% (setq eepitch-preprocess-regexp "^")
% (setq eepitch-preprocess-regexp "^%T ")
%
%T  (eepitch-maxima)
%T  (eepitch-kill)
%T  (eepitch-maxima)
%T b(x) := exp(x);
%T c(x) := x^4 + 2;
%T d(x) := log(x)^2;
%T F : d(c(b(x)));
%T f : diff(F, x);
%T integrate(f, x);

}\anothercol{
}}



\newpage

%     _                         
%    / \   _ __   _____  _____  
%   / _ \ | '_ \ / _ \ \/ / _ \ 
%  / ___ \| | | |  __/>  < (_) |
% /_/   \_\_| |_|\___/_/\_\___/ 
%                               
% (c2m222p2p 5 "questao-1-gab")
% (c2m222p2a   "questao-1-gab")
% «anexo-L»  (to ".anexo-L")
\def\anexoL{

A substituição é:
%
$$\ga{[S]} \;=\;
  \bmat{
    G(x) := x^4 + 5 \\
    H(y) := y^2 + 3 \\
    g(x) := 4x^3 \\
    h(y) := 2y \\
    H^{-1}(x) := \sqrt{x-3} \\
  }
$$

a) Seja:
%
$$\frac{dy}{dx} = \frac{4x^3}{2y} \qquad (*)$$

b)
%
 $\begin{array}[t]{lrcl}
  \text{Seja:}  & H^{-1}(x) &=& \sqrt{x-3}. \\
  \text{Temos:} & H^{-1}(H(y)) &=& \sqrt{H(y)-3} \\
                &              &=& \sqrt{(y^2+3)-3} \\
                &              &=& y. \\
  \end{array}
 $

\msk

c) $\begin{array}[t]{lrcl}
       & y &=& H^{-1}(G(x)+C_3) \\
          &&=& \sqrt{(G(x)+C_3)-3} \\
          &&=& \sqrt{((x^4+5)+C_3)-3} \\
          &&=& \sqrt{x^4+2+C_3} \\
    \text{Seja:} &
      f(x) &=& \sqrt{x^4+2+C_3}. \\
    \end{array}
   $

}


% «anexo-R»  (to ".anexo-R")

\def\anexoR{

d) $\begin{array}[t]{l}
    \text{Será que $f(x)$ obedece $(*)$?} \\
    \text{Temos }
    f'(x) = \frac{2x^3}{\sqrt{x^4 + 2 + C_3}},
    \text{ e com isso:}
    \\
    \\[-5pt]
    \left(
      f'(x) = \frac{4x^3}{2f(x)}
    \right)
    \bmat{
      f(x) = \sqrt{x^4+2+C_3} \\
      f'(x) = \frac{2x^3}{\sqrt{x^4 + 2 + C_3}} \\
    }
    \\
    = \;\;
    \left(
      \frac{2x^3}{\sqrt{x^4 + 2 + C_3}}
      = \frac{4x^3}{2\sqrt{x^4+2+C_3}}
    \right)
    \qquad \smile \\
    \end{array}
   $

\bsk

e) $\begin{array}[t]{lrcl}
    \text{Se}    & f(x_1) &=& y_1, \\
    \text{i.e.,} & f(1)   &=& 2,   \\
    \text{então} & f(1)   &=& \sqrt{1^4+2+C_3} \\
                         &&=& \sqrt{3+C_3} \\
                         &&=& 2 \\
                 & 2^2    &=& \sqrt{3+C_3}^2 \\
                 & 4      &=&       3+C_3    \\
                 & C_3    &=& 1 \\
                 & f(x)   &=& \sqrt{x^4+2+C_3} \\
                 &        &=& \sqrt{x^4+3} \\
    \text{Seja:} & f_1(x) &=& \sqrt{x^4+3}. \\
    \end{array}
   $

\bsk

f) $\begin{array}[t]{lrcl}
    \text{Será que} & f_1(x_1) &=& y_1, \\
    \text{i.e.,}    & f_1(1)   &=& 2?   \\
                & \sqrt{1^4+3} &=& \sqrt{4} \\
                              &&=& 2 \qquad \smile \\
    \end{array}
   $

}

% «anexo»  (to ".anexo")

\scalebox{0.6}{\def\colwidth{9cm}\firstcol{

\vspace*{-0.5cm}

{\bf Anexo: gabarito de}

{\bf uma questão da P2}

\ssk

\anexoL

}\anothercol{

\anexoR

}}



\GenericWarning{Success:}{Success!!!}  % Used by `M-x cv'

\end{document}



%  ____  _             _         
% |  _ \(_)_   ___   _(_)_______ 
% | | | | \ \ / / | | | |_  / _ \
% | |_| | |\ V /| |_| | |/ /  __/
% |____// | \_/  \__,_|_/___\___|
%     |__/                       
%
% «djvuize»  (to ".djvuize")
% (find-LATEXgrep "grep --color -nH --null -e djvuize 2020-1*.tex")

 (eepitch-shell)
 (eepitch-kill)
 (eepitch-shell)
# (find-fline "~/2022.2-C2/")
# (find-fline "~/LATEX/2022-2-C2/")
# (find-fline "~/bin/djvuize")

cd /tmp/
for i in *.jpg; do echo f $(basename $i .jpg); done

f () { rm -v $1.pdf;  textcleaner -f 50 -o  5 $1.jpg $1.png; djvuize $1.pdf; xpdf $1.pdf }
f () { rm -v $1.pdf;  textcleaner -f 50 -o 10 $1.jpg $1.png; djvuize $1.pdf; xpdf $1.pdf }
f () { rm -v $1.pdf;  textcleaner -f 50 -o 20 $1.jpg $1.png; djvuize $1.pdf; xpdf $1.pdf }

f () { rm -fv $1.png $1.pdf; djvuize $1.pdf }
f () { rm -fv $1.png $1.pdf; djvuize WHITEBOARDOPTS="-m 1.0 -f 15" $1.pdf; xpdf $1.pdf }
f () { rm -fv $1.png $1.pdf; djvuize WHITEBOARDOPTS="-m 1.0 -f 30" $1.pdf; xpdf $1.pdf }
f () { rm -fv $1.png $1.pdf; djvuize WHITEBOARDOPTS="-m 1.0 -f 45" $1.pdf; xpdf $1.pdf }
f () { rm -fv $1.png $1.pdf; djvuize WHITEBOARDOPTS="-m 0.5" $1.pdf; xpdf $1.pdf }
f () { rm -fv $1.png $1.pdf; djvuize WHITEBOARDOPTS="-m 0.25" $1.pdf; xpdf $1.pdf }
f () { cp -fv $1.png $1.pdf       ~/2022.2-C2/
       cp -fv        $1.pdf ~/LATEX/2022-2-C2/
       cat <<%%%
% (find-latexscan-links "C2" "$1")
%%%
}

f 20201213_area_em_funcao_de_theta
f 20201213_area_em_funcao_de_x
f 20201213_area_fatias_pizza



%  __  __       _        
% |  \/  | __ _| | _____ 
% | |\/| |/ _` | |/ / _ \
% | |  | | (_| |   <  __/
% |_|  |_|\__,_|_|\_\___|
%                        
% <make>

 (eepitch-shell)
 (eepitch-kill)
 (eepitch-shell)
# (find-LATEXfile "2019planar-has-1.mk")
make -f 2019.mk STEM=2022-2-C2-VS veryclean
make -f 2019.mk STEM=2022-2-C2-VS pdf

% Local Variables:
% coding: utf-8-unix
% ee-tla: "c2vs"
% ee-tla: "c2m222vs"
% End:
