% (find-LATEX "2022-2-C2-VR.tex")
% (defun c () (interactive) (find-LATEXsh "lualatex -record 2022-2-C2-VR.tex" :end))
% (defun C () (interactive) (find-LATEXsh "lualatex 2022-2-C2-VR.tex" "Success!!!"))
% (defun D () (interactive) (find-pdf-page      "~/LATEX/2022-2-C2-VR.pdf"))
% (defun d () (interactive) (find-pdftools-page "~/LATEX/2022-2-C2-VR.pdf"))
% (defun e () (interactive) (find-LATEX "2022-2-C2-VR.tex"))
% (defun o () (interactive) (find-LATEX "2022-1-C2-VR.tex"))
% (defun o () (interactive) (find-LATEX "2022-2-C2-P2.tex"))
% (defun u () (interactive) (find-latex-upload-links "2022-2-C2-VR"))
% (defun v () (interactive) (find-2a '(e) '(d)))
% (defun d0 () (interactive) (find-ebuffer "2022-2-C2-VR.pdf"))
% (defun cv () (interactive) (C) (ee-kill-this-buffer) (v) (g))
%          (code-eec-LATEX "2022-2-C2-VR")
% (find-pdf-page   "~/LATEX/2022-2-C2-VR.pdf")
% (find-sh0 "cp -v  ~/LATEX/2022-2-C2-VR.pdf /tmp/")
% (find-sh0 "cp -v  ~/LATEX/2022-2-C2-VR.pdf /tmp/pen/")
%     (find-xournalpp "/tmp/2022-2-C2-VR.pdf")
%   file:///home/edrx/LATEX/2022-2-C2-VR.pdf
%               file:///tmp/2022-2-C2-VR.pdf
%           file:///tmp/pen/2022-2-C2-VR.pdf
% http://angg.twu.net/LATEX/2022-2-C2-VR.pdf
% (find-LATEX "2019.mk")
% (find-sh0 "cd ~/LUA/; cp -v Pict2e1.lua Pict2e1-1.lua Piecewise1.lua ~/LATEX/")
% (find-sh0 "cd ~/LUA/; cp -v Pict2e1.lua Pict2e1-1.lua Pict3D1.lua ~/LATEX/")
% (find-sh0 "cd ~/LUA/; cp -v C2Subst1.lua C2Formulas1.lua ~/LATEX/")
% (find-CN-aula-links "2022-2-C2-VR" "2" "c2m222vr" "c2vr")

% «.defs»		(to "defs")
% «.defs-T-and-B»	(to "defs-T-and-B")
% «.title»		(to "title")
% «.links»		(to "links")
% «.questao-2»		(to "questao-2")
%
% «.djvuize»		(to "djvuize")



% <videos>
% Video (not yet):
% (find-ssr-links     "c2m222vr" "2022-2-C2-VR")
% (code-eevvideo      "c2m222vr" "2022-2-C2-VR")
% (code-eevlinksvideo "c2m222vr" "2022-2-C2-VR")
% (find-c2m222vrvideo "0:00")

\documentclass[oneside,12pt]{article}
\usepackage[colorlinks,citecolor=DarkRed,urlcolor=DarkRed]{hyperref} % (find-es "tex" "hyperref")
\usepackage{amsmath}
\usepackage{amsfonts}
\usepackage{amssymb}
\usepackage{pict2e}
\usepackage[x11names,svgnames]{xcolor} % (find-es "tex" "xcolor")
\usepackage{colorweb}                  % (find-es "tex" "colorweb")
%\usepackage{tikz}
%
% (find-dn6 "preamble6.lua" "preamble0")
%\usepackage{proof}   % For derivation trees ("%:" lines)
%\input diagxy        % For 2D diagrams ("%D" lines)
%\xyoption{curve}     % For the ".curve=" feature in 2D diagrams
%
\usepackage{edrx21}               % (find-LATEX "edrx21.sty")
\input edrxaccents.tex            % (find-LATEX "edrxaccents.tex")
\input edrx21chars.tex            % (find-LATEX "edrx21chars.tex")
\input edrxheadfoot.tex           % (find-LATEX "edrxheadfoot.tex")
\input edrxgac2.tex               % (find-LATEX "edrxgac2.tex")
%\usepackage{emaxima}              % (find-LATEX "emaxima.sty")
%
% (find-es "tex" "geometry")
\usepackage[a6paper, landscape,
            top=1.5cm, bottom=.25cm, left=1cm, right=1cm, includefoot
           ]{geometry}
%
\begin{document}

\catcode`\^^J=10
\directlua{dofile "dednat6load.lua"}  % (find-LATEX "dednat6load.lua")
%L dofile "Piecewise1.lua"           -- (find-LATEX "Piecewise1.lua")
%%L dofile "QVis1.lua"                -- (find-LATEX "QVis1.lua")
%%L dofile "Pict3D1.lua"              -- (find-LATEX "Pict3D1.lua")
%L dofile "C2Formulas1.lua"          -- (find-LATEX "C2Formulas1.lua")
%L dofile "2022-1-C2-P2.lua"         -- (find-LATEX "2022-1-C2-P2.lua")
%L Pict2e.__index.suffix = "%"
\pu
\def\pictgridstyle{\color{GrayPale}\linethickness{0.3pt}}
\def\pictaxesstyle{\linethickness{0.5pt}}
\def\pictnaxesstyle{\color{GrayPale}\linethickness{0.5pt}}
\celllower=2.5pt

% «defs»  (to ".defs")
% (find-LATEX "edrx21defs.tex" "colors")
% (find-LATEX "edrx21.sty")

\def\u#1{\par{\footnotesize \url{#1}}}

\def\drafturl{http://angg.twu.net/LATEX/2022-2-C2.pdf}
\def\drafturl{http://angg.twu.net/2022.2-C2.html}
\def\draftfooter{\tiny \href{\drafturl}{\jobname{}} \ColorBrown{\shorttoday{} \hours}}

\sa{[M]}{\CFname{M}{}}
\sa{[F]}{\CFname{F}{}}

% «defs-T-and-B»  (to ".defs-T-and-B")
\long\def\ColorOrange#1{{\color{orange!90!black}#1}}
\def\T(Total: #1 pts){{\bf(Total: #1)}}
\def\T(Total: #1 pts){{\bf(Total: #1 pts)}}
\def\T(Total: #1 pts){\ColorRed{\bf(Total: #1 pts)}}
\def\B       (#1 pts){\ColorOrange{\bf(#1 pts)}}



%  _____ _ _   _                               
% |_   _(_) |_| | ___   _ __   __ _  __ _  ___ 
%   | | | | __| |/ _ \ | '_ \ / _` |/ _` |/ _ \
%   | | | | |_| |  __/ | |_) | (_| | (_| |  __/
%   |_| |_|\__|_|\___| | .__/ \__,_|\__, |\___|
%                      |_|          |___/      
%
% «title»  (to ".title")
% (c2m222vrp 1 "title")
% (c2m222vra   "title")

\thispagestyle{empty}

\begin{center}

\vspace*{1.2cm}

{\bf \Large Cálculo 2 - 2022.2}

\bsk

Prova de reposição (VR)

\bsk

Eduardo Ochs - RCN/PURO/UFF

\url{http://angg.twu.net/2022.2-C2.html}

\end{center}

\newpage

% «links»  (to ".links")
% (c2m222p1p 1 "questao-1")
% (c2m222p1a   "questao-1")


\scalebox{0.6}{\def\colwidth{9cm}\firstcol{

{\bf Questão 1}

\T(Total: 4.0 pts)

\msk

Calcule:

$$\intx{x^3 \sqrt{1-4x^2}}\;.$$

\bsk

{\bf Dicas:} 1) Você provavelmente vai precisar de pelo menos duas
mudanças de variável pra chegar no resultado final. 2) No curso nós
vimos dois modos de fazer mudanças de variável de um jeito legível: um
modo usava chaves sob subexpressões e o outro modo usava ``caixinhas
de anotações'' como a abaixo,
%
$$\bmat{
  u = \sen x \\
  \frac{du}{dx} = \frac{d}{dx} \sen x = \cos x \\
  \cos x \, dx = du \\
  x = \arcsen u \\
  }
$$

em que todas as outras linhas da caixinha eram consequência da
primeira.

}\anothercol{

% «questao-2»       (to ".questao-2")
% «int-por-partes»  (to ".int-por-partes")
% (c2m222p1p 2 "int-por-partes")
% (c2m222p1a   "int-por-partes")
% (c2m222ippp 1 "title")
% (c2m222ippa   "title")

{\bf Questão 2}

\T(Total: 1.0 pts)

\msk

Calcule:
%
$$\Intx{5}{6}{(2x+3)^4} 
$$

Lembre que no curso nós vimos que existem várias noções diferentes do
que é ``simplificar uma expressão''... por exemplo, no ensino médio os
professores exigem que a gente ``simplifique''
$\frac{1}{11} + \frac{1}{12} + \frac{1}{13}$ pra $\frac{661}{3036}$,
mas em Cálculo 2 a gente geralmente considera
$\frac{1}{11} + \frac{1}{12} + \frac{1}{13}$ mais ``simples'' do que
$\frac{661}{3036}$.



}}



\newpage


% «questao-2»  (to ".questao-2")

\newpage

% (c2m222vrp 2 "questao-2")
% (c2m222vra   "questao-2")
% (c2m222p2p 2 "questao-1")
% (c2m222p2a   "questao-1")

{\bf Questão 3}

%L namedang("EDOVSintro", "", [[
%L    \begin{array}{rcl}
%L      \ga{[M]} &=& <EDOVSG> \\ \\[-5pt]
%L      \ga{[F]} &=& <EDOVSP> \\
%L    \end{array}
%L ]])
%L EDOVSintro:sa("FOO"):output()
\pu

\scalebox{0.55}{\def\colwidth{10cm}\firstcol{

\vspace*{-0.4cm}

\T(Total: 5.0 pts)

Lembre que nós vimos que o ``método'' para resolver EDOs com variáveis
separáveis --- ``EDOVSs'' --- pode ser escrito como a demonstração
$\ga{[M]}$ abaixo, e a ``fórmula'' para resolver EDOVSs pode ser
escrita como $\ga{[F]}$:

\bsk

$\ga{FOO}$

}\anothercol{

  Quando a gente quer criar exercícios de EDOVSs que sejam fácil de
  resolver a gente começa escolhendo $G(x)$ e $H(y)$, não $g(x)$ e
  $h(y)$.

  Digamos que $G(x)=x^4+5$ e $H(y)=y^2+3$.

  \msk

  a) \B (0.5 pts) Diga qual é a EDO da forma
  $\frac{dy}{dx} = \frac{g(x)}{h(y)}$ associada a esta escolha de
  $G(x)$ e $H(y)$. Chame-a de $(*)$. Não esqueça do ``Seja''!

  \ssk

  b) \B (0.5 pts) Escolha uma função $H^{-1}$ adequada. Defina ela com
  um ``Seja'' e verifique que ela obedece o que esperamos dela.

  \ssk

  c) \B (1.0 pts) Encontre a solução geral da EDO $(*)$. Chame-a de
  $f(x)$ e defina ela com um ``Seja''.

  \ssk

  d) \B (1.0 pts) Verifique que essa função $f(x)$ obedece $(*)$.

  \ssk

  e) \B (1.0 pts) Encontre uma solução da EDO $(*)$ que passe pelo
  ponto $(x_1,y_1)=(1,2)$. Chame-a de $f_1(x)$ e defina-a com um
  ``Seja''.

  \ssk

  f) \B (1.0 pts) Verifique que a sua $f_1(x)$ realmente passa pelo
  ponto $(x_1,y_1)$.



% (find-es "maxima" "2022-2-C2-P2")

}}



\GenericWarning{Success:}{Success!!!}  % Used by `M-x cv'

\end{document}

%  ____  _             _         
% |  _ \(_)_   ___   _(_)_______ 
% | | | | \ \ / / | | | |_  / _ \
% | |_| | |\ V /| |_| | |/ /  __/
% |____// | \_/  \__,_|_/___\___|
%     |__/                       
%
% «djvuize»  (to ".djvuize")
% (find-LATEXgrep "grep --color -nH --null -e djvuize 2020-1*.tex")

 (eepitch-shell)
 (eepitch-kill)
 (eepitch-shell)
# (find-fline "~/2022.2-C2/")
# (find-fline "~/LATEX/2022-2-C2/")
# (find-fline "~/bin/djvuize")

cd /tmp/
for i in *.jpg; do echo f $(basename $i .jpg); done

f () { rm -v $1.pdf;  textcleaner -f 50 -o  5 $1.jpg $1.png; djvuize $1.pdf; xpdf $1.pdf }
f () { rm -v $1.pdf;  textcleaner -f 50 -o 10 $1.jpg $1.png; djvuize $1.pdf; xpdf $1.pdf }
f () { rm -v $1.pdf;  textcleaner -f 50 -o 20 $1.jpg $1.png; djvuize $1.pdf; xpdf $1.pdf }

f () { rm -fv $1.png $1.pdf; djvuize $1.pdf }
f () { rm -fv $1.png $1.pdf; djvuize WHITEBOARDOPTS="-m 1.0 -f 15" $1.pdf; xpdf $1.pdf }
f () { rm -fv $1.png $1.pdf; djvuize WHITEBOARDOPTS="-m 1.0 -f 30" $1.pdf; xpdf $1.pdf }
f () { rm -fv $1.png $1.pdf; djvuize WHITEBOARDOPTS="-m 1.0 -f 45" $1.pdf; xpdf $1.pdf }
f () { rm -fv $1.png $1.pdf; djvuize WHITEBOARDOPTS="-m 0.5" $1.pdf; xpdf $1.pdf }
f () { rm -fv $1.png $1.pdf; djvuize WHITEBOARDOPTS="-m 0.25" $1.pdf; xpdf $1.pdf }
f () { cp -fv $1.png $1.pdf       ~/2022.2-C2/
       cp -fv        $1.pdf ~/LATEX/2022-2-C2/
       cat <<%%%
% (find-latexscan-links "C2" "$1")
%%%
}

f 20201213_area_em_funcao_de_theta
f 20201213_area_em_funcao_de_x
f 20201213_area_fatias_pizza



%  __  __       _        
% |  \/  | __ _| | _____ 
% | |\/| |/ _` | |/ / _ \
% | |  | | (_| |   <  __/
% |_|  |_|\__,_|_|\_\___|
%                        
% <make>

 (eepitch-shell)
 (eepitch-kill)
 (eepitch-shell)
# (find-LATEXfile "2019planar-has-1.mk")
make -f 2019.mk STEM=2022-2-C2-VR veryclean
make -f 2019.mk STEM=2022-2-C2-VR pdf

% Local Variables:
% coding: utf-8-unix
% ee-tla: "c2vr"
% ee-tla: "c2m222vr"
% End:
