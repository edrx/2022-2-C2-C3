% (find-LATEX "2022-2-C2-mudanca-de-variaveis.tex")
% (defun c () (interactive) (find-LATEXsh "lualatex -record 2022-2-C2-mudanca-de-variaveis.tex" :end))
% (defun C () (interactive) (find-LATEXsh "lualatex 2022-2-C2-mudanca-de-variaveis.tex" "Success!!!"))
% (defun D () (interactive) (find-pdf-page      "~/LATEX/2022-2-C2-mudanca-de-variaveis.pdf"))
% (defun d () (interactive) (find-pdftools-page "~/LATEX/2022-2-C2-mudanca-de-variaveis.pdf"))
% (defun e () (interactive) (find-LATEX "2022-2-C2-mudanca-de-variaveis.tex"))
% (defun o () (interactive) (find-LATEX "2022-2-C2-mudanca-de-variaveis.tex"))
% (defun u () (interactive) (find-latex-upload-links "2022-2-C2-mudanca-de-variaveis"))
% (defun v () (interactive) (find-2a '(e) '(d)))
% (defun d0 () (interactive) (find-ebuffer "2022-2-C2-mudanca-de-variaveis.pdf"))
% (defun cv () (interactive) (C) (ee-kill-this-buffer) (v) (g))
%          (code-eec-LATEX "2022-2-C2-mudanca-de-variaveis")
% (find-pdf-page   "~/LATEX/2022-2-C2-mudanca-de-variaveis.pdf")
% (find-sh0 "cp -v  ~/LATEX/2022-2-C2-mudanca-de-variaveis.pdf /tmp/")
% (find-sh0 "cp -v  ~/LATEX/2022-2-C2-mudanca-de-variaveis.pdf /tmp/pen/")
%     (find-xournalpp "/tmp/2022-2-C2-mudanca-de-variaveis.pdf")
%   file:///home/edrx/LATEX/2022-2-C2-mudanca-de-variaveis.pdf
%               file:///tmp/2022-2-C2-mudanca-de-variaveis.pdf
%           file:///tmp/pen/2022-2-C2-mudanca-de-variaveis.pdf
% http://angg.twu.net/LATEX/2022-2-C2-mudanca-de-variaveis.pdf
% (find-LATEX "2019.mk")
% (find-sh0 "cd ~/LUA/; cp -v Pict2e1.lua Pict2e1-1.lua Piecewise1.lua ~/LATEX/")
% (find-sh0 "cd ~/LUA/; cp -v Pict2e1.lua Pict2e1-1.lua Pict3D1.lua ~/LATEX/")
% (find-sh0 "cd ~/LUA/; cp -v C2Subst1.lua C2Formulas1.lua ~/LATEX/")
% (find-CN-aula-links "2022-2-C2-mudanca-de-variaveis" "2" "c2m222mv" "c2mv")

% «.defs»		(to "defs")
% «.title»		(to "title")
% «.sas»		(to "sas")
% «.links»		(to "links")
% «.uma-conta-com-mv»	(to "uma-conta-com-mv")
% «.justificando-cada»	(to "justificando-cada")
% «.exercicio-1»	(to "exercicio-1")
% «.exercicio-2»	(to "exercicio-2")
%
% «.djvuize»		(to "djvuize")



% <videos>
% Video (not yet):
% (find-ssr-links     "c2m222mv" "2022-2-C2-mudanca-de-variaveis")
% (code-eevvideo      "c2m222mv" "2022-2-C2-mudanca-de-variaveis")
% (code-eevlinksvideo "c2m222mv" "2022-2-C2-mudanca-de-variaveis")
% (find-c2m222mvvideo "0:00")

\documentclass[oneside,12pt]{article}
\usepackage[colorlinks,citecolor=DarkRed,urlcolor=DarkRed]{hyperref} % (find-es "tex" "hyperref")
\usepackage{amsmath}
\usepackage{amsfonts}
\usepackage{amssymb}
\usepackage{pict2e}
\usepackage[x11names,svgnames]{xcolor} % (find-es "tex" "xcolor")
\usepackage{colorweb}                  % (find-es "tex" "colorweb")
%\usepackage{tikz}
%
% (find-dn6 "preamble6.lua" "preamble0")
%\usepackage{proof}   % For derivation trees ("%:" lines)
%\input diagxy        % For 2D diagrams ("%D" lines)
%\xyoption{curve}     % For the ".curve=" feature in 2D diagrams
%
\usepackage{edrx21}               % (find-LATEX "edrx21.sty")
\input edrxaccents.tex            % (find-LATEX "edrxaccents.tex")
\input edrx21chars.tex            % (find-LATEX "edrx21chars.tex")
\input edrxheadfoot.tex           % (find-LATEX "edrxheadfoot.tex")
\input edrxgac2.tex               % (find-LATEX "edrxgac2.tex")
%\usepackage{emaxima}              % (find-LATEX "emaxima.sty")
%
%\usepackage[backend=biber,
%   style=alphabetic]{biblatex}            % (find-es "tex" "biber")
%\addbibresource{catsem-slides.bib}        % (find-LATEX "catsem-slides.bib")
%
% (find-es "tex" "geometry")
\usepackage[a6paper, landscape,
            top=1.5cm, bottom=.25cm, left=1cm, right=1cm, includefoot
           ]{geometry}
%
\begin{document}

\catcode`\^^J=10
\directlua{dofile "dednat6load.lua"}  % (find-LATEX "dednat6load.lua")
%L dofile "Piecewise1.lua"           -- (find-LATEX "Piecewise1.lua")
%L dofile "QVis1.lua"                -- (find-LATEX "QVis1.lua")
%L dofile "Pict3D1.lua"              -- (find-LATEX "Pict3D1.lua")
%L dofile "C2Formulas1.lua"          -- (find-LATEX "C2Formulas1.lua")
%L Pict2e.__index.suffix = "%"
\pu
\def\pictgridstyle{\color{GrayPale}\linethickness{0.3pt}}
\def\pictaxesstyle{\linethickness{0.5pt}}
\def\pictnaxesstyle{\color{GrayPale}\linethickness{0.5pt}}
\celllower=2.5pt

% «defs»  (to ".defs")
% (find-LATEX "edrx21defs.tex" "colors")
% (find-LATEX "edrx21.sty")

\def\u#1{\par{\footnotesize \url{#1}}}

\def\drafturl{http://angg.twu.net/LATEX/2022-2-C2.pdf}
\def\drafturl{http://angg.twu.net/2022.2-C2.html}
\def\draftfooter{\tiny \href{\drafturl}{\jobname{}} \ColorBrown{\shorttoday{} \hours}}



%  _____ _ _   _                               
% |_   _(_) |_| | ___   _ __   __ _  __ _  ___ 
%   | | | | __| |/ _ \ | '_ \ / _` |/ _` |/ _ \
%   | | | | |_| |  __/ | |_) | (_| | (_| |  __/
%   |_| |_|\__|_|\___| | .__/ \__,_|\__, |\___|
%                      |_|          |___/      
%
% «title»  (to ".title")
% (c2m222mvp 1 "title")
% (c2m222mva   "title")

\thispagestyle{empty}

\begin{center}

\vspace*{1.2cm}

{\bf \Large Cálculo 2 - 2022.2}

\bsk

Aula 10: Mudança de variáveis

(e integrais de potências de senos e cossenos)

\bsk

Eduardo Ochs - RCN/PURO/UFF

\url{http://angg.twu.net/2022.2-C2.html}

\end{center}

\newpage

%  ___              _       
% ( ) \   ___  __ _( )  ___ 
%  \|\ \ / __|/ _` |/  / __|
%     \ \\__ \ (_| |   \__ \
%      \_\___/\__,_|   |___/
%                           
% «sas»  (to ".sas")

\def\P    #1{\left(#1\right)}
\def\und#1#2{\underbrace{#1}_{#2}}
\def\uu   #1{\und{#1}{u}}
\def\ududx#1{\und{#1}{\frac{du}{dx}}}
\def\udu  #1{\und{#1}{du}}
\def\intxu#1#2{\int #1 \udu{\ududx{#2}\,dx}}

\sa{[MVI]}{\CFname{MVI}{}}
\sa{(MVI)}{\P {\ga{MVI}}}
\sa {MVI} {{\D \intx{f'(g(x))g'(x)} \;=\; \intu{f'(u)}}}

\sa{[MVA]}{\CFname{MVA}{}}
\sa{(MVA)}{\P {\ga{MVA}}}
\sa {MVA} {
  \begin{array}{rcl}
  \D \intxu {f'( \uu{g(x)} )}{g'(x)}
    &=& \D \intu{f'(u)} \\[-25pt]
    &=& \D f(u)         \\
    &=& \D f(g(x))      \\
  \end{array}
  }

\sa{(MVA sen)}{\P{\ga{MVA sen}}}
\sa {MVA sen} {
  \begin{array}{rcl}
  \D \intxu {\cos( \uu{x^2} )·}{2x}
    &=& \D \intu{\cos(u)} \\[-25pt]
    &=& \D \sen(u)        \\
    &=& \D \sen(x^2)      \\
  \end{array}
  }

\sa{S1}{\bmat{
     f(x) := \sen x \\
    f'(x) := \cos x \\
     g(x) := x^2 \\
    g'(x) := 2x \\
  }}


\newpage

% «links»  (to ".links")
% (c2m222mvp 2 "links")
% (c2m222mva   "links")
% (c2m221atisp 1 "title")
% (c2m221atisa   "title")
% (c2m221atisp 12 "substituicao-figura")
% (c2m221atisa    "substituicao-figura")
% (c2m221atisp 14 "exemplo-contas")
% (c2m221atisa    "exemplo-contas")
% (c2m221atisp 16 "exemplo-contas-2")
% (c2m221atisa    "exemplo-contas-2")
% (c2m221vsbp 8 "questao-3-gab")
% (c2m221vsba   "questao-3-gab")
% (find-books "__analysis/__analysis.el" "leithold")
% (find-books "__analysis/__analysis.el" "leithold" "5.2.1. Regra da cadeia")
% (find-books "__analysis/__analysis.el" "leithold" "9.2" "potências de seno e co-seno")
% (find-books "__analysis/__analysis.el" "thomas")
% (find-books "__analysis/__analysis.el" "thomas" "5: The substitution rule")
% (find-books "__analysis/__analysis.el" "thomas" "5: The substitution rule" "Example 3")
% (find-books "__analysis/__analysis.el" "thomas" "Substitution in definite integrals")
% (find-books "__analysis/__analysis.el" "miranda")
% (find-books "__analysis/__analysis.el" "miranda" "6.2 Integração por Substituição")
% (find-books "__analysis/__analysis.el" "miranda" "Exemplo 6.6")
% (find-books "__analysis/__analysis.el" "miranda" "8.3 Integrais Trigonométricas")
% (find-pdf-page "/home/angg_slow_html/2020.2-C2/thomas_secoes_5.5_e_5.6.pdf")
% http://angg.twu.net/2020.2-C2/thomas_secoes_5.5_e_5.6.pdf
% (find-fline "/home/angg_slow_html/eev-videos/" "2020_int_subst_1.mp4")
% (find-LATEX "2020-1-C2-int-subst.tex" "videos" "2020_int_subst_1")

{\bf Links}


\scalebox{0.7}{\def\colwidth{14cm}\firstcol{

    A mudança de variáveis na integral {\sl indefinida} (``MVI'') é
    uma gambiarra que eu até hoje ainda não sei interpretar
    geometricamente.

\msk

A mudança de variáveis na integral {\sl definida} (``MVD'') -- que nós
vamos ver depois! -- é fácil de interpretar geometricamente: ela muda
os limites de integração. Veja esta figura aqui:

\ssk

{\footnotesize

% (c2m221atisp 12 "substituicao-figura")
% (c2m221atisa    "substituicao-figura")
%    http://angg.twu.net/LATEX/2022-1-C2-algumas-t-ints.pdf#page=12
\url{http://angg.twu.net/LATEX/2022-1-C2-algumas-t-ints.pdf\#page=12}

}

\msk

Nós vamos tentar entender a MVI pela seção 5.2.1 do Leithold, pela
seção 6.2 do Miranda e pela seção 5.5 do Thomas. Links:

\ssk

{\footnotesize

% (find-dmirandacalcpage 189 "6.2 Integração por Substituição")
% http://hostel.ufabc.edu.br/~daniel.miranda/calculo/calculo.pdf#page=192
\url{http://hostel.ufabc.edu.br/~daniel.miranda/calculo/calculo.pdf\#page=192}

% http://angg.twu.net/2020.2-C2/thomas_secoes_5.5_e_5.6.pdf
% (find-pdf-page "/home/angg_slow_html/2020.2-C2/thomas_secoes_5.5_e_5.6.pdf")
\url{http://angg.twu.net/2020.2-C2/thomas_secoes_5.5_e_5.6.pdf}

}

\msk

O Miranda explica como integrar potências de senos e cossenos na seção
8.3:

\ssk

{\footnotesize

% (find-dmirandacalcpage 255 "8.3 Integrais Trigonométricas")
% http://hostel.ufabc.edu.br/~daniel.miranda/calculo/calculo.pdf#page=255
\url{http://hostel.ufabc.edu.br/~daniel.miranda/calculo/calculo.pdf\#page=255}

}

\msk

Veja também:

\ssk

{\footnotesize

% (c2m221atisp 16 "um-exemplo")
% (c2m221atisa    "um-exemplo")
%    http://angg.twu.net/LATEX/2022-1-C2-algumas-t-ints.pdf#page=16
\url{http://angg.twu.net/LATEX/2022-1-C2-algumas-t-ints.pdf\#page=16}

% (c2m221atisp 39 "gambiarras")
% (c2m221atisa    "gambiarras")
%    http://angg.twu.net/LATEX/2022-1-C2-algumas-t-ints.pdf#page=39
\url{http://angg.twu.net/LATEX/2022-1-C2-algumas-t-ints.pdf\#page=39}

% (c2m221vsbp 8 "questao-3-gab")
% (c2m221vsba   "questao-3-gab")
%    http://angg.twu.net/LATEX/2022-1-C2-VSB.pdf#page=8
\url{http://angg.twu.net/LATEX/2022-1-C2-VSB.pdf\#page=8}

% (find-extra-file-links "/sda5/videos/Math/Matematica_Calculo_2_-_Aula_05_-_Regra_da_Substituicao-PTCUjrEBc4g.webm" "rsubs")
% (code-video "rsubsvideo" "/sda5/videos/Math/Matematica_Calculo_2_-_Aula_05_-_Regra_da_Substituicao-PTCUjrEBc4g.webm")
% (find-rsubsvideo "0:00")
% (find-rsubsvideo "4:35")
\url{http://www.youtube.com/watch?v=PTCUjrEBc4g\#t=4m35s} (até 10:00)

}

}\anothercol{
}}




\newpage

% «uma-conta-com-mv»  (to ".uma-conta-com-mv")
% (c2m222mvp 3 "uma-conta-com-mv")
% (c2m222mva   "uma-conta-com-mv")

{\bf Uma conta com mudança de variáveis}

\vspace*{-0.3cm}

\scalebox{0.7}{\def\colwidth{12cm}\firstcol{

\ga{[MVI]}: mudança de variáveis na integral indefinida.

\ga{[MVA]}: uma aplicação típica do \ga{[MVI]} (caso geral).

$\ga{[MVA]}\,[\ldots]$: uma aplicação típica do \ga{[MVI]} (caso particular).

\ssk

Compare com o Teorema 5 e o Exemplo 3 daqui:

{\footnotesize

% (find-pdf-page "/home/angg_slow_html/2020.2-C2/thomas_secoes_5.5_e_5.6.pdf")
%    http://angg.twu.net/2020.2-C2/thomas_secoes_5.5_e_5.6.pdf
\url{http://angg.twu.net/2020.2-C2/thomas_secoes_5.5_e_5.6.pdf}

}

$$\begin{array}{rcl}
    \ga{[MVI]}         &=& \ga{(MVI)} \\ \\[-5pt]
    \ga{[MVA]}         &=& \ga{(MVA)} \\ \\[-5pt]
    \ga{[MVA]} \ga{S1} &=& \ga{(MVA sen)} \\
  \end{array}
$$

}\anothercol{
}}



\newpage

% «justificando-cada»  (to ".justificando-cada")
% (c2m222mvp 4 "justificando-cada")
% (c2m222mva   "justificando-cada")

{\bf Justificando cada igualdade}

% do \ga{[MVA]}

\sa{MVA H}{
   \D \intx{f'(g(x))g'(x)}
   \;=\;
   \D \intu{f'(u)}
   \;=\;
   f(u)
   \;=\;
   f(g(x))
  }
\sa{MVA Hund}{
   \und{ \D\ddx\P{ \intx{f'(g(x))g'(x)}} }{f'(g(x))g'(x)}
   \;=\;
   \und{ \D\ddx \und{\und{\intu{f'(u)}}{f(u)}}{f(g(x))} }{f'(g(x))g'(x)}
   \;=\;
   \und{ \D\ddx \und{f(u)}{f(g(x))} }{f'(g(x))g'(x)}
   \;=\;
   \und{ \D\ddx f(g(x)) }{f'(g(x))g'(x)}
  }


\scalebox{0.7}{\def\colwidth{12cm}\firstcol{

Quando os livros dizem ``$u=g(x)$'' eles não dizem claramente

em quais lugares podemos substituir $u$ por $g(x)$...

Por exemplo, será que podemos fazer isto aqui?
%
$$\intu{f'(u)} \;=\; \intu{f'(g(x))}$$

Eu \ColorRed{acho} que não, mas até hoje eu não sei as regras exatas...

\ColorRed{Me parece} que o melhor modo de justificar estas três igualdades
%
$$\begin{array}{c}
  \ga{MVA H} \\
  \end{array}
$$

é assim:
%
$$\begin{array}{c}
  \ga{MVA Hund} \\
  \end{array}
$$

}\anothercol{
}}


\newpage

% «exercicio-1»  (to ".exercicio-1")
% (c2m222mvp 4 "exercicio-1")
% (c2m222mva   "exercicio-1")

{\bf Exercício 1.}

Leia a seção 6.2 do Miranda. Ela começa na página 189:

\ssk

{\scriptsize

% (find-dmirandacalcpage 189 "6.2 Integração por Substituição")
% (find-dmirandacalcpage 192   "Exemplo 6.6")
% (find-dmirandacalcpage 196   "Exercícios")
% http://hostel.ufabc.edu.br/~daniel.miranda/calculo/calculo.pdf#page=189
\url{http://hostel.ufabc.edu.br/~daniel.miranda/calculo/calculo.pdf\#page=189}

}

\ssk

Entenda bem o exemplo 6.6 da página 192 dele.

Faça os exercícios de 1 a 13 das páginas 196 e 197,

exceto pelos exercícios que envolvem secantes.

\msk

Dê uma olhada no meu modo preferido de não me enrolar

na mudança de variáveis, que é o método das ``caixinhas

de anotações'', explicado aqui, nas páginas 39--44,

\ssk

{\scriptsize

% (c2m221atisp 39 "gambiarras")
% (c2m221atisa    "gambiarras")
%    http://angg.twu.net/LATEX/2022-1-C2-algumas-t-ints.pdf#page=39
\url{http://angg.twu.net/LATEX/2022-1-C2-algumas-t-ints.pdf\#page=39}

}

\ssk

e tente usá-lo.


\newpage

% «exercicio-2»  (to ".exercicio-2")
% (c2m222mvp 6 "exercicio-2")
% (c2m222mva   "exercicio-2")
% (c2m221atisp 16 "exemplo-contas-2")
% (c2m221atisa    "exemplo-contas-2")
% (find-es "maxima" "int-pow-sin-cos")

{\bf Exercício 2.}

Entenda bem os exemplos 1, 2 e 3 da seção 8.3 do Miranda,

\ssk

{\scriptsize

% (find-books "__analysis/__analysis.el" "miranda" "8.3 Integrais Trigonométricas")
% (find-dmirandacalcpage 255 "8.3 Integrais Trigonométricas")
% http://hostel.ufabc.edu.br/~daniel.miranda/calculo/calculo.pdf#page=255
\url{http://hostel.ufabc.edu.br/~daniel.miranda/calculo/calculo.pdf\#page=255}

}

\ssk

e as ``caixinhas de anotações'':

\ssk

{\footnotesize

% (c2m221atisp 12 "substituicao-figura")
% (c2m221atisa    "substituicao-figura")
%    http://angg.twu.net/LATEX/2022-1-C2-algumas-t-ints.pdf#page=39
\url{http://angg.twu.net/LATEX/2022-1-C2-algumas-t-ints.pdf\#page=39}

%    http://angg.twu.net/LATEX/2022-1-C2-algumas-t-ints.pdf#page=16
\url{http://angg.twu.net/LATEX/2022-1-C2-algumas-t-ints.pdf\#page=16}

}

\msk

Depois disso resolva estas três integrais:

\ssk

a) $\intx{(\cos x)^4 \sen x}$

b) $\intx{(\cos x)^4 (\sen x)^3}$

c) $\intx{(\cos x)^4 (\sen x)^5}$


% (find-books "__analysis/__analysis.el" "miranda")
% (find-books "__analysis/__analysis.el" "miranda" "6.2 Integração por Substituição")
% (find-books "__analysis/__analysis.el" "miranda" "Exemplo 6.6")




%\printbibliography

\GenericWarning{Success:}{Success!!!}  % Used by `M-x cv'

\end{document}

%  ____  _             _         
% |  _ \(_)_   ___   _(_)_______ 
% | | | | \ \ / / | | | |_  / _ \
% | |_| | |\ V /| |_| | |/ /  __/
% |____// | \_/  \__,_|_/___\___|
%     |__/                       
%
% «djvuize»  (to ".djvuize")
% (find-LATEXgrep "grep --color -nH --null -e djvuize 2020-1*.tex")

 (eepitch-shell)
 (eepitch-kill)
 (eepitch-shell)
# (find-fline "~/2022.2-C2/")
# (find-fline "~/LATEX/2022-2-C2/")
# (find-fline "~/bin/djvuize")

cd /tmp/
for i in *.jpg; do echo f $(basename $i .jpg); done

f () { rm -v $1.pdf;  textcleaner -f 50 -o  5 $1.jpg $1.png; djvuize $1.pdf; xpdf $1.pdf }
f () { rm -v $1.pdf;  textcleaner -f 50 -o 10 $1.jpg $1.png; djvuize $1.pdf; xpdf $1.pdf }
f () { rm -v $1.pdf;  textcleaner -f 50 -o 20 $1.jpg $1.png; djvuize $1.pdf; xpdf $1.pdf }

f () { rm -fv $1.png $1.pdf; djvuize $1.pdf }
f () { rm -fv $1.png $1.pdf; djvuize WHITEBOARDOPTS="-m 1.0 -f 15" $1.pdf; xpdf $1.pdf }
f () { rm -fv $1.png $1.pdf; djvuize WHITEBOARDOPTS="-m 1.0 -f 30" $1.pdf; xpdf $1.pdf }
f () { rm -fv $1.png $1.pdf; djvuize WHITEBOARDOPTS="-m 1.0 -f 45" $1.pdf; xpdf $1.pdf }
f () { rm -fv $1.png $1.pdf; djvuize WHITEBOARDOPTS="-m 0.5" $1.pdf; xpdf $1.pdf }
f () { rm -fv $1.png $1.pdf; djvuize WHITEBOARDOPTS="-m 0.25" $1.pdf; xpdf $1.pdf }
f () { cp -fv $1.png $1.pdf       ~/2022.2-C2/
       cp -fv        $1.pdf ~/LATEX/2022-2-C2/
       cat <<%%%
% (find-latexscan-links "C2" "$1")
%%%
}

f 20201213_area_em_funcao_de_theta
f 20201213_area_em_funcao_de_x
f 20201213_area_fatias_pizza



%  __  __       _        
% |  \/  | __ _| | _____ 
% | |\/| |/ _` | |/ / _ \
% | |  | | (_| |   <  __/
% |_|  |_|\__,_|_|\_\___|
%                        
% <make>

 (eepitch-shell)
 (eepitch-kill)
 (eepitch-shell)
# (find-LATEXfile "2019planar-has-1.mk")
make -f 2019.mk STEM=2022-2-C2-mudanca-de-variaveis veryclean
make -f 2019.mk STEM=2022-2-C2-mudanca-de-variaveis pdf

% Local Variables:
% coding: utf-8-unix
% ee-tla: "c2mv"
% ee-tla: "c2m222mv"
% End:
