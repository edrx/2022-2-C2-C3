% (find-LATEX "2022-2-C2-TFC1-e-TFC2.tex")
% (defun c () (interactive) (find-LATEXsh "lualatex -record 2022-2-C2-TFC1-e-TFC2.tex" :end))
% (defun C () (interactive) (find-LATEXsh "lualatex 2022-2-C2-TFC1-e-TFC2.tex" "Success!!!"))
% (defun D () (interactive) (find-pdf-page      "~/LATEX/2022-2-C2-TFC1-e-TFC2.pdf"))
% (defun d () (interactive) (find-pdftools-page "~/LATEX/2022-2-C2-TFC1-e-TFC2.pdf"))
% (defun e () (interactive) (find-LATEX "2022-2-C2-TFC1-e-TFC2.tex"))
% (defun o () (interactive) (find-LATEX "2022-2-C2-TFC1-e-TFC2.tex"))
% (defun u () (interactive) (find-latex-upload-links "2022-2-C2-TFC1-e-TFC2"))
% (defun v () (interactive) (find-2a '(e) '(d)))
% (defun d0 () (interactive) (find-ebuffer "2022-2-C2-TFC1-e-TFC2.pdf"))
% (defun cv () (interactive) (C) (ee-kill-this-buffer) (v) (g))
%          (code-eec-LATEX "2022-2-C2-TFC1-e-TFC2")
% (find-pdf-page   "~/LATEX/2022-2-C2-TFC1-e-TFC2.pdf")
% (find-sh0 "cp -v  ~/LATEX/2022-2-C2-TFC1-e-TFC2.pdf /tmp/")
% (find-sh0 "cp -v  ~/LATEX/2022-2-C2-TFC1-e-TFC2.pdf /tmp/pen/")
%     (find-xournalpp "/tmp/2022-2-C2-TFC1-e-TFC2.pdf")
%   file:///home/edrx/LATEX/2022-2-C2-TFC1-e-TFC2.pdf
%               file:///tmp/2022-2-C2-TFC1-e-TFC2.pdf
%           file:///tmp/pen/2022-2-C2-TFC1-e-TFC2.pdf
% http://angg.twu.net/LATEX/2022-2-C2-TFC1-e-TFC2.pdf
% (find-LATEX "2019.mk")
% (find-sh0 "cd ~/LUA/; cp -v Pict2e1.lua Pict2e1-1.lua Piecewise1.lua ~/LATEX/")
% (find-sh0 "cd ~/LUA/; cp -v Pict2e1.lua Pict2e1-1.lua Pict3D1.lua ~/LATEX/")
% (find-sh0 "cd ~/LUA/; cp -v C2Subst1.lua C2Formulas1.lua ~/LATEX/")
% (find-CN-aula-links "2022-2-C2-TFC1-e-TFC2" "2" "c2m222tfcs" "c2tf")

% «.defs»		(to "defs")
% «.title»		(to "title")
% «.links»		(to "links")
% «.def-particao»	(to "def-particao")
% «.def-inf-e-sup»	(to "def-inf-e-sup")
% «.def-integral»	(to "def-integral")
% «.exercicio-2»	(to "exercicio-2")
% «.exercicio-3»	(to "exercicio-3")
% «.exercicio-4»	(to "exercicio-4")
%



% <videos>
% Video (not yet):
% (find-ssr-links     "c2m222tfcs" "2022-2-C2-TFC1-e-TFC2")
% (code-eevvideo      "c2m222tfcs" "2022-2-C2-TFC1-e-TFC2")
% (code-eevlinksvideo "c2m222tfcs" "2022-2-C2-TFC1-e-TFC2")
% (find-c2m222tfcsvideo "0:00")

\documentclass[oneside,12pt]{article}
\usepackage[colorlinks,citecolor=DarkRed,urlcolor=DarkRed]{hyperref} % (find-es "tex" "hyperref")
\usepackage{amsmath}
\usepackage{amsfonts}
\usepackage{amssymb}
\usepackage{pict2e}
\usepackage[x11names,svgnames]{xcolor} % (find-es "tex" "xcolor")
\usepackage{colorweb}                  % (find-es "tex" "colorweb")
%\usepackage{tikz}
%
% (find-dn6 "preamble6.lua" "preamble0")
%\usepackage{proof}   % For derivation trees ("%:" lines)
%\input diagxy        % For 2D diagrams ("%D" lines)
%\xyoption{curve}     % For the ".curve=" feature in 2D diagrams
%
\usepackage{edrx21}               % (find-LATEX "edrx21.sty")
\input edrxaccents.tex            % (find-LATEX "edrxaccents.tex")
\input edrx21chars.tex            % (find-LATEX "edrx21chars.tex")
\input edrxheadfoot.tex           % (find-LATEX "edrxheadfoot.tex")
\input edrxgac2.tex               % (find-LATEX "edrxgac2.tex")
%\usepackage{emaxima}              % (find-LATEX "emaxima.sty")
%
%\usepackage[backend=biber,
%   style=alphabetic]{biblatex}            % (find-es "tex" "biber")
%\addbibresource{catsem-slides.bib}        % (find-LATEX "catsem-slides.bib")
%
% (find-es "tex" "geometry")
\usepackage[a6paper, landscape,
            top=1.5cm, bottom=.25cm, left=1cm, right=1cm, includefoot
           ]{geometry}
%
\begin{document}

\catcode`\^^J=10
\directlua{dofile "dednat6load.lua"}  % (find-LATEX "dednat6load.lua")
%L dofile "Piecewise1.lua"           -- (find-LATEX "Piecewise1.lua")
%L dofile "QVis1.lua"                -- (find-LATEX "QVis1.lua")
%L dofile "Pict3D1.lua"              -- (find-LATEX "Pict3D1.lua")
%L dofile "C2Formulas1.lua"          -- (find-LATEX "C2Formulas1.lua")
%L Pict2e.__index.suffix = "%"
\pu
\def\pictgridstyle{\color{GrayPale}\linethickness{0.3pt}}
\def\pictaxesstyle{\linethickness{0.5pt}}
\def\pictnaxesstyle{\color{GrayPale}\linethickness{0.5pt}}
\celllower=2.5pt

% «defs»  (to ".defs")
% (find-LATEX "edrx21defs.tex" "colors")
% (find-LATEX "edrx21.sty")

\def\u#1{\par{\footnotesize \url{#1}}}

\def\drafturl{http://angg.twu.net/LATEX/2022-2-C2.pdf}
\def\drafturl{http://angg.twu.net/2022.2-C2.html}
\def\draftfooter{\tiny \href{\drafturl}{\jobname{}} \ColorBrown{\shorttoday{} \hours}}

% (find-LATEX "edrxgac2.tex" "C2")

\def\Rext{\overline{\R}}

\def\into{\overline ∫}
\def\intu{\underline∫}
\def\intou{\overline{\underline∫}}
\def\INTx#1#2#3#4{#1_{x=#2}^{x=#3} #4 \, dx}
\def\INTP  #1#2#3{#1_{#2}          #3 \, dx}

\def\mname#1{\ensuremath{[\text{#1}]}}
\def\minf{\mname{inf}}
\def\msup{\mname{sup}}
\def\sse {\text{sse}}

\sa{into_P     f(x) dx}{\INTP{\into} {P}{f(x)}}
\sa{intu_P     f(x) dx}{\INTP{\intu} {P}{f(x)}}
\sa{intou_P    f(x) dx}{\INTP{\intou}{P}{f(x)}}
\sa{into_ab2k  f(x) dx}{\INTP{\into} {[a,b]_{2^k}}{f(x)}}
\sa{intu_ab2k  f(x) dx}{\INTP{\intu} {[a,b]_{2^k}}{f(x)}}
\sa{intou_ab2k f(x) dx}{\INTP{\intou}{[a,b]_{2^k}}{f(x)}}
\sa{into_xab   f(x) dx}{\INTx{\into} {a}{b}{f(x)}}
\sa{intu_xab   f(x) dx}{\INTx{\intu} {a}{b}{f(x)}}
\sa{intou_xab  f(x) dx}{\INTx{\intou}{a}{b}{f(x)}}
\sa{int_xab    f(x) dx}{\INTx{\int}  {a}{b}{f(x)}}




%  _____ _ _   _                               
% |_   _(_) |_| | ___   _ __   __ _  __ _  ___ 
%   | | | | __| |/ _ \ | '_ \ / _` |/ _` |/ _ \
%   | | | | |_| |  __/ | |_) | (_| | (_| |  __/
%   |_| |_|\__|_|\___| | .__/ \__,_|\__, |\___|
%                      |_|          |___/      
%
% «title»  (to ".title")
% (c2m222tfcsp 1 "title")
% (c2m222tfcsa   "title")

\thispagestyle{empty}

\begin{center}

\vspace*{1.2cm}

{\bf \Large Cálculo 2 - 2022.2}

\bsk

Aula 19: o TFC1 e o TFC2.

\bsk

Eduardo Ochs - RCN/PURO/UFF

\url{http://angg.twu.net/2022.2-C2.html}

\end{center}

\newpage

% % «links»  (to ".links")
% % (c2m222tfcsp 2 "links")
% % (c2m222tfcsa   "links")
% 
% Infs e sups
% % (c2m222srp 25 "na-semana-academica")
% % (c2m222sra    "na-semana-academica")
% 
% Partições
% % (c2m212somas1p 9 "particoes")
% % (c2m212somas1a   "particoes")
% 
% Partição preferida
% 
% Partições cada vez mais finas
% % (c2m202escadasp 4 "exercicio-2")
% % (c2m202escadas    "exercicio-2")
% 
% Definição da integral
% % (c2m221isp 19 "definicao-integral")
% % (c2m221isa    "definicao-integral")
% 
% Funções de Dirichlet (a original e a diagonal)
% % (c2m212somas2p 51 "dirichlet")
% % (c2m212somas2a    "dirichlet")
% 
% TFC1 no vídeo do Mathologer (localizar)
% % (find-TH "mathologer-calculus-easy" "legendas")
% 
% TFC2 no vídeo do Mathologer (localizar)
% 
% Revisão de derivadas laterais
% % (c2m221tfc1p 11 "exercicio-4-dicas")
% % (c2m221tfc1a    "exercicio-4-dicas" "laterais")
% 
% Quando os TFCs não valem
% 
% Quais livros usam ai e bi? Quais usam $Δx$?
% 
% Animacoes sobre def integral e TFC1
% 
% % (find-books "__analysis/__analysis.el" "leithold" "5.5. A integral definida")
% % (find-books "__analysis/__analysis.el" "martins-martins" "Integral Definida")
% % (find-books "__analysis/__analysis.el" "miranda" "7.2. Integral definida")
% (find-books "__analysis/__analysis.el" "thomas" "5.3 The definite integral")

\newpage

Os próximos 3 slides são uma versão melhorada (?)

das definições de partições, de inf e sup, e de

integral definida destes PDFs antigos:

\ssk

Partições:

{\scriptsize

% (c2m212somas1p 9 "particoes")
% (c2m212somas1a   "particoes")
%    http://angg.twu.net/LATEX/2021-2-C2-somas-1.pdf#page=9
\url{http://angg.twu.net/LATEX/2021-2-C2-somas-1.pdf\#page=9} (até a p.12)

}

\ssk


Infs e sups:

{\scriptsize

% (c2m221isp 2 "uma-figura")
% (c2m221isa   "uma-figura")
%    http://angg.twu.net/LATEX/2022-1-C2-infs-e-sups.pdf#page=2
\url{http://angg.twu.net/LATEX/2022-1-C2-infs-e-sups.pdf\#page=2} (até a p.15)

}

\ssk

Integral definida como limite -- definições:

{\scriptsize

% (c2m221isp 16 "aproximacoes-por-cima")
% (c2m221isa    "aproximacoes-por-cima")
%    http://angg.twu.net/LATEX/2022-1-C2-infs-e-sups.pdf#page=16
\url{http://angg.twu.net/LATEX/2022-1-C2-infs-e-sups.pdf\#page=16} (até a p.21)

% (c2m212somas2p 35 "exercicio-13")
% (c2m212somas2a    "exercicio-13")
%    http://angg.twu.net/LATEX/2021-2-C2-somas-2.pdf#page=35
\url{http://angg.twu.net/LATEX/2021-2-C2-somas-2.pdf\#page=35} -- $[a,b]_n$

% (c2m212somas1p 16 "exercicio-9-dicas")
% (c2m212somas1a    "exercicio-9-dicas")
% http://angg.twu.net/LATEX/2021-1-C2-somas-1.pdf#page=16
\url{http://angg.twu.net/LATEX/2021-1-C2-somas-1.pdf\#page=16} -- simplificações

}

\ssk

Integral definida como limite -- uma animação:

{\scriptsize

% (c2m221tfc1p 35 "descontinuidades-2")
% (c2m221tfc1a    "descontinuidades-2")
%    http://angg.twu.net/LATEX/2022-1-C2-TFC1.pdf#page=35
\url{http://angg.twu.net/LATEX/2022-1-C2-TFC1.pdf\#page=35} (até a p.41)

}



\newpage

% «def-particao»  (to ".def-particao")
% (c2m222tfcsp 3 "def-particao")
% (c2m222tfcsa   "def-particao")

{\bf A definição de partição}

\scalebox{0.85}{\def\colwidth{11cm}\firstcol{

Se $P$ é um subconjunto \ColorRed{finito} e \ColorRed{não-vazio} de $\R$,

então podemos interpretar $P$ como uma partição...

Por exemplo, se $P=\{200,20,42,99,63,33,20,20\}$

então $P=\{20,33,42,63,99,200\}$, e aí vamos interpretar

esse conjunto de 6 pontos -- ordenados em ordem crescente --

como uma partição do intervalo $I = [a,b] = [20,200]$ em

5 subintervalos (``$N=5$''), assim:

$$\begin{array}{ccccccl}
  20 & 33 & 42 & 63 & 99 & 200 \\
  x_0 & x_1 & x_2 & x_3 & x_4 & x_5 \\
  a_1 & b_1 &     &     &     &     & I_1=[a_1,b_1] \\
      & a_2 & b_2 &     &     &     & I_2=[a_2,b_2] \\
      &     & a_3 & b_3 &     &     & I_3=[a_3,b_3] \\
      &     &     & a_4 & b_4 &     & I_4=[a_4,b_4] \\
      &     &     &     & a_5 & b_5 & I_5=[a_5,b_5] \\
   a  &     &     &     &     &  b  & I  = [a,b] = [x_0,x_N]\\
  \end{array}
$$

}\anothercol{
}}


\newpage

% «def-inf-e-sup»  (to ".def-inf-e-sup")
% (c2m222tfcsp 4 "def-inf-e-sup")
% (c2m222tfcsa   "def-inf-e-sup")
% (c2m221isp 3 "algumas-definicoes")
% (c2m221isa   "algumas-definicoes")

{\bf As definições de inf e sup}

\scalebox{0.9}{\def\colwidth{10cm}\firstcol{

Digamos que $f:\R→\R$ e $B⊂\R$.

Vamos definir $\inf(f(B))$ e $\sup(f(B))$ ---

e também $\inf(D)$ e $\sup(D)$, pra $D⊂\R$ ---

desta forma:
%
$$\begin{array}{rcl}
  \Rext &=& \R∪\{-∞,+∞\} \\
  C  &=& \setofst{(x,f(x))}{x∈B} \\
  D  &=& \setofst{f(x)}{x∈B} \\
  D' &=& \setofst{y∈\R}{∃x∈B.\ f(x)=y} \\
  L &=& \setofst{y∈\Rext}{∀d∈D.\;y≤d} \\
  U &=& \setofst{y∈\Rext}{∀d∈D.\;d≤y} \\
  (α=\inf(D)) &=& α∈L ∧ (∀ℓ∈L.\;ℓ \le α) \\
  (β=\sup(D)) &=& β∈U ∧ (∀u∈U.\;β \le u) \\
  \end{array}
$$

Com isto podemos definir a integral definida.

A definição formal dela está na próxima página.

}\anothercol{
}}

\newpage

% «def-integral»  (to ".def-integral")
% (c2m222tfcsp 5 "def-integral")
% (c2m222tfcsa   "def-integral")

\vspace*{-0.25cm}

$$\scalebox{0.44}{$
  \begin{array}{rcl}
  [a,b]_N &=& \setofst{a+k(\frac{b-a}{N})}{k∈\{0,\ldots,N\}} \\
          &=& \{ a+0(\frac{b-a}{N}),
              \; a+1(\frac{b-a}{N}),
              \; \ldots,
              \; a+N(\frac{b-a}{N}) \} \\
          &=& \{ a,
              \; a + \frac{b-a}{N},
              \; a + 2\frac{b-a}{N},
              \; a + 3\frac{b-a}{N},
              \; \ldots, \; b\} \\
  \D \ga{into_P  f(x) dx} &=&    \msup_P \\[-5pt]
                          &=& \D \sum_{i=1}^{n} \sup(f([a_i,b_i])) (b_i-a_i) \\
  \D \ga{intu_P  f(x) dx} &=&    \minf_P \\[-5pt]
                          &=& \D \sum_{i=1}^{n} \inf(f([a_i,b_i])) (b_i-a_i) \\ \\[-5pt]
  \D \ga{intou_P f(x) dx} &=& \D \INTP{\into}{P}{f(x)}
                               - \INTP{\intu}{P}{f(x)} \\ \\[-5pt]
  \D \ga{into_xab  f(x) dx} &=& \D \lim_{k→∞} \ga{into_ab2k f(x) dx} \\
  \D \ga{intu_xab  f(x) dx} &=& \D \lim_{k→∞} \ga{intu_ab2k f(x) dx} \\ \\[-5pt]
  \D \ga{intou_xab f(x) dx} &=& \D \ga{into_xab f(x) dx}
                                 - \ga{intu_xab f(x) dx} \\ \\[-5pt]
  \D \left( \ga{int_xab f(x) dx} \text{\;\;existe} \right)
                             &=& \D \left( \ga{into_xab  f(x) dx}
                                         = \ga{intu_xab  f(x) dx} \right)     \\ \\[-7pt]
                             &=& \D \left( \ga{intou_xab f(x) dx} = 0 \right) \\ \\[-7pt]
  \D \ga{int_xab f(x) dx}    &=& \D \ga{into_xab f(x) dx}
                                 \qquad \text{(se a integral existir)} \\  \\[-7pt]
                             &=& \D \ga{intu_xab f(x) dx}
                                 \qquad \text{(se a integral existir)} \\
  \end{array}
  $}
$$


\newpage

{\bf Exercício 1.}

Leia isto aqui:

\ssk

{\scriptsize

% (c2m212somas1p 9 "particoes")
% (c2m212somas1a   "particoes")
%    http://angg.twu.net/LATEX/2021-2-C2-somas-1.pdf#page=9
\url{http://angg.twu.net/LATEX/2021-2-C2-somas-1.pdf\#page=9} (até a p.12)

}

\msk

a) Seja $P=\{4,2,1,1.5\}$.

Interprete $P$ como uma partição.

Diga quem são o $N$, o $a$ e o $b$ dela e monte

a tabela dos subintervalos dela (p.10 do link acima).

\msk

b) Seja $P=[2,4]_6$.

Diga quem são os pontos da partição $P$.

\msk

c) Seja $P=[2,5]_{2^3}$.

Diga quem são os pontos da partição $P$.



\newpage

% «exercicio-2»  (to ".exercicio-2")
% (c2m222tfcsp 6 "exercicio-2")
% (c2m222tfcsa    "exercicio-2")
% (c2m221isp 12 "exercicio-5")
% (c2m221isa    "exercicio-5")

{\bf Exercício 2.}

%L Pict2e.bounds = PictBounds.new(v(0,0), v(9,7))
%L spec   = "(0,3)--(2,1)o (2,3)c (2,5)o--(7,0)"
%L pws    = PwSpec.from(spec)
%L curve  = pws:topict()
%L p = PictList { curve:prethickness("2pt") }
%L p:addputstrat(v(2.7,5.5), "\\cell{(2,5)}")
%L p:addputstrat(v(7.7,0.5), "\\cell{(7,0)}")
%L p:pgat("pgatc"):preunitlength("17pt"):sa("Exercicio 2"):output()
\pu

\msk

Sejam
%
$f(x) = \scalebox{0.5}{$\ga{Exercicio 2}$}$

e $B=[1,3]$.

\msk

Represente graficamente B, $f(B)$ e os conjuntos abaixo:
%
$$\begin{array}{rcl}

  % \Rext &=& \R∪\{-∞,+∞\} \\
  C  &=& \setofst{(x,f(x))}{x∈B} \\
  D  &=& \setofst{f(x)}{x∈B} \\
  D' &=& \setofst{y∈\R}{∃x∈B.\ f(x)=y} \\
  L &=& \setofst{y∈\Rext}{∀d∈D.\;y≤d} \\
  U &=& \setofst{y∈\Rext}{∀d∈D.\;d≤y} \\
  % (α=\inf(D)) &=& α∈L ∧ (∀ℓ∈L.\;ℓ \le α) \\
  % (β=\sup(D)) &=& β∈U ∧ (∀u∈U.\;β \le u) \\
  \end{array}
$$


\newpage

% «exercicio-3»  (to ".exercicio-3")
% (c2m222tfcsp 8 "exercicio-3")
% (c2m222tfcsa   "exercicio-3")

{\bf Exercício 3.}

\scalebox{0.9}{\def\colwidth{9cm}\firstcol{

Sejam
%
$f(x) = \scalebox{0.5}{$\ga{Exercicio 2}$}$ \; .

e $P=\{1,3,4,5\}$.

\msk

Represente graficamente:

\msk

a) $\ga{into_P  f(x) dx}$

\msk

b) $\ga{intu_P  f(x) dx}$

\msk

c) $\ga{intou_P f(x) dx}$

\msk

d) $\INTP{\intou}{[1,5]_2}{f(x)}$

\msk

e) $\INTP{\intou}{[1,5]_4}{f(x)}$

}\anothercol{
}}


\newpage

% «exercicio-4»  (to ".exercicio-4")
% (c2m222tfcsp 9 "exercicio-4")
% (c2m222tfcsa   "exercicio-4")

{\bf Exercício 4.}


\scalebox{0.9}{\def\colwidth{12cm}\firstcol{

Dê uma olhada nesses 4 slides sobre a função de Dirichlet:

\ssk

{\scriptsize

% (c2m212somas2p 51 "dirichlet")
% (c2m212somas2a    "dirichlet")
%    http://angg.twu.net/LATEX/2021-2-C2-somas-2.pdf#page=51
\url{http://angg.twu.net/LATEX/2021-2-C2-somas-2.pdf\#page=51} (até a p.54)

}

\ssk

Sejam:
%
$$\begin{array}{rcl}
  f(x) &=&
  \begin{cases}
     0 & \text{quando $x∈\Q$}, \\
     1 & \text{quando $x∈\R∖\Q$} \\
  \end{cases} \; , \\
  g(x) &=& f(x) + x. \\
  \end{array}
$$

Represente graficamente:

\msk

a) $g(x)$

b) $\INTP{\intou}{[0,4]_1}{g(x)}$

c) $\INTP{\intou}{[0,4]_2}{g(x)}$

d) $\INTP{\intou}{[0,4]_4}{g(x)}$

e) $\INTP{\intou}{[0,4]_8}{g(x)}$

}\anothercol{
}}


% (c2m212somas2p 51 "dirichlet")
% (c2m212somas2a    "dirichlet")





\GenericWarning{Success:}{Success!!!}  % Used by `M-x cv'

\end{document}

%  __  __       _        
% |  \/  | __ _| | _____ 
% | |\/| |/ _` | |/ / _ \
% | |  | | (_| |   <  __/
% |_|  |_|\__,_|_|\_\___|
%                        
% <make>

 (eepitch-shell)
 (eepitch-kill)
 (eepitch-shell)
# (find-LATEXfile "2019planar-has-1.mk")
make -f 2019.mk STEM=2022-2-C2-TFC1-e-TFC2 veryclean
make -f 2019.mk STEM=2022-2-C2-TFC1-e-TFC2 pdf

% Local Variables:
% coding: utf-8-unix
% ee-tla: "c2tf"
% ee-tla: "c2m222tfcs"
% End:
