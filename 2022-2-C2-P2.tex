% (find-LATEX "2022-2-C2-P2.tex")
% (defun c () (interactive) (find-LATEXsh "lualatex -record 2022-2-C2-P2.tex" :end))
% (defun C () (interactive) (find-LATEXsh "lualatex 2022-2-C2-P2.tex" "Success!!!"))
% (defun D () (interactive) (find-pdf-page      "~/LATEX/2022-2-C2-P2.pdf"))
% (defun d () (interactive) (find-pdftools-page "~/LATEX/2022-2-C2-P2.pdf"))
% (defun e () (interactive) (find-LATEX "2022-2-C2-P2.tex"))
% (defun o () (interactive) (find-LATEX "2022-1-C2-P2.tex"))
% (defun u () (interactive) (find-latex-upload-links "2022-2-C2-P2"))
% (defun v () (interactive) (find-2a '(e) '(d)))
% (defun d0 () (interactive) (find-ebuffer "2022-2-C2-P2.pdf"))
% (defun cv () (interactive) (C) (ee-kill-this-buffer) (v) (g))
%          (code-eec-LATEX "2022-2-C2-P2")
% (find-pdf-page   "~/LATEX/2022-2-C2-P2.pdf")
% (find-sh0 "cp -v  ~/LATEX/2022-2-C2-P2.pdf /tmp/")
% (find-sh0 "cp -v  ~/LATEX/2022-2-C2-P2.pdf /tmp/pen/")
%     (find-xournalpp "/tmp/2022-2-C2-P2.pdf")
%   file:///home/edrx/LATEX/2022-2-C2-P2.pdf
%               file:///tmp/2022-2-C2-P2.pdf
%           file:///tmp/pen/2022-2-C2-P2.pdf
% http://angg.twu.net/LATEX/2022-2-C2-P2.pdf
% (find-LATEX "2019.mk")
% (find-sh0 "cd ~/LUA/; cp -v Pict2e1.lua Pict2e1-1.lua Piecewise1.lua ~/LATEX/")
% (find-sh0 "cd ~/LUA/; cp -v Pict2e1.lua Pict2e1-1.lua Pict3D1.lua ~/LATEX/")
% (find-sh0 "cd ~/LUA/; cp -v C2Subst1.lua C2Formulas1.lua ~/LATEX/")
% (find-CN-aula-links "2022-2-C2-P2" "2" "c2m222p2" "c2p2")

% «.defs»		(to "defs")
% «.defs-T-and-B»	(to "defs-T-and-B")
% «.title»		(to "title")
% «.links»		(to "links")
%   «.links-edovs»	(to "links-edovs")
%   «.links-edolcc»	(to "links-edolcc")
% «.questao-1»		(to "questao-1")
%   «.edovs»		(to "edovs")
% «.questao-2»		(to "questao-2")
%   «.edolccs»		(to "edolccs")
% «.questao-3»		(to "questao-3")
% «.questao-1-gab»	(to "questao-1-gab")
% «.questao-2-gab»	(to "questao-2-gab")
% «.questao-3-gab»	(to "questao-3-gab")
%
% «.djvuize»		(to "djvuize")



% <videos>
% Video (not yet):
% (find-ssr-links     "c2m222p2" "2022-2-C2-P2")
% (code-eevvideo      "c2m222p2" "2022-2-C2-P2")
% (code-eevlinksvideo "c2m222p2" "2022-2-C2-P2")
% (find-c2m222p2video "0:00")

\documentclass[oneside,12pt]{article}
\usepackage[colorlinks,citecolor=DarkRed,urlcolor=DarkRed]{hyperref} % (find-es "tex" "hyperref")
\usepackage{amsmath}
\usepackage{amsfonts}
\usepackage{amssymb}
\usepackage{pict2e}
\usepackage[x11names,svgnames]{xcolor} % (find-es "tex" "xcolor")
\usepackage{colorweb}                  % (find-es "tex" "colorweb")
\usepackage{emoji}                     % (find-es "tex" "emoji")
%\usepackage{tikz}
%
% (find-dn6 "preamble6.lua" "preamble0")
%\usepackage{proof}   % For derivation trees ("%:" lines)
%\input diagxy        % For 2D diagrams ("%D" lines)
%\xyoption{curve}     % For the ".curve=" feature in 2D diagrams
%
\usepackage{edrx21}               % (find-LATEX "edrx21.sty")
\input edrxaccents.tex            % (find-LATEX "edrxaccents.tex")
\input edrx21chars.tex            % (find-LATEX "edrx21chars.tex")
\input edrxheadfoot.tex           % (find-LATEX "edrxheadfoot.tex")
\input edrxgac2.tex               % (find-LATEX "edrxgac2.tex")
%\usepackage{emaxima}              % (find-LATEX "emaxima.sty")
%
%\usepackage[backend=biber,
%   style=alphabetic]{biblatex}            % (find-es "tex" "biber")
%\addbibresource{catsem-slides.bib}        % (find-LATEX "catsem-slides.bib")
%
% (find-es "tex" "geometry")
\usepackage[a6paper, landscape,
            top=1.5cm, bottom=.25cm, left=1cm, right=1cm, includefoot
           ]{geometry}
%
\begin{document}

\catcode`\^^J=10
\directlua{dofile "dednat6load.lua"}  % (find-LATEX "dednat6load.lua")
%L dofile "Piecewise1.lua"           -- (find-LATEX "Piecewise1.lua")
%L -- dofile "QVis1.lua"             -- (find-LATEX "QVis1.lua")
%L -- dofile "Pict3D1.lua"           -- (find-LATEX "Pict3D1.lua")
%L dofile "C2Formulas1.lua"          -- (find-LATEX "C2Formulas1.lua")
%L dofile "Lazy5.lua"                -- (find-LATEX "Lazy5.lua")
%L dofile "2022-1-C2-P2.lua"         -- (find-LATEX "2022-1-C2-P2.lua")
%L Pict2e.__index.suffix = "%"
\pu
\def\pictgridstyle{\color{GrayPale}\linethickness{0.3pt}}
\def\pictaxesstyle{\linethickness{0.5pt}}
\def\pictnaxesstyle{\color{GrayPale}\linethickness{0.5pt}}
\celllower=2.5pt

% «defs»  (to ".defs")
% (find-LATEX "edrx21defs.tex" "colors")
% (find-LATEX "edrx21.sty")

\def\u#1{\par{\footnotesize \url{#1}}}

\def\drafturl{http://angg.twu.net/LATEX/2022-2-C2.pdf}
\def\drafturl{http://angg.twu.net/2022.2-C2.html}
\def\draftfooter{\tiny \href{\drafturl}{\jobname{}} \ColorBrown{\shorttoday{} \hours}}

\sa{[M]}{\CFname{M}{}}
\sa{[F]}{\CFname{F}{}}
\sa{[S]}{\CFname{S}{}}

% (find-LATEXgrep "grep --color=auto -nH --null -e mname 202{1,2}*.tex")
\def\sumiN#1{\sum_{i=1}^N #1 (b_i-a_i)}
\def\mname#1{\text{[#1]}}

\def\Smile{\emoji{slightly-smiling-face}}


% «defs-T-and-B»  (to ".defs-T-and-B")
\long\def\ColorOrange#1{{\color{orange!90!black}#1}}
\def\T(Total: #1 pts){{\bf(Total: #1)}}
\def\T(Total: #1 pts){{\bf(Total: #1 pts)}}
\def\T(Total: #1 pts){\ColorRed{\bf(Total: #1 pts)}}
\def\B       (#1 pts){\ColorOrange{\bf(#1 pts)}}



%  _____ _ _   _                               
% |_   _(_) |_| | ___   _ __   __ _  __ _  ___ 
%   | | | | __| |/ _ \ | '_ \ / _` |/ _` |/ _ \
%   | | | | |_| |  __/ | |_) | (_| | (_| |  __/
%   |_| |_|\__|_|\___| | .__/ \__,_|\__, |\___|
%                      |_|          |___/      
%
% «title»  (to ".title")
% (c2m222p2p 1 "title")
% (c2m222p2a   "title")

\thispagestyle{empty}

\begin{center}

\vspace*{1.2cm}

{\bf \Large Cálculo 2 - 2022.2}

\bsk

P2 (Segunda prova)

\bsk

Eduardo Ochs - RCN/PURO/UFF

\url{http://angg.twu.net/2022.2-C2.html}

\end{center}

\newpage

% «links»  (to ".links")
% (c2m222p2p 2 "links")
% (c2m222p2a   "links")
% (c2m222dp2p 2 "links")
% (c2m222dp2a   "links")

% «links-edovs»  (to ".links-edovs")
% (c2m222p2p 2 "links-edovs")
% (c2m222p2a   "links-edovs")
% (c2m222edovsp 2 "links")
% (c2m222edovsa   "links")
% (c2m221vsbp 5 "questao-4")
% (c2m221vsba   "questao-4")
% (find-es "maxima" "separable-2")

% «links-edolcc»  (to ".links-edolcc")
% (c2m222edolsp 2 "links")
% (c2m222edolsa   "links")

% (c2m222dp2p 3 "somas-de-riemann")
% (c2m222dp2a   "somas-de-riemann")


\newpage

%  _     _____ ____   _____     ______  
% / |   | ____|  _ \ / _ \ \   / / ___| 
% | |   |  _| | | | | | | \ \ / /\___ \ 
% | |_  | |___| |_| | |_| |\ V /  ___) |
% |_(_) |_____|____/ \___/  \_/  |____/ 
%                                       
% «questao-1»  (to ".questao-1")
% (c2m222p2p 2 "questao-1")
% (c2m222p2a   "questao-1")
% «edovs»  (to ".edovs")
% (c2m222p2p 99 "edovs")
% (c2m222p2a    "edovs")
% (find-es "maxima" "separable-2")
% (find-es "maxima" "2022-2-C2-P2-edovs")

{\bf Questão 1}

%L namedang("EDOVSintro", "", [[
%L    \begin{array}{rcl}
%L      \ga{[M]} &=& <EDOVSG> \\ \\[-5pt]
%L      \ga{[F]} &=& <EDOVSP> \\
%L    \end{array}
%L ]])
%L EDOVSintro:sa("FOO"):output()
\pu

\scalebox{0.55}{\def\colwidth{10cm}\firstcol{

\vspace*{-0.4cm}

\T(Total: 6.0 pts)

Lembre que nós vimos que o ``método'' para resolver EDOs com variáveis
separáveis --- ``EDOVSs'' --- pode ser escrito como a demonstração
$\ga{[M]}$ abaixo, e a ``fórmula'' para resolver EDOVSs pode ser
escrita como $\ga{[F]}$:

\bsk

$\ga{FOO}$

}\anothercol{

  Quando a gente quer criar exercícios de EDOVSs que sejam fácil de
  resolver a gente começa escolhendo $G(x)$ e $H(y)$, não $g(x)$ e
  $h(y)$.

  Digamos que $G(x)=x^4+5$ e $H(y)=y^2+3$.

  \msk

  a) \B (0.5 pts) Diga qual é a EDO da forma
  $\frac{dy}{dx} = \frac{g(x)}{h(y)}$ associada a esta escolha de
  $G(c)$ e $H(y)$. Chame-a de $(*)$. Não esqueça do ``Seja''!

  \ssk

  b) \B (0.5 pts) Escolha uma função $H^{-1}$ adequada. Defina ela com
  um ``Seja'' e verifique que ela obedece o que esperamos dela.

  \ssk

  c) \B (1.0 pts) Encontre a solução geral da EDO $(*)$. Chame-a de
  $f(x)$ e defina ela com um ``Seja''.

  \ssk

  d) \B (1.5 pts) Verifique que essa função $f(x)$ obedece $(*)$.

  \ssk

  e) \B (1.0 pts) Encontre uma solução $f_1(x)$ que passe pelo ponto
  $(x_1,y_1)=(1,2)$. Defina-a com um ``Seja''.

  \ssk


  f) \B (1.5 pts) Teste a sua solução $f_1(x)$.


% (find-es "maxima" "2022-2-C2-P2")



}}

\newpage

%  ____      _____ ____   ___  _     ____ ____     
% |___ \    | ____|  _ \ / _ \| |   / ___/ ___|___ 
%   __) |   |  _| | | | | | | | |  | |  | |   / __|
%  / __/ _  | |___| |_| | |_| | |__| |__| |___\__ \
% |_____(_) |_____|____/ \___/|_____\____\____|___/
%                                                  
% «questao-2»  (to ".questao-2")
% (c2m222p2p 3 "questao-2")
% (c2m222p2a   "questao-2")
% «edolccs»  (to ".edolccs")
% (c2m222p2p 3 "edolccs")
% (c2m222p2a   "edolccs")
% (find-es "maxima" "2022-2-C2-P2-edolccs")

{\bf Questão 2}

\scalebox{0.6}{\def\colwidth{9cm}\firstcol{

\vspace*{-0.4cm}

\T(Total: 3.0 pts)

No curso nós vimos um modo de resolver EDOs lineares com coeficientes
constantes --- ``EDOLCCs'' --- no qual a gente traduzia a EDO ``pra
Álgebra Linear'', fatorava uma ``matriz'', e aí encontrava as soluções
básicas dessa EDO e tratava elas como ``vetores''... por exemplo,
%
$$\begin{array}{rcl}
  y'' + 5y' + 6y &=& 0 \\
  (D^2 + 5D + 6)f &=& 0 \\
  (D+2)(D+3)f &=& 0 \\
  M &=& (D+2)(D+3) \\
  M e^{-2x} &=& 0 \\
  M e^{-3x} &=& 0 \\
  M e^{-2x} &=& 0 \\
  M(42e^{-2x} + 99e^{-3x}) &=& 0 \\
  \end{array}
$$


Seja $(*)$ esta EDO:
%
$$y'' + y' - 20y \;=\; 0 \qquad (*)
$$


}\anothercol{

  a) \B (0.2 pts) Traduza a EDO $(*)$ para ``Álgebra Linear'' e
  fatore-a. Chame essa versão fatorada de $(**)$, e defina-a com um
  ``Seja''.

  \msk

  b) \B (0.3 pts) Encontre as duas soluções básicas para a EDO $(*)$.
  Chame elas de $f_1$ e $f_2$. Não esqueça o ``Sejam''!

  \msk

  c) \B (0.5 pts) Encontre a solução geral para a EDO $(*)$ e chame-a
  de $f$. Não esqueça o ``Seja''!

  \msk

  d) \B (2.0 pts) Encontre uma solução $g$ para a EDO $(*)$ que
  obedeça $g(0)=7$ e $g'(0)=1$. Defina esta $g$ com um ``seja'' e
  verifique que ela realmente obedece $g(0)=7$ e $g'(0)=1$.


% (find-es "maxima" "2022-2-C2-P2")



}}

\newpage


%   ___                  _                _____ 
%  / _ \ _   _  ___  ___| |_ __ _  ___   |___ / 
% | | | | | | |/ _ \/ __| __/ _` |/ _ \    |_ \ 
% | |_| | |_| |  __/\__ \ || (_| | (_) |  ___) |
%  \__\_\\__,_|\___||___/\__\__,_|\___/  |____/ 
%                                               
% «questao-3»  (to ".questao-3")
% (c2m222p2p 4 "questao-3")
% (c2m222p2a   "questao-3")

%L Pict2e.bounds = PictBounds.new(v(0,0), v(7,6))
%L spec = "(0,1)--(1,1)--(2,4)--(3,5)--(4,4)o (4,3)c (4,1)o--(6,3)--(7,3)"
%L pws = PwSpec.from(spec)
%L pws:topict():prethickness("1pt"):pgat("pgatc"):sa("F(x)"):output()
\pu

\unitlength=10pt

{\bf Questão 3}


\scalebox{0.55}{\def\colwidth{10.5cm}\firstcol{

\vspace*{-0.25cm}

\T(Total: 1.5 pts)

Lembre que nós vimos estes tipos de Somas de Riemann,
%
$$\scalebox{0.95}{$
  \begin{array}{ccl}
  \mname{L}    &=& \sumiN {f(a_i)}                    \\[2pt]
  \mname{R}    &=& \sumiN {f(b_i)}                    \\[2pt]
  \mname{Trap} &=& \sumiN {\frac{f(a_i) + f(b_i)}{2}} \\[2pt]
  \mname{M}    &=& \sumiN {f(\frac{a_i+b_i}{2})}      \\[2pt]
  \mname{min}  &=& \sumiN {\min(f(a_i), f(b_i))}      \\[2pt]
  \mname{max}  &=& \sumiN {\max(f(a_i), f(b_i))}      \\[2pt]
  \mname{inf}  &=& \sumiN {\inf(f([a_i,b_i]))}        \\[2pt]
  \mname{sup}  &=& \sumiN {\sup(f([a_i,b_i]))}        \\
  \end{array}
  $}
$$

e vimos que o $\mname{Trap}$ pode ser interpretado tanto como uma soma
de trapézios como como uma soma de retângulos.

\msk

Seja $f(x)$ a função dos gráficos à direita.

Represente graficamente:

\msk

a) $\mname{inf}_{\{1,2,3,4\}}$

b) $\mname{sup}_{\{1,2,3,4\}}$

c) $\mname{M}_{\{1,3,5\}}$

d) $\mname{Trap}_{\{1,3,5\}}$ usando retângulos

e) $\mname{Trap}_{\{1,3,5\}}$ usando trapézios

\msk

Indique claramente qual desenho é a resposta final de cada item e
quais desenhos são rascunhos.

}\anothercol{

\vspace*{0cm}

\def\Fx{\scalebox{1.2}{$\ga{F(x)}$}}

$\begin{matrix}
 \Fx & \Fx & \Fx \\ \\[-5pt]
 \Fx & \Fx & \Fx \\ \\[-5pt]
 \Fx & \Fx & \Fx \\ \\[-5pt]
 \Fx & \Fx & \Fx \\
 \end{matrix}
$

}}


\newpage

% «questao-1-gab»  (to ".questao-1-gab")
% (c2m222p2p 5 "questao-1-gab")
% (c2m222p2a   "questao-1-gab")

\scalebox{0.6}{\def\colwidth{9cm}\firstcol{

{\bf Questão 1: gabarito}

A substituição é:
%
$$\ga{[S]} \;=\;
  \bmat{
    G(x) := x^4 + 5 \\
    H(y) := y^2 + 3 \\
    g(x) := 4x^3 \\
    h(y) := 2y \\
    H^{-1}(x) := \sqrt{x-3} \\
  }
$$

a) Seja:
%
$$\frac{dy}{dx} = \frac{4x^3}{2y} \qquad (*)$$

b)
%
 $\begin{array}[t]{lrcl}
  \text{Seja:}  & H^{-1}(x) &=& \sqrt{x-3}. \\
  \text{Temos:} & H^{-1}(H(y)) &=& \sqrt{H(y)-3} \\
                &              &=& \sqrt{(y^2+3)-3} \\
                &              &=& y. \\
  \end{array}
 $

\msk

c) $\begin{array}[t]{lrcl}
       & y &=& H^{-1}(G(x)+C_3) \\
          &&=& \sqrt{(G(x)+C_3)-3} \\
          &&=& \sqrt{((x^4+5)+C_3)-3} \\
          &&=& \sqrt{x^4+2+C_3} \\
    \text{Seja:} &
      f(x) &=& \sqrt{x^4+2+C_3}. \\
    \end{array}
   $


}\anothercol{

\vspace*{0cm}

d) $\begin{array}[t]{l}
    \text{Será que $f(x)$ obedece $(*)$?} \\
    \text{Temos }
    f'(x) = \frac{2x^3}{\sqrt{x^4 + 2 + C_3}},
    \text{ e com isso:}
    \\
    \\[-5pt]
    \left(
      f'(x) = \frac{4x^3}{2f(x)}
    \right)
    \bmat{
      f(x) = \sqrt{x^4+2+C_3} \\
      f'(x) = \frac{2x^3}{\sqrt{x^4 + 2 + C_3}} \\
    }
    \\
    = \;\;
    \left(
      \frac{2x^3}{\sqrt{x^4 + 2 + C_3}}
      = \frac{4x^3}{2\sqrt{x^4+2+C_3}}
    \right)
    \qquad \smile \\
    \end{array}
   $

\bsk

e) $\begin{array}[t]{lrcl}
    \text{Se}    & f(x_1) &=& y_1, \\
    \text{i.e.,} & f(1)   &=& 2,   \\
    \text{então} & f(1)   &=& \sqrt{1^4+2+C_3} \\
                         &&=& \sqrt{3+C_3} \\
                         &&=& 2 \\
                 & 2^2    &=& \sqrt{3+C_3}^2 \\
                 & 4      &=&       3+C_3    \\
                 & C_3    &=& 1 \\
                 & f(x)   &=& \sqrt{x^4+2+C_3} \\
                 &        &=& \sqrt{x^4+3} \\
    \text{Seja:} & f_1(x) &=& \sqrt{x^4+3}. \\
    \end{array}
   $

\bsk

f) $\begin{array}[t]{lrcl}
    \text{Será que} & f_1(x_1) &=& y_1, \\
    \text{i.e.,}    & f_1(1)   &=& 2?   \\
                & \sqrt{1^4+3} &=& \sqrt{4} \\
                              &&=& 2 \qquad \smile \\
    \end{array}
   $


}}


\newpage

% «questao-2-gab»  (to ".questao-2-gab")
% (c2m222p2p 6 "questao-2-gab")
% (c2m222p2a   "questao-2-gab")

\scalebox{0.6}{\def\colwidth{9cm}\firstcol{

{\bf Questão 2: gabarito}

\msk

a) Temos: $D^2 + D - 20 = (D+5)(D-4)$.

\phantom{a)}
Seja $(**)$ esta EDO:
%
$$(D+5)(D-4)f \; = \; 0. \qquad (**)
$$

\msk

b) Sejam $f_1(x) = e^{4x}$,
         $f_2(x) = e^{-5x}$, 

\msk

c) Seja
%
$$\begin{array}{rcl}
  f(x) &=& af_1(x) + bf_2(x) \\
       &=& ae^{4x} + be^{-5x}. \\
  \end{array}
$$

d)
%
$\begin{array}[t]{lrcl}
    \text{Digamos que} &
        g(x) &=& af_1(x) + bf_2(x) \\
            &&=& ae^{4x} + be^{-5x}, \\
      & g(0) &=& 7, \\
      & g'(0) &=& 1. \\
    \text{Então:}
      & g(0)  &=& ae^0 + be^0, \\
             &&=& a + b, \\
      & g'(0) &=& a·4e^0 + b·(-5)e^0, \\
             &&=& 4a -5b, \\
      &     a &=& 4, \\
      &     b &=& 3, \\
      &  g(x) &=& 4e^{4x} +3e^{-5x}, \\
      &  g(0) &=& 4 + 3 \;\;=\;\; 7, \qquad \smile \\
      &  g'(0) &=& 16 - 15 \;\;=\;\; 1, \quad\, \smile. \\
  \end{array}
$

}\anothercol{

% «questao-3-gab»  (to ".questao-3-gab")
% (c2m222p2p 6 "questao-3-gab")
% (c2m222p2a   "questao-3-gab")

{\bf Questão 3: gabarito (sem desenhos)}

\bsk

\def\Item#1{\text{#1) }}



$\begin{array}{lcl}
 \Item{a} \mname{inf}_{\{1,2,3,4\}}
   &=& 1(2-1) + 4(3-2) + 3(4-3) \\
 \Item{b} \mname{sup}_{\{1,2,3,4\}}
   &=& 4(2-1) + 5(3-2) + 5(4-3) \\
 \Item{c} \mname{M}  _{\{1,3,5\}}
   &=& 4(3-1) + 3(5-3) \\
 \Item{d} \mname{Trap}_{\{1,3,5\}}
   &=& 3(3-1) + 3.5(5-3) \\
 \Item{e} \mname{Trap}_{\{1,3,5\}}
   &=& \frac{1+5}{2}(3-1) + \frac{5+2}{2}(5-3) \\
 \end{array}
$

\bsk
\bsk

$$\unitlength=20pt
  \ga{F(x)}
$$

}}



%L Pict2e.bounds = PictBounds.new(v(0,0), v(7,6))
%L spec = "(0,1)--(1,1)--(2,4)--(3,5)--(4,4)o (4,3)c (4,1)o--(6,3)--(7,3)"
%L pws = PwSpec.from(spec)
%L pws:topict():prethickness("1pt"):pgat("pgatc"):sa("F(x)"):output()
\pu




%\printbibliography

\GenericWarning{Success:}{Success!!!}  % Used by `M-x cv'

\end{document}

%  ____  _             _         
% |  _ \(_)_   ___   _(_)_______ 
% | | | | \ \ / / | | | |_  / _ \
% | |_| | |\ V /| |_| | |/ /  __/
% |____// | \_/  \__,_|_/___\___|
%     |__/                       
%
% «djvuize»  (to ".djvuize")
% (find-LATEXgrep "grep --color -nH --null -e djvuize 2020-1*.tex")

 (eepitch-shell)
 (eepitch-kill)
 (eepitch-shell)
# (find-fline "~/2022.2-C2/")
# (find-fline "~/LATEX/2022-2-C2/")
# (find-fline "~/bin/djvuize")

cd /tmp/
for i in *.jpg; do echo f $(basename $i .jpg); done

f () { rm -v $1.pdf;  textcleaner -f 50 -o  5 $1.jpg $1.png; djvuize $1.pdf; xpdf $1.pdf }
f () { rm -v $1.pdf;  textcleaner -f 50 -o 10 $1.jpg $1.png; djvuize $1.pdf; xpdf $1.pdf }
f () { rm -v $1.pdf;  textcleaner -f 50 -o 20 $1.jpg $1.png; djvuize $1.pdf; xpdf $1.pdf }

f () { rm -fv $1.png $1.pdf; djvuize $1.pdf }
f () { rm -fv $1.png $1.pdf; djvuize WHITEBOARDOPTS="-m 1.0 -f 15" $1.pdf; xpdf $1.pdf }
f () { rm -fv $1.png $1.pdf; djvuize WHITEBOARDOPTS="-m 1.0 -f 30" $1.pdf; xpdf $1.pdf }
f () { rm -fv $1.png $1.pdf; djvuize WHITEBOARDOPTS="-m 1.0 -f 45" $1.pdf; xpdf $1.pdf }
f () { rm -fv $1.png $1.pdf; djvuize WHITEBOARDOPTS="-m 0.5" $1.pdf; xpdf $1.pdf }
f () { rm -fv $1.png $1.pdf; djvuize WHITEBOARDOPTS="-m 0.25" $1.pdf; xpdf $1.pdf }
f () { cp -fv $1.png $1.pdf       ~/2022.2-C2/
       cp -fv        $1.pdf ~/LATEX/2022-2-C2/
       cat <<%%%
% (find-latexscan-links "C2" "$1")
%%%
}

f 20201213_area_em_funcao_de_theta
f 20201213_area_em_funcao_de_x
f 20201213_area_fatias_pizza



%  __  __       _        
% |  \/  | __ _| | _____ 
% | |\/| |/ _` | |/ / _ \
% | |  | | (_| |   <  __/
% |_|  |_|\__,_|_|\_\___|
%                        
% <make>

 (eepitch-shell)
 (eepitch-kill)
 (eepitch-shell)
# (find-LATEXfile "2019planar-has-1.mk")
make -f 2019.mk STEM=2022-2-C2-P2 veryclean
make -f 2019.mk STEM=2022-2-C2-P2 pdf

% Local Variables:
% coding: utf-8-unix
% ee-tla: "c2p2"
% ee-tla: "c2m222p2"
% End:
