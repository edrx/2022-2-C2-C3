% (find-LATEX "2022-2-C2-fracoes-parciais.tex")
% (defun c () (interactive) (find-LATEXsh "lualatex -record 2022-2-C2-fracoes-parciais.tex" :end))
% (defun C () (interactive) (find-LATEXsh "lualatex 2022-2-C2-fracoes-parciais.tex" "Success!!!"))
% (defun D () (interactive) (find-pdf-page      "~/LATEX/2022-2-C2-fracoes-parciais.pdf"))
% (defun d () (interactive) (find-pdftools-page "~/LATEX/2022-2-C2-fracoes-parciais.pdf"))
% (defun e () (interactive) (find-LATEX "2022-2-C2-fracoes-parciais.tex"))
% (defun o () (interactive) (find-LATEX "2021-2-C2-fracoes-parciais.tex"))
% (defun u () (interactive) (find-latex-upload-links "2022-2-C2-fracoes-parciais"))
% (defun v () (interactive) (find-2a '(e) '(d)))
% (defun d0 () (interactive) (find-ebuffer "2022-2-C2-fracoes-parciais.pdf"))
% (defun cv () (interactive) (C) (ee-kill-this-buffer) (v) (g))
%          (code-eec-LATEX "2022-2-C2-fracoes-parciais")
% (find-pdf-page   "~/LATEX/2022-2-C2-fracoes-parciais.pdf")
% (find-sh0 "cp -v  ~/LATEX/2022-2-C2-fracoes-parciais.pdf /tmp/")
% (find-sh0 "cp -v  ~/LATEX/2022-2-C2-fracoes-parciais.pdf /tmp/pen/")
%     (find-xournalpp "/tmp/2022-2-C2-fracoes-parciais.pdf")
%   file:///home/edrx/LATEX/2022-2-C2-fracoes-parciais.pdf
%               file:///tmp/2022-2-C2-fracoes-parciais.pdf
%           file:///tmp/pen/2022-2-C2-fracoes-parciais.pdf
% http://angg.twu.net/LATEX/2022-2-C2-fracoes-parciais.pdf
% (find-LATEX "2019.mk")
% (find-sh0 "cd ~/LUA/; cp -v Pict2e1.lua Pict2e1-1.lua Piecewise1.lua ~/LATEX/")
% (find-sh0 "cd ~/LUA/; cp -v Pict2e1.lua Pict2e1-1.lua Pict3D1.lua ~/LATEX/")
% (find-sh0 "cd ~/LUA/; cp -v C2Subst1.lua C2Formulas1.lua ~/LATEX/")
% (find-CN-aula-links "2022-2-C2-fracoes-parciais" "2" "c2m222fp" "c2fp")

% «.defs»		(to "defs")
% «.title»		(to "title")
% «.links»		(to "links")
%  «.lua»		(to "lua")
% «.contas-sem-vai-um»	(to "contas-sem-vai-um")
% «.div-polis»		(to "div-polis")
% «.exercicio-1»	(to "exercicio-1")
% «.exercicio-2»	(to "exercicio-2")
% «.derivadas-formais»	(to "derivadas-formais")
% «.together»		(to "together")
% «.exercicio-3»	(to "exercicio-3")
% «.exercicio-4»	(to "exercicio-4")
% «.exercicio-4a»	(to "exercicio-4a")
% «.exercicio-4a-2»	(to "exercicio-4a-2")
% «.exercicio-4b»	(to "exercicio-4b")
% «.exercicio-4c»	(to "exercicio-4c")
% «.exercicio-5»	(to "exercicio-5")
% «.exercicio-6»	(to "exercicio-6")
%
% «.djvuize»	(to "djvuize")



% <videos>
% Video (not yet):
% (find-ssr-links     "c2m222fp" "2022-2-C2-fracoes-parciais")
% (code-eevvideo      "c2m222fp" "2022-2-C2-fracoes-parciais")
% (code-eevlinksvideo "c2m222fp" "2022-2-C2-fracoes-parciais")
% (find-c2m222fpvideo "0:00")

\documentclass[oneside,12pt]{article}
\usepackage[colorlinks,citecolor=DarkRed,urlcolor=DarkRed]{hyperref} % (find-es "tex" "hyperref")
\usepackage{amsmath}
\usepackage{amsfonts}
\usepackage{amssymb}
\usepackage{pict2e}
\usepackage[x11names,svgnames]{xcolor} % (find-es "tex" "xcolor")
\usepackage{colorweb}                  % (find-es "tex" "colorweb")
%\usepackage{tikz}
%
% (find-dn6 "preamble6.lua" "preamble0")
%\usepackage{proof}   % For derivation trees ("%:" lines)
%\input diagxy        % For 2D diagrams ("%D" lines)
%\xyoption{curve}     % For the ".curve=" feature in 2D diagrams
%
\usepackage{edrx21}               % (find-LATEX "edrx21.sty")
\input edrxaccents.tex            % (find-LATEX "edrxaccents.tex")
\input edrx21chars.tex            % (find-LATEX "edrx21chars.tex")
\input edrxheadfoot.tex           % (find-LATEX "edrxheadfoot.tex")
\input edrxgac2.tex               % (find-LATEX "edrxgac2.tex")
%\usepackage{emaxima}              % (find-LATEX "emaxima.sty")
%
%\usepackage[backend=biber,
%   style=alphabetic]{biblatex}            % (find-es "tex" "biber")
%\addbibresource{catsem-slides.bib}        % (find-LATEX "catsem-slides.bib")
%
% (find-es "tex" "geometry")
\usepackage[a6paper, landscape,
            top=1.5cm, bottom=.25cm, left=1cm, right=1cm, includefoot
           ]{geometry}
%
\begin{document}

\catcode`\^^J=10
\directlua{dofile "dednat6load.lua"}  % (find-LATEX "dednat6load.lua")
%L dofile "Piecewise1.lua"           -- (find-LATEX "Piecewise1.lua")
%L dofile "QVis1.lua"                -- (find-LATEX "QVis1.lua")
%L dofile "Pict3D1.lua"              -- (find-LATEX "Pict3D1.lua")
%L dofile "C2Formulas1.lua"          -- (find-LATEX "C2Formulas1.lua")
%L Pict2e.__index.suffix = "%"
\pu
\def\pictgridstyle{\color{GrayPale}\linethickness{0.3pt}}
\def\pictaxesstyle{\linethickness{0.5pt}}
\def\pictnaxesstyle{\color{GrayPale}\linethickness{0.5pt}}
\celllower=2.5pt

% «defs»  (to ".defs")
% (find-LATEX "edrx21defs.tex" "colors")
% (find-LATEX "edrx21.sty")

\def\u#1{\par{\footnotesize \url{#1}}}

\def\drafturl{http://angg.twu.net/LATEX/2022-2-C2.pdf}
\def\drafturl{http://angg.twu.net/2022.2-C2.html}
\def\draftfooter{\tiny \href{\drafturl}{\jobname{}} \ColorBrown{\shorttoday{} \hours}}

\def\together   {\mathsf{together}}
\def\togetherp#1{\mathsf{together}\left(#1\right)}
\def\apart      {\mathsf{apart}}



%  _____ _ _   _                               
% |_   _(_) |_| | ___   _ __   __ _  __ _  ___ 
%   | | | | __| |/ _ \ | '_ \ / _` |/ _` |/ _ \
%   | | | | |_| |  __/ | |_) | (_| | (_| |  __/
%   |_| |_|\__|_|\___| | .__/ \__,_|\__, |\___|
%                      |_|          |___/      
%
% «title»  (to ".title")
% (c2m222fpp 1 "title")
% (c2m222fpa   "title")

\thispagestyle{empty}

\begin{center}

\vspace*{1.2cm}

{\bf \Large Cálculo 2 - 2022.2}

\bsk

Aula 9: Frações Parciais

\bsk

Eduardo Ochs - RCN/PURO/UFF

\url{http://angg.twu.net/2022.2-C2.html}

\end{center}

\newpage

%  _     _       _        
% | |   (_)_ __ | | _____ 
% | |   | | '_ \| |/ / __|
% | |___| | | | |   <\__ \
% |_____|_|_| |_|_|\_\___/
%                         
% «links»  (to ".links")
% (c2m222fpp 2 "links")
% (c2m222fpa   "links")
% (c2m212fpa "title")
% (c2m212fpa "title" "Aula nn: frações parciais")

{\bf Links}

% (c2m202itp 2 "div-polis")
% (c2m202it    "div-polis")
% (c2m221dfip 5 "demonstracao-complicada")
% (c2m221dfia   "demonstracao-complicada")
% (find-books "__analysis/__analysis.el" "leithold")
% (find-books "__analysis/__analysis.el" "leithold" "frações parciais")
% (find-books "__analysis/__analysis.el" "miranda")
% (find-books "__analysis/__analysis.el" "miranda" "Frações Parciais")

Tanto o Leithold quanto o Daniel Miranda

têm seções sobre frações parciais.

A seção do Leithold é a 9.5.

A do Miranda é a 8.1:

\ssk

{\scriptsize

% (find-dmirandacalcpage 240 "8.1 Frações Parciais")
%    http://hostel.ufabc.edu.br/~daniel.miranda/calculo/calculo.pdf\#page=240
\url{http://hostel.ufabc.edu.br/~daniel.miranda/calculo/calculo.pdf#page=240}

}

\msk

A idéia de frações parciais que vai ser

mais importante pra outras matérias é que

operações como polinômios são como

operações sobre números ``sem vai um''.

Tem figuras (manuscritas) sobre isso aqui:

% (find-angg ".emacs" "c2q191")
% (find-angg ".emacs" "c2q191" "Frações parciais")
% (find-angg ".emacs" "c2q192")
% (find-angg ".emacs" "c2q192" "funções racionais")
% (c2q191 26 "20190517" "Frações parciais; truques com polinômios")
% (c2q192 43 "20190913 gde aula 9: ...parte 2: truques com polinômios, Heaviside")

\ssk

{\scriptsize

% http://angg.twu.net/2019.1-C2/2019.1-C2.pdf#page=26
\url{http://angg.twu.net/2019.1-C2/2019.1-C2.pdf\#page=26}

% http://angg.twu.net/2019.2-C2/2019.2-C2.pdf#page=43
\url{http://angg.twu.net/2019.2-C2/2019.2-C2.pdf\#page=43}

}

% «lua»  (to ".lua")
% (c2m202itp 2 "div-polis")
% (c2m202it    "div-polis")


\newpage

% «contas-sem-vai-um»  (to ".contas-sem-vai-um")
% (c2m222fpp 3 "contas-sem-vai-um")
% (c2m222fpa   "contas-sem-vai-um")
% (c2m202fpp 6 "contas-sem-vai-um")
% (c2m202fp    "contas-sem-vai-um")

{\bf Slogan: contas sem ``vai um'' podem ser traduzidas

pra contas com polinômios.}

\ssk

O que mais nos interessa pra Frações Parciais

é \ColorRed{divisão com resto}. Exemplo:

% (find-fline "~/LATEX/2020-1-C2/20201118_C2_div_com_resto_1.pdf")
\includegraphics[width=11cm]{2020-1-C2/20201118_C2_div_com_resto_1.pdf}

\newpage

% «div-polis»  (to ".div-polis")
% (c2m222fpp 4 "div-polis")
% (c2m222fpa   "div-polis")
% (c2m202fpp 7 "div-polis")
% (c2m202fp    "div-polis")

...e tradução do exemplo para polinômios:

% (find-fline "~/LATEX/2020-1-C2/20201118_C2_div_com_resto_2.pdf")
\includegraphics[height=4cm]{2020-1-C2/20201118_C2_div_com_resto_2.pdf}

% (find-fline "~/LATEX/2020-1-C2/20201118_C2_div_com_resto_3.pdf")
\includegraphics[height=3cm]{2020-1-C2/20201118_C2_div_com_resto_3.pdf}

\newpage

% «exercicio-1»  (to ".exercicio-1")
% (c2m222fpp 5 "exercicio-1")
% (c2m222fpa   "exercicio-1")

{\bf Exercício 1}

\sa{[RC]}{\CFname{RC}{}}
\sa {RC} {\ddx f(g(x)) = f'(g(x))g'(x)}
\sa{(RC)}{\left(\ga{RC}\right)}

Algumas consequências da regra da cadeia...
%
$$\ga{[RC]} \;=\; \ga{(RC)}$$

Obtenha os seguintes casos particulares da \ga{[RC]}:

\msk

a) $g(x) = 2x$

b) $g(x) = 2x+3$

c) $g(x) = x+3$

d) $g(x) = x+3$, $f(x)=\ln x$

e) $g(x) = -x$

f) $g(x) = -x$, $f(x) = \ln x$

g) $g(x) = -x+200$, $f(x) = \ln x$

\newpage

% «exercicio-2»  (to ".exercicio-2")
% (c2m222fpp 6 "exercicio-2")
% (c2m222fpa   "exercicio-2")
% (c2m201fracparcp 1 "title")
% (c2m201fracparc    "title")
% (c2m202fpp 2 "exercicio-1")
% (c2m202fp    "exercicio-1")


{\bf Exercício 2.}

\msk

a) $\D \intx{\frac{1}{3x}} = \ColorRed{?}$

\ssk

b) $\D \intx{\frac{1}{3x+4}} = \ColorRed{?}$

\ssk

c) $\D \intx{\frac{2}{3x+4}} = \ColorRed{?}$

\ssk

d) $\D \intx{\frac{a}{bx+c}} = \ColorRed{?}$


\newpage

% «derivadas-formais»  (to ".derivadas-formais")
% (c2m222fpp 7 "derivadas-formais")
% (c2m222fpa   "derivadas-formais")

{\bf Derivadas formais (de novo)}

Todas estas igualdades são verdadeiras,

mas se tentarmos formalizar elas com

todos os detalhes vamos ver que várias delas

falam de funções com domínios diferentes...
%
$$\begin{array}[t]{rcl}
  \ddx \ln x &=& \frac1x \\
  \ddx \ln (-x) &=& \frac1x \\
  \ddx \ln |x| &=& \frac1x \\
  \end{array}
  \qquad
  \begin{array}[t]{rcl}
  \intx{\frac1x} &=& \ln(x) \\
  \intx{\frac1x} &=& \ln(x) + C \\
  \intx{\frac1x} &=& \ln(-x) \\
  \intx{\frac1x} &=& \ln(-x) + C \\
  \intx{\frac1x} &=& \ln(|x|) \\
  \intx{\frac1x} &=& \ln(|x|) + C \\
  \intx{\frac1x} &=& 
    \begin{cases}
      \ln(-x) + C_1 & \text{quando $x<0$}, \\
      \ln(x) + C_2 & \text{quando $x>0$} \\
    \end{cases} \\
  \end{array}
$$




\newpage

%  _____                _   _               
% |_   _|__   __ _  ___| |_| |__   ___ _ __ 
%   | |/ _ \ / _` |/ _ \ __| '_ \ / _ \ '__|
%   | | (_) | (_| |  __/ |_| | | |  __/ |   
%   |_|\___/ \__, |\___|\__|_| |_|\___|_|   
%            |___/                          
%
% «together»  (to ".together")
% (c2m222fpp 8 "together")
% (c2m222fpa   "together")
% (c2m212fpp 3 "together")
% (c2m212fpa   "together")
% (c2m202fpp 3 "together")
% (c2m202fp    "together")
% (find-fline       "~/LATEX/2020-1-C2/20201112_C2_fracoes_parciais_2.pdf")
\includegraphics[height=8cm]{2020-1-C2/20201112_C2_fracoes_parciais_2.pdf}

% (find-es "sympy" "tut-apart")
% (find-es "sympy" "tut-together")
% (find-es "maxima" "partial-fractions")
% (setq eepitch-preprocess-regexp "^")
% (setq eepitch-preprocess-regexp "^%T ")
%
%T  (eepitch-isympy)
%T  (eepitch-kill)
%T  (eepitch-isympy)
%T f = 1/(x+1) + 1/(x-1)
%T f
%T g = together(f)
%T g
%T apart(g)
%T
%T  (eepitch-maxima)
%T  (eepitch-kill)
%T  (eepitch-maxima)
%T f : 2/(x+3) + 4/(x-5);
%T g : ratsimp(f);
%T ff : partfrac(g, x);

\newpage

% «exercicio-3»  (to ".exercicio-3")
% (c2m222fpp 9 "exercicio-3")
% (c2m222fpa   "exercicio-3")

{\bf Exercício 3.}

\msk

a) $\D \togetherp{\frac{1}{x+1} + \frac{1}{x-1}} = \ColorRed{?}$ 

\ssk

b) $\D \togetherp{\frac{A}{x-a} + \frac{B}{x-b}} = \ColorRed{?}$ 

\ssk

c) $\D \togetherp{\frac{A}{x-a} + \frac{B}{x-b} + \frac{C}{x-c}} = \ColorRed{?}$ 

%        (find-fline "~/LATEX/2020-1-C2/20201112_C2_fracoes_parciais_3.pdf")
%\includegraphics[height=7cm]{2020-1-C2/20201112_C2_fracoes_parciais_3.pdf}

\newpage

% «exercicio-4»  (to ".exercicio-4")
% (c2m222fpp 10 "exercicio-4")
% (c2m222fpa    "exercicio-4")
% (c2m212fpp 5 "exercicio-3")
% (c2m212fpa   "exercicio-3")

{\bf Exercício 4.}

% (find-fline "~/LATEX/2020-1-C2/20201112_C2_fracoes_parciais_4.pdf")
\includegraphics[height=7cm]{2020-1-C2/20201112_C2_fracoes_parciais_4.pdf}

\newpage

% «exercicio-4a»  (to ".exercicio-4a")
% (c2m222fpp 11 "exercicio-4a")
% (c2m222fpa    "exercicio-4a")
% (c2m212fpp 6 "exercicio-3-maxima")
% (c2m212fpa   "exercicio-3-maxima")
% (find-es "maxima" "partial-fractions")

{\bf Exercício 4: uma solução pro item (a)}

% (setq eepitch-preprocess-regexp "^")
% (setq eepitch-preprocess-regexp "^%T ")
%
%T  (eepitch-maxima)
%T  (eepitch-kill)
%T  (eepitch-maxima)
%T f1 : A/(x-a) + B/(x-b);
%T f2 : ratsimp(f1);
%T (f1 = f2);
%T f3 : (c*x+d) / (x^2+e*x+f); 
%T fs : [e=-a-b, f=a*b, c=A+B, d=-A*b-a*B];
%T f4 : subst(fs, f3);
%T f2 - f4;
%T 
%T g1 : (2*x + 3) / (x^2 - 7*x + 10)$
%T g2 : partfrac(g1, x)$
%T (g1 = g2);
%T g2;
%T f1;
%T gs : [b=5, B=13/3, a=2, A=-7/3];
%T subst(gs, f1);
%T ratsimp(subst(gs, f1) - g1);

\sa{f1}{\frac{A}{x-a} + \frac{B}{x-b}}
\sa{f1.5}{\frac{A(x-b)}{(x-a)(x-b)} + \frac{B(x-a)}{(x-a)(x-b)}}
\sa{f1.6}{\frac{A(x-b) +                    B(x-a)}{(x-a)(x-b)}}
\sa{f2}{\frac {(A+B)x + (-Ab-Ba)} {x^2 + (-a-b)x + ab}}
\sa{f3}{\frac {cx+d} {x^2+ex+f}}
\sa{g1}{\frac {2x+3} {x^2-7x+10}}
\sa{g2}{\frac {2x+3} {(x-2)(x-5)}}
\sa{g3}{\frac {A(x-2)} {(x-2)(x-5)} + \frac {B(x-5)} {(x-2)(x-5)}}

\msk

a) $\scalebox{1.0}{$
    \begin{array}[t]{rcl}
      \ga{f1} &=& \ga{f3}   \\[7pt]
      \ga{f1} &=& \ga{f1.5} \\[6pt]
              &=& \ga{f1.6} \\[6pt]
              &=& \ga{f2}   \\[7pt]
            c &=& A+B    \\
            d &=& -Ab-Ba \\
            e &=& -a-b   \\
            f &=& ab     \\[15pt]
    \end{array}
    $}
   $

\newpage

% «exercicio-4a-2»  (to ".exercicio-4a-2")
% (c2m222fpp 12 "exercicio-4a-2")
% (c2m222fpa    "exercicio-4a-2")

{\bf Exercício 4: uma solução pro item (a), cont...}

\scalebox{0.6}{\def\colwidth{9cm}\firstcol{

Dá pra gente reescrever isso usando o `$[:=]$':
%
$$\begin{array}{l}
  \left( \ga{f1} \;\;=\;\; \ga{f3} \right)
    \bsm{ c:=A+B \\ d:=-Ab-Ba \\ e:=-a-b \\ f:=ab} \\[15pt]
  = \left( \ga{f1} \;\;=\;\; \ga{f2} \right),
 \end{array}
$$

e sabemos que esta igualdade é verdade:
%
$$\ga{f1} \;\;=\;\; \ga{f2}$$

então isto aqui
%
$$\begin{array}{rcl}
    c &=& A+B \\
    d &=& -Ab-Ba \\
    e &=& -a-b \\
    f &=& ab \\
  \end{array}
$$

é \ColorRed{uma} solução para a equação
%
$$\ga{f1} \;\;=\;\; \ga{f3} \;\; ...$$

mas não sabemos se é a \ColorRed{única} solução!

}\anothercol{

Sempre dá pra escrever soluções de equações

usando o `$[:=]$'. Por exemplo, as duas soluções

da equação
%
$$ (x-2)(x-5) = 0:$$

São:
%
$$\begin{array}{l}
    \left( (x-2)(x-5)=0 \right) [x:=2] \; = \\
    \left( (2-2)(2-5)=0 \right)             \\
    \left( (x-2)(x-5)=0 \right) [x:=5] \; = \\
    \left( (5-2)(5-5)=0 \right)             \\
  \end{array}
$$

\ColorRed{Nenhum} livro ``\ColorRed{básico}'' define

``solução de uma equação'' desse jeito ---

como ``a substituição que transforma a

equação numa igualdade verdadeira'' ---

mas eu acho isso um bom modo de

entender o que são ``equações'' e

``soluções''...

\msk

Ah, note que eu não fiquei repetindo a

condição ``as suas fórmulas para $c,d,e,f$

não podem conter `$x$'\,'' o tempo todo...

eu deixei isso implícito. \quad \smile


}}


\newpage

% «exercicio-4b»  (to ".exercicio-4b")
% (c2m222fpp 13 "exercicio-4b")
% (c2m222fpa    "exercicio-4b")

{\bf Exercício 4: uma solução pro item (b)}


\scalebox{1.0}{\def\colwidth{9cm}\firstcol{

Temos duas soluções para
%
$$(x-a)(x-b) = x^2-7x+10:$$

uma é $a=2$ e $b=5$, e a outra é $a=5$ e $b=2$.

Lembre que Cálculo 2 é sobre \ColorRed{chutar} e \ColorRed{testar}.

A gente pode chutar que $a=5$, $b=2$, e que

$c,d,e,f$ são os que a gente obtém pelo

item (a), e aí ver se isso nos leva a uma

solução...

\msk

(Obs: isso funciona!!!) 

}\anothercol{
}}





\newpage

% «exercicio-4c»  (to ".exercicio-4c")
% (c2m222fpp 14 "exercicio-4c")
% (c2m222fpa    "exercicio-4c")
% (c2m212fpp 9 "exercicio-3c")
% (c2m212fpa   "exercicio-3c")
% (find-books "__analysis/__analysis.el" "miranda")
% (find-dmirandacalcpage 246 "8.1.2 Fatores quadráticos")
% (find-dmirandacalcpage 251 "Exercícios")

{\bf Exercício 4: item (c)}

Seja [PFP] esta igualdade aqui -- o

``princípio por trás das frações parciais'':
%
$$\text{[PFP]} \;\;=\;\;
  \left(\ga{f1} \;\;=\;\; \ga{f1.6}
  \right)
$$

\msk

c) Resolva o exercício 8.7.2 do livro do Miranda --

\ssk

{\scriptsize

\url{http://hostel.ufabc.edu.br/~daniel.miranda/calculo/calculo.pdf#page=251}

}

\ssk

\def\rq{\ColorRed{?}}

e depois mostre qual é a substituição da forma
%
$$\text{[PFP]} \bsm{a:=\rq \\ b:=\rq \\ A:=\rq \\ B:=\rq}
$$

que ``está por trás'' da sua solução.



\newpage

% «exercicio-5»  (to ".exercicio-5")
% (c2m222fpp 15 "exercicio-5")
% (c2m222fpa    "exercicio-5")
% (c2m201fracparcp 8 "exercicio-4")
% (c2m201fracparc    "exercicio-4")

{\bf Exercício 5.}

\ssk

Use estas idéias para integrar:

$$\intx{\frac{2x^3 + 7x^2 + 7x + 3}{x+2}} \;\; = \;\; ?$$



\newpage

% «exercicio-6»  (to ".exercicio-6")
% (c2m222fpp 16 "exercicio-6")
% (c2m222fpa    "exercicio-6")

{\bf Exercício 6.}

\ssk

O que acontece nos casos em que ``teria vai um''?

\ssk

a) Tente fazer a divisão com resto de $x^3$ por $x+2$.

Mais precisamente, encontre um polinômios $R(x)$ e $Q(x)$ tais que

$(x^3) = Q(x) · (x+2) + R(x)$ e $R(x)$ é no máximo de grau 1.

Teste a sua resposta!

\ssk

b) Calcule $\intx{\frac{x^3}{x+2}}$ pelo método acima.

Teste a sua resposta derivando a sua antiderivada para $\frac{x^3}{x+2}$.

\ssk

c) Calcule $\intx{\frac{x^3}{x+2}}$ fazendo a substituição $u=x+2$.

Você deve obter o mesmo resultado que na (b).

\bsk

d) Calcule $\intx{\frac{x^2}{(x+1)(x-1)}}$ por frações parciais.


\newpage

{\bf Dica importante}

\ssk

Lembre que uns dos meus slogans é

``eu só vou corrigir os sinais de igual''...

No slide ?? a igualdade mais importante é a da última linha.

Nós vamos usá-la assim, pra transformar a integral original

em algo fácil de integrar:

\msk

% (find-fline "~/LATEX/2020-1-C2/20201119_C2_div_com_resto_4.pdf")
\includegraphics[height=4cm]{2020-1-C2/20201119_C2_div_com_resto_4.pdf}


\newpage

% «P1-2020.1»  (to ".P1-2020.1")
% (c2m202fpp 11 "P1-2020.1")
% (c2m202fp     "P1-2020.1")

{\bf Uma questão da P1 de 2020.1}

A questão 3 da P1 de 2020.1,

\ssk

% (c2m201p1p 5 "questao-3")
% (c2m201p1    "questao-3")
% (c2m201p1p 9 "gabarito-3a")
% (c2m201p1a   "gabarito-3a")
%    http://angg.twu.net/LATEX/2020-1-C2-P1.pdf
\url{http://angg.twu.net/LATEX/2020-1-C2-P1.pdf}

\ssk

era de frações parciais, e eu pus nesse PDF um gabarito

parcial dela, que não inclui nem as contas da divisão de

polinômios nem a verificação de que a nossa integral

está certa. Faça a questão, incluindo a parte que

não está no gabarito.




\GenericWarning{Success:}{Success!!!}  % Used by `M-x cv'

\end{document}

%  ____  _             _         
% |  _ \(_)_   ___   _(_)_______ 
% | | | | \ \ / / | | | |_  / _ \
% | |_| | |\ V /| |_| | |/ /  __/
% |____// | \_/  \__,_|_/___\___|
%     |__/                       
%
% «djvuize»  (to ".djvuize")
% (find-LATEXgrep "grep --color -nH --null -e djvuize 2020-1*.tex")

 (eepitch-shell)
 (eepitch-kill)
 (eepitch-shell)
# (find-fline "~/2022.2-C2/")
# (find-fline "~/LATEX/2022-2-C2/")
# (find-fline "~/bin/djvuize")

cd /tmp/
for i in *.jpg; do echo f $(basename $i .jpg); done

f () { rm -v $1.pdf;  textcleaner -f 50 -o  5 $1.jpg $1.png; djvuize $1.pdf; xpdf $1.pdf }
f () { rm -v $1.pdf;  textcleaner -f 50 -o 10 $1.jpg $1.png; djvuize $1.pdf; xpdf $1.pdf }
f () { rm -v $1.pdf;  textcleaner -f 50 -o 20 $1.jpg $1.png; djvuize $1.pdf; xpdf $1.pdf }

f () { rm -fv $1.png $1.pdf; djvuize $1.pdf }
f () { rm -fv $1.png $1.pdf; djvuize WHITEBOARDOPTS="-m 1.0 -f 15" $1.pdf; xpdf $1.pdf }
f () { rm -fv $1.png $1.pdf; djvuize WHITEBOARDOPTS="-m 1.0 -f 30" $1.pdf; xpdf $1.pdf }
f () { rm -fv $1.png $1.pdf; djvuize WHITEBOARDOPTS="-m 1.0 -f 45" $1.pdf; xpdf $1.pdf }
f () { rm -fv $1.png $1.pdf; djvuize WHITEBOARDOPTS="-m 0.5" $1.pdf; xpdf $1.pdf }
f () { rm -fv $1.png $1.pdf; djvuize WHITEBOARDOPTS="-m 0.25" $1.pdf; xpdf $1.pdf }
f () { cp -fv $1.png $1.pdf       ~/2022.2-C2/
       cp -fv        $1.pdf ~/LATEX/2022-2-C2/
       cat <<%%%
% (find-latexscan-links "C2" "$1")
%%%
}

f 20201213_area_em_funcao_de_theta
f 20201213_area_em_funcao_de_x
f 20201213_area_fatias_pizza



%  __  __       _        
% |  \/  | __ _| | _____ 
% | |\/| |/ _` | |/ / _ \
% | |  | | (_| |   <  __/
% |_|  |_|\__,_|_|\_\___|
%                        
% <make>

 (eepitch-shell)
 (eepitch-kill)
 (eepitch-shell)
# (find-LATEXfile "2019planar-has-1.mk")
make -f 2019.mk STEM=2022-2-C2-fracoes-parciais veryclean
make -f 2019.mk STEM=2022-2-C2-fracoes-parciais pdf

% Local Variables:
% coding: utf-8-unix
% ee-tla: "c2fp"
% ee-tla: "c2m222fp"
% End:

