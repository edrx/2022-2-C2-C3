% (find-LATEX "2022-2-C2-mathologermovel.tex")
% (defun c () (interactive) (find-LATEXsh "lualatex -record 2022-2-C2-mathologermovel.tex" :end))
% (defun C () (interactive) (find-LATEXsh "lualatex 2022-2-C2-mathologermovel.tex" "Success!!!"))
% (defun D () (interactive) (find-pdf-page      "~/LATEX/2022-2-C2-mathologermovel.pdf"))
% (defun d () (interactive) (find-pdftools-page "~/LATEX/2022-2-C2-mathologermovel.pdf"))
% (defun e () (interactive) (find-LATEX "2022-2-C2-mathologermovel.tex"))
% (defun o () (interactive) (find-LATEX "2022-2-C2-mathologermovel.tex"))
% (defun u () (interactive) (find-latex-upload-links "2022-2-C2-mathologermovel"))
% (defun v () (interactive) (find-2a '(e) '(d)))
% (defun d0 () (interactive) (find-ebuffer "2022-2-C2-mathologermovel.pdf"))
% (defun cv () (interactive) (C) (ee-kill-this-buffer) (v) (g))
%          (code-eec-LATEX "2022-2-C2-mathologermovel")
% (find-pdf-page   "~/LATEX/2022-2-C2-mathologermovel.pdf")
% (find-sh0 "cp -v  ~/LATEX/2022-2-C2-mathologermovel.pdf /tmp/")
% (find-sh0 "cp -v  ~/LATEX/2022-2-C2-mathologermovel.pdf /tmp/pen/")
%     (find-xournalpp "/tmp/2022-2-C2-mathologermovel.pdf")
%   file:///home/edrx/LATEX/2022-2-C2-mathologermovel.pdf
%               file:///tmp/2022-2-C2-mathologermovel.pdf
%           file:///tmp/pen/2022-2-C2-mathologermovel.pdf
% http://angg.twu.net/LATEX/2022-2-C2-mathologermovel.pdf
% (find-LATEX "2019.mk")
% (find-sh0 "cd ~/LUA/; cp -v Pict2e1.lua Pict2e1-1.lua Piecewise1.lua ~/LATEX/")
% (find-sh0 "cd ~/LUA/; cp -v Pict2e1.lua Pict2e1-1.lua Pict3D1.lua ~/LATEX/")
% (find-sh0 "cd ~/LUA/; cp -v C2Subst1.lua C2Formulas1.lua ~/LATEX/")
% (find-CN-aula-links "2022-2-C2-mathologermovel" "2" "c2m222mm" "c2mm")

% «.defs»	(to "defs")
% «.title»	(to "title")
% «.item-3»	(to "item-3")
%
% «.djvuize»	(to "djvuize")



% <videos>
% Video (not yet):
% (find-ssr-links     "c2m222mm" "2022-2-C2-mathologermovel")
% (code-eevvideo      "c2m222mm" "2022-2-C2-mathologermovel")
% (code-eevlinksvideo "c2m222mm" "2022-2-C2-mathologermovel")
% (find-c2m222mmvideo "0:00")

\documentclass[oneside,12pt]{article}
\usepackage[colorlinks,citecolor=DarkRed,urlcolor=DarkRed]{hyperref} % (find-es "tex" "hyperref")
\usepackage{amsmath}
\usepackage{amsfonts}
\usepackage{amssymb}
\usepackage{pict2e}
\usepackage[x11names,svgnames]{xcolor} % (find-es "tex" "xcolor")
\usepackage{colorweb}                  % (find-es "tex" "colorweb")
%\usepackage{tikz}
%
% (find-dn6 "preamble6.lua" "preamble0")
%\usepackage{proof}   % For derivation trees ("%:" lines)
%\input diagxy        % For 2D diagrams ("%D" lines)
%\xyoption{curve}     % For the ".curve=" feature in 2D diagrams
%
\usepackage{edrx21}               % (find-LATEX "edrx21.sty")
\input edrxaccents.tex            % (find-LATEX "edrxaccents.tex")
\input edrx21chars.tex            % (find-LATEX "edrx21chars.tex")
\input edrxheadfoot.tex           % (find-LATEX "edrxheadfoot.tex")
\input edrxgac2.tex               % (find-LATEX "edrxgac2.tex")
%\usepackage{emaxima}              % (find-LATEX "emaxima.sty")
%
%\usepackage[backend=biber,
%   style=alphabetic]{biblatex}            % (find-es "tex" "biber")
%\addbibresource{catsem-slides.bib}        % (find-LATEX "catsem-slides.bib")
%
% (find-es "tex" "geometry")
\usepackage[a6paper, landscape,
            top=1.5cm, bottom=.25cm, left=1cm, right=1cm, includefoot
           ]{geometry}
%
\begin{document}

\catcode`\^^J=10
\directlua{dofile "dednat6load.lua"}  % (find-LATEX "dednat6load.lua")
%L dofile "Piecewise1.lua"           -- (find-LATEX "Piecewise1.lua")
%L dofile "QVis1.lua"                -- (find-LATEX "QVis1.lua")
%L dofile "Pict3D1.lua"              -- (find-LATEX "Pict3D1.lua")
%L dofile "C2Formulas1.lua"          -- (find-LATEX "C2Formulas1.lua")
%L Pict2e.__index.suffix = "%"
\pu
\def\pictgridstyle{\color{GrayPale}\linethickness{0.3pt}}
\def\pictaxesstyle{\linethickness{0.5pt}}
\def\pictnaxesstyle{\color{GrayPale}\linethickness{0.5pt}}
\celllower=2.5pt

% «defs»  (to ".defs")
% (find-LATEX "edrx21defs.tex" "colors")
% (find-LATEX "edrx21.sty")

\def\u#1{\par{\footnotesize \url{#1}}}

\def\drafturl{http://angg.twu.net/LATEX/2022-2-C2.pdf}
\def\drafturl{http://angg.twu.net/2022.2-C2.html}
\def\draftfooter{\tiny \href{\drafturl}{\jobname{}} \ColorBrown{\shorttoday{} \hours}}



%  _____ _ _   _                               
% |_   _(_) |_| | ___   _ __   __ _  __ _  ___ 
%   | | | | __| |/ _ \ | '_ \ / _` |/ _` |/ _ \
%   | | | | |_| |  __/ | |_) | (_| | (_| |  __/
%   |_| |_|\__|_|\___| | .__/ \__,_|\__, |\___|
%                      |_|          |___/      
%
% «title»  (to ".title")
% (c2m222mmp 1 "title")
% (c2m222mma   "title")

\thispagestyle{empty}

\begin{center}

\vspace*{1.2cm}

{\bf \Large Cálculo 2 - 2022.2}

\bsk

Aula 2: derivação e integração com o Mathologermóvel

\bsk

Eduardo Ochs - RCN/PURO/UFF

\url{http://angg.twu.net/2022.2-C2.html}

\end{center}

\newpage

% (c2m221p1p 7 "escadas-defs")
% (c2m221p1a   "escadas-defs")

%L hx = function (x, y) return format(" (%s,%s)c--(%s,%s)o", x-1,y, x,y) end
%L hxs = function (ys)
%L     local str = ""
%L     for x,y in ipairs(ys) do str = str .. hx(x, y) end
%L     return str
%L   end
%L mtintegralspec = function (specf, xmax, y0)
%L     local pws = PwSpec.from(specf)
%L     local f = pws:fun()
%L     local ys = {[0] = y0}
%L     for x=1,xmax do
%L       PP("FOO", x, f(x-0.5), ys)
%L       ys[x] = ys[x - 1] + f(x - 0.5)
%L     end
%L     local strx = function (x) return tostring(v(x, ys[x])) end
%L     local specF = mapconcat(strx, seq(0, xmax), "--")
%L     return specF
%L   end
%L
%L ysf   = {1, 2, 1, 0, -1, -2, -1, 0, 1, 2, 1, 0}
%L specf = hxs(ysf)
%L ysg   = {0, 1, 2, 3, -2, -1, 0, -1, -2, 3, 2, 1, 0}
%L specg = hxs(ysg)
%L specF = mtintegralspec(specf, #ysf,  0)
%L specG = mtintegralspec(specf, #ysf, -3)
%L specI = mtintegralspec(specg, #ysg,  0)
%L pwsf  = PwSpec.from(specf)
%L pwsg  = PwSpec.from(specg)
%L pwsF  = PwSpec.from(specF)
%L pwsG  = PwSpec.from(specG)
%L pwsI  = PwSpec.from(specI)
%L pf    = pwsf:topict():setbounds(v(0,-2), v(#ysf,2)):pgat("pgatc")
%L pg    = pwsg:topict():setbounds(v(0,-2), v(#ysg,3)):pgat("pgatc")
%L pF    = pwsF:topict():setbounds(v(0,-0), v(#ysf,4)):pgat("pgatc")
%L pG    = pwsG:topict():setbounds(v(0,-3), v(#ysf,1)):pgat("pgatc")
%L pI    = pwsI:topict():setbounds(v(0,0),  v(#ysg,6)):pgat("pgatc")
%L pf:sa("Fig f"):output()
%L pg:sa("Fig g"):output()
%L pF:sa("Fig F"):output()
%L pG:sa("Fig G"):output()
%L pI:sa("Fig I"):output()
%L
%L PictList{}:setbounds(v(0,-4),v(13,4)):pgat("pgatc"):sa("respgrid"):output()
%L
%L mtintegralspec2 = function (x0, y0, Dys, dot0, dot1)
%L     local mkxy = function (x,y) return format("(%d,%d)", x, y) end
%L     local xys = { mkxy(x0,y0) .. (dot0 or "") }
%L     local x,y = x0,y0
%L     for i,Dy in ipairs(Dys) do
%L       x = x + 1
%L       y = y + Dy
%L       table.insert(xys, mkxy(x,y))
%L     end
%L     xys[#xys] = xys[#xys] .. (dot1 or "")
%L     return table.concat(xys, "--")
%L   end
%L
%L -- = mtintegralspec2(10, 20, {1, 2, -3, -3}, "a", "b")
%L ysf   = {1, 2, 1, 0, -1, -2, -1, 0, 1, 2, 1, 0}
%L ysf_  = {1, 2, 1, 0, -1, -2, -1}
%L ysg   = {0, 1, 2, 3, -2, -1,  0, -1, -2, 3, 2, 1, 0}
%L ysg_  =                         {-1, -2, 3, 2, 1}
%L specH = mtintegralspec2(0, -4, ysf_, "", "o\n") ..
%L         mtintegralspec2(7,  1, ysg_, "o", "")
%L specM = mtintegralspec2(0, -4, ysf_, "", "o\n") ..
%L         mtintegralspec2(7,  2, ysg_, "o", "")
%L -- = specH
%L -- = specM
%L pwsH  = PwSpec.from(specH)
%L pwsM  = PwSpec.from(specM)
%L pH    = pwsH:topict():setbounds(v(0,-4), v(12,4)):pgat("pgatc")
%L pM    = pwsM:topict():setbounds(v(0,-4),  v(12,5)):pgat("pgatc")
%L pH:sa("Fig H"):output()
%L pM:sa("Fig M"):output()
\pu



\newpage

Este PDF vai ser refeito depois!

Por enquanto:

\msk

1) assista a parte do vídeo do Mathologer sobre como usar um carro pra
derivar e integrar --- essa parte começa no 3:12. Link:

\ssk

{\footnotesize

%    http://angg.twu.net/mathologer-calculus-easy.html#03:08
\url{http://angg.twu.net/mathologer-calculus-easy.html\#03:08}

}

\ssk

Repare que ele sempre põe o gráfico da distância em cima e o gráfico
da velocidade embaixo; quando ele fala de derivação ele começa com uma
função ``original'', $f$, em cima e ele desenha, ou escreve, a
derivada dela, $f'$, embaixo.

\msk

2) O Leithold define a inclinação de uma reta na página 17 (no
capítulo 1) e na página 150 (no capítulo 3) ele discute a derivada da
função $|x|$. Leia estes trechos.

% (find-books "__analysis/__analysis.el" "leithold")
% (find-leitholdptpage (+ 17  17)   "inclinação")
% (find-leitholdptpage (+ 17 150)   "|x|")

\newpage

% «item-3»  (to ".item-3")

3) Considere que a função $G(x)$ do exercício 4 daqui

\ssk

{\footnotesize

% (c2m221tfc1p 10 "exercicio-4")
% (c2m221tfc1a    "exercicio-4")
%    http://angg.twu.net/LATEX/2022-1-C2-TFC1.pdf#page=10
\url{http://angg.twu.net/LATEX/2022-1-C2-TFC1.pdf#page=10}

}

\ssk

é um gráfico da posição do mathologermóvel no tempo. Copie esse
gráfico num papel e abaixo dele faça o gráfico correspondente da
velocidade do mathologermóvel no tempo.

\msk

Tem uma espécie de gabarito desse exercício aqui:

\ssk

{\footnotesize

% (c2m212mt3p 4 "gabarito")
% (c2m212mt3a   "gabarito")
%    http://angg.twu.net/LATEX/2021-2-C2-MT3.pdf#page=4
\url{http://angg.twu.net/LATEX/2021-2-C2-MT3.pdf#page=4}

}





\newpage

4) Na P1 do semestre passado --- link:

\ssk

{\footnotesize

% (c2m221p1p 7 "escadas")
% (c2m221p1a   "escadas")
%    http://angg.twu.net/LATEX/2022-1-C2-P1.pdf#page=7
\url{http://angg.twu.net/LATEX/2022-1-C2-P1.pdf\#page=7}

}

\ssk

eu defini as funções $f(x)$ e $g(x)$ desta forma:

\unitlength=9pt

\bsk

$f(x) = \ga{Fig f}
 \qquad
 g(x) = \ga{Fig g}
$

% $
% \ga{Fig F}
% \ga{Fig G}
% \ga{Fig I}
% \ga{Fig M}
% $

\msk

Interprete esses gráficos da $f(x)$ e da $g(x)$ como dois gráficos
diferentes da velocidade do mathologermóvel no tempo. Copie elas num
papel e acima de cada um deles faça o gráfico correspondente da
posição do mathologermóvel no tempo.

\msk

Tem uma espécie de gabarito disso aqui:

\ssk

{\footnotesize

% (c2m221p1p 8 "escadas-gab")
% (c2m221p1a   "escadas-gab")
%    http://angg.twu.net/LATEX/2022-1-C2-P1.pdf#page=8
\url{http://angg.twu.net/LATEX/2022-1-C2-P1.pdf#page=8}

}






\newpage


\msk

5) Faça o exercício 1 daqui:

\ssk

{\footnotesize

% (c2m221tfc1p 7 "exercicio-1")
% (c2m221tfc1a   "exercicio-1")
%    http://angg.twu.net/LATEX/2022-1-C2-TFC1.pdf#page=7
\url{http://angg.twu.net/LATEX/2022-1-C2-TFC1.pdf\#page=7}

}

\ssk

Pra fazer ele você vai ter que interpretar o gráfico da $f(x)$ como um
gráfico de velocidade, e você vai que interpretar expressões como esta aqui
%
$$\Intx{1.5}{2}{f(x)}$$
%
como o quanto a posição do mathologermóvel varia entre o ``instante
inicial'', que é $t=1.5$, e o ``instante final'', que é $t=2$.











% (find-TH "mathologer-calculus-easy" "legendas")
% (find-TH "mathologer-calculus-easy" "legendas" "03:08")




%\printbibliography

\GenericWarning{Success:}{Success!!!}  % Used by `M-x cv'

\end{document}

%  ____  _             _         
% |  _ \(_)_   ___   _(_)_______ 
% | | | | \ \ / / | | | |_  / _ \
% | |_| | |\ V /| |_| | |/ /  __/
% |____// | \_/  \__,_|_/___\___|
%     |__/                       
%
% «djvuize»  (to ".djvuize")
% (find-LATEXgrep "grep --color -nH --null -e djvuize 2020-1*.tex")

 (eepitch-shell)
 (eepitch-kill)
 (eepitch-shell)
# (find-fline "~/2022.2-C2/")
# (find-fline "~/LATEX/2022-2-C2/")
# (find-fline "~/bin/djvuize")

cd /tmp/
for i in *.jpg; do echo f $(basename $i .jpg); done

f () { rm -v $1.pdf;  textcleaner -f 50 -o  5 $1.jpg $1.png; djvuize $1.pdf; xpdf $1.pdf }
f () { rm -v $1.pdf;  textcleaner -f 50 -o 10 $1.jpg $1.png; djvuize $1.pdf; xpdf $1.pdf }
f () { rm -v $1.pdf;  textcleaner -f 50 -o 20 $1.jpg $1.png; djvuize $1.pdf; xpdf $1.pdf }

f () { rm -fv $1.png $1.pdf; djvuize $1.pdf }
f () { rm -fv $1.png $1.pdf; djvuize WHITEBOARDOPTS="-m 1.0 -f 15" $1.pdf; xpdf $1.pdf }
f () { rm -fv $1.png $1.pdf; djvuize WHITEBOARDOPTS="-m 1.0 -f 30" $1.pdf; xpdf $1.pdf }
f () { rm -fv $1.png $1.pdf; djvuize WHITEBOARDOPTS="-m 1.0 -f 45" $1.pdf; xpdf $1.pdf }
f () { rm -fv $1.png $1.pdf; djvuize WHITEBOARDOPTS="-m 0.5" $1.pdf; xpdf $1.pdf }
f () { rm -fv $1.png $1.pdf; djvuize WHITEBOARDOPTS="-m 0.25" $1.pdf; xpdf $1.pdf }
f () { cp -fv $1.png $1.pdf       ~/2022.2-C2/
       cp -fv        $1.pdf ~/LATEX/2022-2-C2/
       cat <<%%%
% (find-latexscan-links "C2" "$1")
%%%
}

f 20201213_area_em_funcao_de_theta
f 20201213_area_em_funcao_de_x
f 20201213_area_fatias_pizza



%  __  __       _        
% |  \/  | __ _| | _____ 
% | |\/| |/ _` | |/ / _ \
% | |  | | (_| |   <  __/
% |_|  |_|\__,_|_|\_\___|
%                        
% <make>

 (eepitch-shell)
 (eepitch-kill)
 (eepitch-shell)
# (find-LATEXfile "2019planar-has-1.mk")
make -f 2019.mk STEM=2022-2-C2-mathologermovel veryclean
make -f 2019.mk STEM=2022-2-C2-mathologermovel pdf

% Local Variables:
% coding: utf-8-unix
% ee-tla: "c2mm"
% ee-tla: "c2m222mm"
% End:
