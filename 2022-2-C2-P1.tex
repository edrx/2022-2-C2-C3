% (find-LATEX "2022-2-C2-P1.tex")
% (defun c  () (interactive) (find-LATEXsh "lualatex -record 2022-2-C2-P1.tex" :end))
% (defun C  () (interactive) (find-LATEXsh "lualatex 2022-2-C2-P1.tex" "Success!!!"))
% (defun D  () (interactive) (find-pdf-page      "~/LATEX/2022-2-C2-P1.pdf"))
% (defun d  () (interactive) (find-pdftools-page "~/LATEX/2022-2-C2-P1.pdf"))
% (defun e  () (interactive) (find-LATEX "2022-2-C2-P1.tex"))
% (defun o  () (interactive) (find-LATEX "2022-2-C2-dicas-pra-P1.tex"))
% (defun oo () (interactive) (find-LATEX "2022-1-C2-P1.tex"))
% (defun u  () (interactive) (find-latex-upload-links "2022-2-C2-P1"))
% (defun v  () (interactive) (find-2a '(e) '(d)))
% (defun d0 () (interactive) (find-ebuffer "2022-2-C2-P1.pdf"))
% (defun cv () (interactive) (C) (ee-kill-this-buffer) (v) (g))
%          (code-eec-LATEX "2022-2-C2-P1")
% (find-pdf-page   "~/LATEX/2022-2-C2-P1.pdf")
% (find-sh0 "cp -v  ~/LATEX/2022-2-C2-P1.pdf /tmp/")
% (find-sh0 "cp -v  ~/LATEX/2022-2-C2-P1.pdf /tmp/pen/")
%     (find-xournalpp "/tmp/2022-2-C2-P1.pdf")
%   file:///home/edrx/LATEX/2022-2-C2-P1.pdf
%               file:///tmp/2022-2-C2-P1.pdf
%           file:///tmp/pen/2022-2-C2-P1.pdf
% http://angg.twu.net/LATEX/2022-2-C2-P1.pdf
% (find-LATEX "2019.mk")
% (find-sh0 "cd ~/LUA/; cp -v Pict2e1.lua Pict2e1-1.lua Piecewise1.lua ~/LATEX/")
% (find-sh0 "cd ~/LUA/; cp -v Pict2e1.lua Pict2e1-1.lua Pict3D1.lua ~/LATEX/")
% (find-sh0 "cd ~/LUA/; cp -v C2Subst1.lua C2Formulas1.lua ~/LATEX/")
% (find-CN-aula-links "2022-2-C2-P1" "2" "c2m222p1" "c2p1")

% «.defs»		(to "defs")
% «.defs-T-and-B»	(to "defs-T-and-B")
% «.title»		(to "title")
% «.questao-1»		(to "questao-1")
%   «.subst-trig»	(to "subst-trig")
% «.questao-2»		(to "questao-2")
%   «.int-por-partes»	(to "int-por-partes")
% «.questao-3»		(to "questao-3")
%   «.TFC2»		(to "TFC2")
% «.questao-4»		(to "questao-4")
%   «.fracoes-parciais»	(to "fracoes-parciais")
% «.questao-5»		(to "questao-5")
%   «.mathologermovel»	(to "mathologermovel")
%   «.questao-5-grids»	(to "questao-5-grids")
%
% «.questao-1-gab»	(to "questao-1-gab")
% «.questao-2-gab»	(to "questao-2-gab")
% «.questao-3-gab»	(to "questao-3-gab")
% «.questao-4-gab»	(to "questao-4-gab")
% «.questao-5-gab»	(to "questao-5-gab")



% <videos>
% Video (not yet):
% (find-ssr-links     "c2m222p1" "2022-2-C2-P1")
% (code-eevvideo      "c2m222p1" "2022-2-C2-P1")
% (code-eevlinksvideo "c2m222p1" "2022-2-C2-P1")
% (find-c2m222p1video "0:00")

\documentclass[oneside,12pt]{article}
\usepackage[colorlinks,citecolor=DarkRed,urlcolor=DarkRed]{hyperref} % (find-es "tex" "hyperref")
\usepackage{amsmath}
\usepackage{amsfonts}
\usepackage{amssymb}
\usepackage{pict2e}
\usepackage[x11names,svgnames]{xcolor} % (find-es "tex" "xcolor")
\usepackage{colorweb}                  % (find-es "tex" "colorweb")
%\usepackage{tikz}
%
% (find-dn6 "preamble6.lua" "preamble0")
%\usepackage{proof}   % For derivation trees ("%:" lines)
%\input diagxy        % For 2D diagrams ("%D" lines)
%\xyoption{curve}     % For the ".curve=" feature in 2D diagrams
%
\usepackage{edrx21}               % (find-LATEX "edrx21.sty")
\input edrxaccents.tex            % (find-LATEX "edrxaccents.tex")
\input edrx21chars.tex            % (find-LATEX "edrx21chars.tex")
\input edrxheadfoot.tex           % (find-LATEX "edrxheadfoot.tex")
\input edrxgac2.tex               % (find-LATEX "edrxgac2.tex")
%\usepackage{emaxima}              % (find-LATEX "emaxima.sty")
%
%\usepackage[backend=biber,
%   style=alphabetic]{biblatex}            % (find-es "tex" "biber")
%\addbibresource{catsem-slides.bib}        % (find-LATEX "catsem-slides.bib")
%
% (find-es "tex" "geometry")
\usepackage[a6paper, landscape,
            top=1.5cm, bottom=.25cm, left=1cm, right=1cm, includefoot
           ]{geometry}
%
\begin{document}

\catcode`\^^J=10
\directlua{dofile "dednat6load.lua"}  % (find-LATEX "dednat6load.lua")
%L dofile "Piecewise1.lua"           -- (find-LATEX "Piecewise1.lua")
%L -- dofile "QVis1.lua"             -- (find-LATEX "QVis1.lua")
%L -- dofile "Pict3D1.lua"           -- (find-LATEX "Pict3D1.lua")
%L -- dofile "C2Formulas1.lua"       -- (find-LATEX "C2Formulas1.lua")
%L Pict2e.__index.suffix = "%"
\pu
\def\pictgridstyle{\color{GrayPale}\linethickness{0.3pt}}
\def\pictaxesstyle{\linethickness{0.5pt}}
\def\pictnaxesstyle{\color{GrayPale}\linethickness{0.5pt}}
\celllower=2.5pt

% «defs»  (to ".defs")
% (find-LATEX "edrx21defs.tex" "colors")
% (find-LATEX "edrx21.sty")

\def\u#1{\par{\footnotesize \url{#1}}}

\def\drafturl{http://angg.twu.net/LATEX/2022-2-C2.pdf}
\def\drafturl{http://angg.twu.net/2022.2-C2.html}
\def\draftfooter{\tiny \href{\drafturl}{\jobname{}} \ColorBrown{\shorttoday{} \hours}}

\sa{[IP]}{\CFname{IP}{}}
\sa{[TFC2]}{\CFname{TFC2}{}}

% «defs-T-and-B»  (to ".defs-T-and-B")
\long\def\ColorOrange#1{{\color{orange!90!black}#1}}
\def\T(Total: #1 pts){{\bf(Total: #1)}}
\def\T(Total: #1 pts){{\bf(Total: #1 pts)}}
\def\T(Total: #1 pts){\ColorRed{\bf(Total: #1 pts)}}
\def\B       (#1 pts){\ColorOrange{\bf(#1 pts)}}



%  _____ _ _   _                               
% |_   _(_) |_| | ___   _ __   __ _  __ _  ___ 
%   | | | | __| |/ _ \ | '_ \ / _` |/ _` |/ _ \
%   | | | | |_| |  __/ | |_) | (_| | (_| |  __/
%   |_| |_|\__|_|\___| | .__/ \__,_|\__, |\___|
%                      |_|          |___/      
%
% «title»  (to ".title")
% (c2m222p1p 1 "title")
% (c2m222p1a   "title")

\thispagestyle{empty}

\begin{center}

\vspace*{1.2cm}

{\bf \Large Cálculo 2 - 2022.2}

\bsk

P1 (Primeira prova)

\bsk

Eduardo Ochs - RCN/PURO/UFF

\url{http://angg.twu.net/2022.2-C2.html}

\end{center}

\newpage

% «questao-1»  (to ".questao-1")
% (c2m222p1p 1 "questao-1")
% (c2m222p1a   "questao-1")
% «subst-trig»  (to ".subst-trig")
% (c2m222p1p 2 "subst-trig")
% (c2m222p1a   "subst-trig")
% (c2m222mva "title")
% (c2m222mva "title" "Aula 10: Mudança de variáveis")
% (c2m222tudop 49)
% (c2m222striga "title")
% (c2m222striga "title" "Aulas 11 e 12: substituição trigonométrica")
% (find-es "maxima" "subst-trig-questions" "FF(3,1,4)")

%\vspace*{-0.4cm}


\scalebox{0.6}{\def\colwidth{9cm}\firstcol{

{\bf Questão 1}

\T(Total: 3.0 pts)

\msk

Calcule:

$$\intx{x^3 \sqrt{1-4x^2}}\;.$$

\bsk

{\bf Dicas:} 1) Você provavelmente vai precisar de pelo menos duas
mudanças de variável pra chegar no resultado final. 2) No curso nós
vimos dois modos de fazer mudanças de variável de um jeito legível: um
modo usava chaves sob subexpressões e o outro modo usava ``caixinhas
de anotações'' como a abaixo,
%
$$\bmat{
  u = \sen x \\
  \frac{du}{dx} = \frac{d}{dx} \sen x = \cos x \\
  \cos x \, dx = du \\
  x = \arcsen u \\
  }
$$

em que todas as outras linhas da caixinha eram consequência da
primeira.

}\anothercol{

% «questao-2»       (to ".questao-2")
% «int-por-partes»  (to ".int-por-partes")
% (c2m222p1p 2 "int-por-partes")
% (c2m222p1a   "int-por-partes")
% (c2m222ippp 1 "title")
% (c2m222ippa   "title")

{\bf Questão 2}

\T(Total: 2.0 pts)

\msk

No curso nós definimos que {\sl pra nós} a ``fórmula da integração por
partes'' seria esta aqui:
%
$$\ga{[IP]}
  \;=\;
  \left( \intx{fg'} \;\;=\;\; fg - \intx{f'g} \right)
$$

Mostre que aplicando integração por partes três vezes dá pra obter uma
fórmula que transforma a integral $\intx{x^3 h'''(x)}$ em algo bem
mais simples. Aqui você vai poder omitir os argumentos das funções se
quiser --- note que na \ga{[IP]} eu abreviei, por exemplo, `$f(x)$'
para `$f$'.

\msk

Nesta questão eu vou ver principalmente se você sabe os truques pra
deixar as contas dela organizadas e legíveis.



}}



\newpage

\scalebox{0.6}{\def\colwidth{9cm}\firstcol{

% «questao-3»  (to ".questao-3")
% «TFC2»       (to ".TFC2")

{\bf Questão 3}

\T(Total: 2.0 pts)

\msk

No curso nós definimos que {\sl pra nós} a ``fórmula'' do TFC2 seria
esta aqui:
%
$$\ga{[TFC2]}
  \;=\;
  \left( \Intx{a}{b}{F'(x)} \;=\; \difx{a}{b}{F(x)} \right)
$$

Mostre que quando $a=1$, $b=3$ e
%
$$F(x) =
  \begin{cases}
     x & \text{quando $x<2$}, \\
     2x & \text{quando $x≥2$} \\
  \end{cases}
$$

a fórmula $\ga{[TFC2]}$ é falsa.

\msk

Dicas: o melhor modo de fazer isto é representando graficamente $F(x)$
e $F'(x)$ e calculando certas coisas a partir dos gráficos. Considere
que o leitor sabe calcular áreas de retângulos, triângulos e trapézios
no olhômetro quando as coordenadas deles são números simples, mas
complemente os seus gráficos com um pouquinho de português quando nem
tudo for óbvio só a partir dos gráficos.



}\anothercol{

% «questao-4»         (to ".questao-4")
% «fracoes-parciais»  (to ".fracoes-parciais")
% (c2m222p1p 3 "fracoes-parciais")
% (c2m222p1a   "fracoes-parciais")
% (c2m222fpp 1 "title")
% (c2m222fpa   "title")

{\bf Questão 4}

\T(Total: 2.0 pts)

\msk

Calcule:

$$\intx{\frac{4x+5}{(x-2)(x+3)}}$$

\msk

e teste o seu resultado.


\bsk
\bsk
\bsk
\bsk
\msk

% «questao-5»        (to ".questao-5")
% «mathologermovel»  (to ".mathologermovel")
% (c2m222p1p 3 "questao-5")
% (c2m222p1a   "questao-5")

{\bf Questão 5}

\T(Total: 1.0 pts)

\msk

Seja $f(x)$ a função no topo da página seguinte.

Seja
%
$$F(x) \;=\; \Intt{2}{x}{f(x)}.$$

Desenhe o gráfico de $F(x)$ em algum dos grids vazios da próxima
página. Indique claramente qual é a versão final e quais desenhos são
rascunhos.

}}

\newpage

% «questao-5-grids»  (to ".questao-5-grids")
% (c2m222p1p 4 "questao-5-grids")
% (c2m222p1a   "questao-5-grids")

%L -- (find-angg "LUA/Pict2e1-1.lua" "FromYs")
%L fryF = FromYs.fromys({0,-1,1,-2,2,-3,3,-3,2,-2,1,-1,0}):getYs(1)
%L fryF:getypict():pgat("pgatc"):sa("fig f"):output()
%L fryF:getYpict():pgat("pgatc"):sa("fig F"):output()
%L fryF:getYgrid(-4,4):
%L                pgat("pgatc"):sa("grid F"):output()
\pu

\unitlength=8pt

$\begin{array}{ll}
 \ga{fig f}  \phantom{mm} & \ga{fig f}  \\ \\
 \ga{grid F} & \ga{grid F} \\ \\
 \ga{grid F} & \ga{grid F} \\
 \end{array}
$


\newpage

% «questao-1-gab»  (to ".questao-1-gab")
% (c2m222p1p 5 "questao-1-gab")
% (c2m222p1a   "questao-1-gab")

% (setq eepitch-preprocess-regexp "^")
% (setq eepitch-preprocess-regexp "^%T ")
%
%T  (eepitch-maxima)
%T  (eepitch-kill)
%T  (eepitch-maxima)
%T s : sqrt(1-4*x^2);
%T f : x^3 * s;
%T F : integrate(f, x);
%T G : (1/16) * (s^5/5 - s^3/3);
%T g : diff(G, x);
%T expand(rat(g*s));
%T expand(rat(f*s));

{\bf Questão 1: gabarito}

$$\scalebox{0.9}{$
  \begin{array}{rcl}
  \intx{x^3 \sqrt{1-4x^2}}
    &=& \intu{\frac18 u^3 \sqrt{1-u^2}·\frac12} \\
    &=& \frac1{16}\intu{u^3 \sqrt{1-u^2}} \\
    &=& \frac1{16}\intth{(\senθ)^3 (\cosθ)(\cosθ)} \\
    &=& \frac1{16}\intth{(\cosθ)^2 (\senθ)^2 (\senθ)} \\
    &=& \frac1{16}\intc{c^2 (1-c^2)(-1)} \\
    &=& \frac1{16}\intc{c^2 (c^2-1)} \\
    &=& \frac1{16}\intc{c^4 - c^2} \\
    &=& \frac1{16}(\frac{c^5}{5} - \frac{c^3}{3}) \\
    &=& \frac1{16}(\frac{(\cosθ)^5}{5} - \frac{(\cosθ)^3}{3}) \\
    &=& \frac1{16}(\frac{\sqrt{1-u^2}^5}{5} - \frac{\sqrt{1-u^2}^3}{3}) \\
    &=& \frac1{16}(\frac{\sqrt{1-4x^2}^5}{5} - \frac{\sqrt{1-4x^2}^3}{3}) \\
  \end{array}
  \quad
  \begin{array}{l}
    \bsm{u = 2x \\
         u^2 = 4x^2 \\
         x = u/2 \\
         x^3 = u^3/8 \\
         du = 2dx \\
         dx = \frac12 du \\
        } \\
    \\[-7pt]
    \bsm{u = \senθ \\
         u^2 = (\senθ)^2 \\
         1-u^2 = (\cosθ)^2 \\
         \sqrt{1-u^2} = \cosθ \\
         \frac{du}{dθ} = \cosθ \\
         du = \cosθ\,dθ \\
        } \\
    \\[-7pt]
    \bsm{c = \cosθ \\
         \frac{dc}{dθ} = -\senθ \\
         dc = -\senθ\,dθ \\
         (-1)dc = \senθ\,dθ \\
         (\senθ)^2 = 1-c^2 \\
        } \\
  \end{array}
  $}
$$



\newpage

% «questao-2-gab»  (to ".questao-2-gab")
% (c2m222p1p 6 "questao-2-gab")
% (c2m222p1a   "questao-2-gab")

{\bf Questão 2: gabarito}

$$\begin{array}{rcl}
  \intx{x^3 h'''}
    &=& x^3 h'' - \intx{3x^2 h''} \\
    &=& x^3 h'' - 3\intx{x^2 h''} \\
  \intx{x^2 h''}
    &=& x^2 h' - \intx{2x h'} \\
    &=& x^2 h' - 2\intx{x h'} \\
  \intx{x h'}
    &=& x h - \intx{1·h} \\
    &=& x h - \intx{h} \\
  \\[-5pt]
  \intx{x^3 h'''}
    &=& x^3 h'' - 3\intx{x^2 h''} \\
    &=& x^3 h'' - 3(x^2 h' - 2\intx{x h'}) \\
    &=& x^3 h'' - 3(x^2 h' - 2(x h - \intx{h})) \\
    &=& x^3 h'' - 3 x^2 h' + 6(x h - \intx{h}) \\
    &=& x^3 h'' - 3 x^2 h' + 6x h - 6\intx{h}) \\
  \end{array}
$$

\newpage

% «questao-3-gab»  (to ".questao-3-gab")
% (c2m222p1p 7 "questao-3-gab")
% (c2m222p1a   "questao-3-gab")
% (c2m221atisp 21 "1-then-2")
% (c2m221atisa    "1-then-2")

{\bf Questão 3: gabarito}

%L Pict2e.bounds = PictBounds.new(v(0,0), v(4,8))
%L spec = "(0,0)--(2,2)c (2,4)o--(4,8)"
%L pws = PwSpec.from(spec)
%L pws:topict():prethickness("1pt"):pgat("pgatc"):sa("F(x)"):output()
%L
%L Pict2e.bounds = PictBounds.new(v(0,0), v(4,8))
%L spec = "(0,1)--(2,1)o (2,2)o--(4,2)"
%L pws = PwSpec.from(spec)
%L pws:topict():prethickness("1pt"):pgat("pgatc"):sa("F'(x)"):output()
%L
%L spec = "(0,1)--(2,1)o (2,2)c--(4,2)"
%L pwsa = PwSpec.from(spec)
%L pf = PictList{
%L   pwsa:topwfunction():areaify(1, 3):Color("Orange"),
%L   pws:topict()
%L }
%L pf:pgat("pgatc"):sa("int F'(x)"):output()

\pu

\msk

\unitlength=5pt

$$F(x) = \ga{F(x)}
 \quad
 F'(x) = \ga{F'(x)}
 \quad
 \textstyle \Intx{1}{3}{F'(x)} = \ga{int F'(x)} = 3
$$

\def\und#1#2{\underbrace{#1}_{#2}}

$$\und{
  \und{\Intx{1}{3}{F'(x)}}{3} \;=\;
  \und{\und{\und{\difx{1}{3}{F(x)}}{F(3)-F(1)}}{6-1}}{5}
  }{\False}
$$

% (c2m221vsbp 6 "questao-1-gab")
% (c2m221vsba   "questao-1-gab")

\newpage

% «questao-4-gab»  (to ".questao-4-gab")
% (c2m222p1p 8 "questao-4-gab")
% (c2m222p1a   "questao-4-gab")

{\bf Questão 4: gabarito}

$$\scalebox{0.8}{$
  \begin{array}{rcl}
  \frac{4x+5}{(x-2)(x+3)} &=& \frac{A}{x-2} + \frac{B}{x+3} \\
                          &=& \frac{A(x+3)}{(x-2)(x+3)} + \frac{B(x-2)}{(x-2)(x+3)} \\
                          &=& \frac{A(x+3)+B(x-2)}{(x-2)(x+3)} \\
                          &=& \frac{Ax+3A+Bx-2B}{(x-2)(x+3)} \\
                          &=& \frac{(A+B)x+(3A-2B)}{(x-2)(x+3)} \\
                          \\[-5pt]
                     4x+5 &=& (A+B)x+(3A-2B) \\
                      A+B &=& 4 \\
                    3A-2B &=& 5 \\
                        A &=& 13/5 \\
                        B &=& 7/5 \\
                          \\[-5pt]
        \frac{4x+5}{(x-2)(x+3)}  &=& \frac{13/5}{x-2} + \frac{7/5}{x+3} \\
  \intx{\frac{4x+5}{(x-2)(x+3)}} &=& \intx{\frac{13/5}{x-2} + \frac{7/5}{x+3}} \\
                                 &=& \frac{13}{5}\intx{\frac{1}{x-2}}
                                   + \frac{7}{5}\intx{\frac{1}{x+3}} \\
                                 &=& \frac{13}{5} \ln |x-2|
                                   + \frac{7}{5}  \ln |x+3| \\
  \end{array}
  $}
$$


% (setq eepitch-preprocess-regexp "^")
% (setq eepitch-preprocess-regexp "^%T ?")
%
%T  (eepitch-maxima)
%T  (eepitch-kill)
%T  (eepitch-maxima)
%T linsolve ([A+B=4, 3*A-2*B=5], [A, B]);
%T
%T f : (4*x + 5) / ((x-2)*(x+3));
%T partfrac(f, x);
%T F : integrate(f, x);

\newpage

% «questao-5-gab»  (to ".questao-5-gab")
% (c2m222p1p 9 "questao-5-gab")
% (c2m222p1a   "questao-5-gab")

{\bf Questão 5: gabarito}

\unitlength=10pt

$$\begin{array}{r}
 f(x) \;=\; \ga{fig f}  \\ \\
 F(x) \;=\; \Intt{2}{x}{f(t)}
      \;=\; \ga{fig F}  \\
 \end{array}
$$





\GenericWarning{Success:}{Success!!!}  % Used by `M-x cv'

\end{document}



% Local Variables:
% coding: utf-8-unix
% ee-tla: "c2p1"
% ee-tla: "c2m222p1"
% End:
