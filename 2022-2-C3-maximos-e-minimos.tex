% (find-LATEX "2022-2-C3-maximos-e-minimos.tex")
% (defun c () (interactive) (find-LATEXsh "lualatex -record 2022-2-C3-maximos-e-minimos.tex" :end))
% (defun C () (interactive) (find-LATEXsh "lualatex 2022-2-C3-maximos-e-minimos.tex" "Success!!!"))
% (defun D () (interactive) (find-pdf-page      "~/LATEX/2022-2-C3-maximos-e-minimos.pdf"))
% (defun d () (interactive) (find-pdftools-page "~/LATEX/2022-2-C3-maximos-e-minimos.pdf"))
% (defun e () (interactive) (find-LATEX "2022-2-C3-maximos-e-minimos.tex"))
% (defun o () (interactive) (find-LATEX "2021-2-C3-diag-nums.tex"))
% (defun u () (interactive) (find-latex-upload-links "2022-2-C3-maximos-e-minimos"))
% (defun v () (interactive) (find-2a '(e) '(d)))
% (defun d0 () (interactive) (find-ebuffer "2022-2-C3-maximos-e-minimos.pdf"))
% (defun cv () (interactive) (C) (ee-kill-this-buffer) (v) (g))
%          (code-eec-LATEX "2022-2-C3-maximos-e-minimos")
% (find-pdf-page   "~/LATEX/2022-2-C3-maximos-e-minimos.pdf")
% (find-sh0 "cp -v  ~/LATEX/2022-2-C3-maximos-e-minimos.pdf /tmp/")
% (find-sh0 "cp -v  ~/LATEX/2022-2-C3-maximos-e-minimos.pdf /tmp/pen/")
%     (find-xournalpp "/tmp/2022-2-C3-maximos-e-minimos.pdf")
%   file:///home/edrx/LATEX/2022-2-C3-maximos-e-minimos.pdf
%               file:///tmp/2022-2-C3-maximos-e-minimos.pdf
%           file:///tmp/pen/2022-2-C3-maximos-e-minimos.pdf
% http://angg.twu.net/LATEX/2022-2-C3-maximos-e-minimos.pdf
% (find-LATEX "2019.mk")
% (find-sh0 "cd ~/LUA/; cp -v Pict2e1.lua Pict2e1-1.lua Piecewise1.lua ~/LATEX/")
% (find-sh0 "cd ~/LUA/; cp -v Pict2e1.lua Pict2e1-1.lua Pict3D1.lua ~/LATEX/")
% (find-sh0 "cd ~/LUA/; cp -v C2Subst1.lua C2Formulas1.lua ~/LATEX/")
% (find-CN-aula-links "2022-2-C3-maximos-e-minimos" "3" "c3m222mms" "c3mm")

% «.defs»			(to "defs")
% «.title»			(to "title")
% «.links»			(to "links")
% «.introducao»			(to "introducao")
% «.sinais-por-numerozinhos»	(to "sinais-por-numerozinhos")
%   «.video-1»			(to "video-1")
%   «.video-2»			(to "video-2")
% «.derivadas-direcionais»	(to "derivadas-direcionais")
% «.versao-mega-rapida»		(to "versao-mega-rapida")
%
% «.djvuize»			(to "djvuize")



% <videos>
% Video (not yet):
% (find-ssr-links     "c3m222mms" "2022-2-C3-maximos-e-minimos")
% (code-eevvideo      "c3m222mms" "2022-2-C3-maximos-e-minimos")
% (code-eevlinksvideo "c3m222mms" "2022-2-C3-maximos-e-minimos")
% (find-c3m222mmsvideo "0:00")

\documentclass[oneside,12pt]{article}
\usepackage[colorlinks,citecolor=DarkRed,urlcolor=DarkRed]{hyperref} % (find-es "tex" "hyperref")
\usepackage{amsmath}
\usepackage{amsfonts}
\usepackage{amssymb}
\usepackage{pict2e}
\usepackage[x11names,svgnames]{xcolor} % (find-es "tex" "xcolor")
\usepackage{colorweb}                  % (find-es "tex" "colorweb")
%\usepackage{tikz}
%
% (find-dn6 "preamble6.lua" "preamble0")
%\usepackage{proof}   % For derivation trees ("%:" lines)
%\input diagxy        % For 2D diagrams ("%D" lines)
%\xyoption{curve}     % For the ".curve=" feature in 2D diagrams
%
\usepackage{edrx21}               % (find-LATEX "edrx21.sty")
\input edrxaccents.tex            % (find-LATEX "edrxaccents.tex")
\input edrx21chars.tex            % (find-LATEX "edrx21chars.tex")
\input edrxheadfoot.tex           % (find-LATEX "edrxheadfoot.tex")
\input edrxgac2.tex               % (find-LATEX "edrxgac2.tex")
%\usepackage{emaxima}              % (find-LATEX "emaxima.sty")
%
%\usepackage[backend=biber,
%   style=alphabetic]{biblatex}            % (find-es "tex" "biber")
%\addbibresource{catsem-slides.bib}        % (find-LATEX "catsem-slides.bib")
%
% (find-es "tex" "geometry")
\usepackage[a6paper, landscape,
            top=1.5cm, bottom=.25cm, left=1cm, right=1cm, includefoot
           ]{geometry}
%
\begin{document}

\catcode`\^^J=10
\directlua{dofile "dednat6load.lua"}  % (find-LATEX "dednat6load.lua")
%L dofile "Piecewise1.lua"           -- (find-LATEX "Piecewise1.lua")
%L dofile "QVis1.lua"                -- (find-LATEX "QVis1.lua")
%L dofile "Pict3D1.lua"              -- (find-LATEX "Pict3D1.lua")
%L dofile "C2Formulas1.lua"          -- (find-LATEX "C2Formulas1.lua")
%L Pict2e.__index.suffix = "%"
\pu
\def\pictgridstyle{\color{GrayPale}\linethickness{0.3pt}}
\def\pictaxesstyle{\linethickness{0.5pt}}
\def\pictnaxesstyle{\color{GrayPale}\linethickness{0.5pt}}
\celllower=2.5pt

% «defs»  (to ".defs")
% (find-LATEX "edrx21defs.tex" "colors")
% (find-LATEX "edrx21.sty")

\def\u#1{\par{\footnotesize \url{#1}}}

\def\drafturl{http://angg.twu.net/LATEX/2022-2-C3.pdf}
\def\drafturl{http://angg.twu.net/2022.2-C3.html}
\def\draftfooter{\tiny \href{\drafturl}{\jobname{}} \ColorBrown{\shorttoday{} \hours}}

\def\ddt{\frac{d}{dt}}


%  _____ _ _   _                               
% |_   _(_) |_| | ___   _ __   __ _  __ _  ___ 
%   | | | | __| |/ _ \ | '_ \ / _` |/ _` |/ _ \
%   | | | | |_| |  __/ | |_) | (_| | (_| |  __/
%   |_| |_|\__|_|\___| | .__/ \__,_|\__, |\___|
%                      |_|          |___/      
%
% «title»  (to ".title")
% (c3m222mmsp 1 "title")
% (c3m222mmsa   "title")

\thispagestyle{empty}

\begin{center}

\vspace*{1.2cm}

{\bf \Large Cálculo 3 - 2022.2}

\bsk

Aula 27: Máximos e mínimos

\bsk

Eduardo Ochs - RCN/PURO/UFF

\url{http://angg.twu.net/2022.2-C3.html}

\end{center}

\newpage

% «links»  (to ".links")
% (c3m222mmsp 2 "links")
% (c3m222mmsa   "links")
% (c3m222topp 2 "links")
% (c3m222topa   "links")
% (find-books "__analysis/__analysis.el" "bortolossi")
% (find-books "__analysis/__analysis.el" "bortolossi" "10. Máximos e mínimos")
% (find-books "__analysis/__analysis.el" "bortolossi" "11. Otimização sem")
% (find-books "__analysis/__analysis.el" "bortolossi" "12. Otimização com")

\newpage

% «introducao»  (to ".introducao")
% (c3m222mmsp 2 "introducao")
% (c3m222mmsa   "introducao")

{\bf Introdução}

\scalebox{0.6}{\def\colwidth{15cm}\firstcol{

Quando a gente aprende polinômios a gente aprende uma matéria chamada
``estudo do sinal de uma função'', em que a gente aprende a marcar em
$\R$ em que regiões as funções que nos interessam são positivas,
negativas, ou 0... algo como isso aqui, mas desenhado de outro jeito:
%
$$\begin{array}{rcccccc}
                     & (-∞,2) & 2   & (2,5) & 5   & (5,+∞) \\
         g(x) = x-2  &   <0   & =0  &  >0   & >0  & >0     \\
         h(x) = x-5  &   <0   & <0  &  <0   & =0  & >0     \\
  f(x) = (x-2)(x-5)  &   >0   & =0  &  <0   & =0  & >0     \\
 \end{array}
$$

Depois a gente aprende derivada e segunda derivada, e aí a gente
estende essa idéia pra representar também as regiões em que derivada é
positiva, negativa, ou 0, e as regiões em que a segunda derivada é
positiva, negativa, ou 0, e a gente usa isso pra descobrir onde a
função é crescente ou decrescente, onde a concavidade dela está pra
baixo ou pra cima, e onde ela tem máximos e mínimos locais e máximos e
mínimos globais. Dê uma olhada nas figuras das seções 5.1 até 5.4 do
Miranda:

% (find-books "__analysis/__analysis.el" "miranda")
% (find-dmirandacalcpage 122 "5.1 Valores Extremos")
% (find-dmirandacalcpage 124 "5.1.2 Extremos Relativos")
% (find-dmirandacalcpage 126 "5.1.3 Extremos em Intervalos Fechados")
% (find-dmirandacalcpage 138 "5.4 Concavidade")

\ssk

{\scriptsize

%    http://hostel.ufabc.edu.br/~daniel.miranda/calculo/calculo.pdf#page=122
\url{http://hostel.ufabc.edu.br/~daniel.miranda/calculo/calculo.pdf\#page=122}

}

\msk

Repare que o Miranda chama esse assunto de ``extremos relativos'' e
``extremos absolutos''; o Bortolossi vai estender essas idéias pra
mais dimensões nos capítulos 10, 11 e 12 dele, e ele vai usar outra
terminologia: ``máximos locais'' e ``máximos globais''.

}\anothercol{
}}

\newpage

% «sinais-por-numerozinhos»  (to ".sinais-por-numerozinhos")
% (c3m222mmsp 3 "sinais-por-numerozinhos")
% (c3m222mmsa   "sinais-por-numerozinhos")

{\bf Estudo de sinal em $\R^2$ por numerozinhos}

Se você achar a abordagem de hoje muito complicada

{\sl comece} pela de 2021... assista estes dois vídeos,


\ssk

{\scriptsize

% «video-1»  (to ".video-1")
% (c3m211qa "video-1")
% (c3m211qa "video-1" "4:25" "x0 e y0")
\url{http://angg.twu.net/eev-videos/2021-1-C3-funcoes-quadraticas.mp4}

\url{https://www.youtube.com/watch?v=2noSv8hyNIk}

\msk

% «video-2»  (to ".video-2")
% (c3m211qa "video-2")
% (c3m211qa "video-2" "3:29" "façam o diagrama de sinais")
% (c3m211qa "video-2" "6:09" "combinar diagramas")
\url{http://angg.twu.net/eev-videos/2021-1-C3-funcoes-quadraticas-2.mp4}

\url{https://www.youtube.com/watch?v=noVh-RsK5Jo}

}

\msk

e faça os exercícios das páginas 8 até 13 daqui:

\ssk

{\scriptsize

% (c3m212dnp 8 "eq-da-superficie")
% (c3m212dna   "eq-da-superficie")
%    http://angg.twu.net/LATEX/2021-2-C3-diag-nums.pdf#page=8
\url{http://angg.twu.net/LATEX/2021-2-C3-diag-nums.pdf#page=8}

}


\newpage

% «derivadas-direcionais»  (to ".derivadas-direcionais")
% (c3m222mmsp 4 "derivadas-direcionais")
% (c3m222mmsa    "derivadas-direcionais")

{\bf Um truque com derivadas direcionais}

\scalebox{0.57}{\def\colwidth{9.75cm}\firstcol{

Vou começar supondo que
%
$$\begin{array}{rcl}
  z &=& a \\
    &+& bx + cy \\
    &+& dx^2 + exy + fy^2 \\
  \end{array}
$$

e que o ponto que nos interessa é $(x_0,y_0)=(0,0)$. Depois que nós
tivermos entendido bem as contas no ponto $(0,0)$ a gente vai ver como
refazê-las numa versão um pouco mais geral, em que $(x_0,y_0)$ é um
ponto qualquer.

\msk

Queremos generalizar as definições de mínimo local e máximo local do
Miranda, que estão aqui,

\ssk

{\scriptsize

% (find-books "__analysis/__analysis.el" "miranda")
% (find-dmirandacalcpage 122 "5.1 Valores Extremos")
% (find-dmirandacalcpage 124 "5.1.2 Extremos Relativos")
%    http://hostel.ufabc.edu.br/~daniel.miranda/calculo/calculo.pdf#page=124
\url{http://hostel.ufabc.edu.br/~daniel.miranda/calculo/calculo.pdf\#page=124}

}

\ssk

pra $\R^2$. Vou usar um truque com derivadas direcionais.

\msk

Vou dizer que o ponto $(0,0)$ é um mínimo local ``na direção
$\VEC{α,β}$'' se a função $z(t)=z(x(t),y(t))=z(αt,βt)$ tem um mínimo
local em $t=0$, e vou dizer que a função $z$ tem um mínimo local em
$(0,0)$ se o ponto $(0,0)$ é um mínimo local ``em todas as direções''
--- exceto pela ``direção'' $\VEC{α,β}=\VEC{0,0}$, que a gente
considera que ``não é uma direção válida'', e ``não interessa''.

\msk

}\anothercol{

Se isto aqui for verdade,
%
$$\begin{array}{l}
  ∀\VEC{α,β}≠\VEC{0,0}. \\
  \ddt(z(αt,βt)) = 0 \text{ e } \\
  \ddt\ddt(z(αt,βt)) > 0 \\
  \end{array}
$$

então o ponto $(0,0)$ vai ser um mínimo local da função $z(x,y)$.
Repare que lá no início eu defini que $z$ era um polinômio de grau 2
em $x$ e $y$; 

}}

\newpage

% «versao-mega-rapida»  (to ".versao-mega-rapida")
% (c3m222mmsp 5 "versao-mega-rapida")
% (c3m222mmsa    "versao-mega-rapida")

{\bf Versão mega-rápida das páginas 365--394 do Bortolossi}


\scalebox{0.72}{\def\colwidth{15cm}\firstcol{

Links:

{\footnotesize

% (c3m222fhp 1 "title")
% (c3m222fha   "title")
%    http://angg.twu.net/LATEX/2022-2-C3-funcoes-homogeneas.pdf
\url{http://angg.twu.net/LATEX/2022-2-C3-funcoes-homogeneas.pdf}

% (find-angg ".emacs" "c3q222")
% (find-angg ".emacs" "c3q222" "Funções homogêneas")
%    http://angg.twu.net/2022.2-C3/C3-quadros.pdf#page=17
\url{http://angg.twu.net/2022.2-C3/C3-quadros.pdf\#page=17}

% (find-books "__analysis/__analysis.el" "bortolossi")
% http://angg.twu.net/2019.2-C3/Bortolossi/bortolossi-cap-10.pdf
\url{http://angg.twu.net/2019.2-C3/Bortolossi/bortolossi-cap-10.pdf}

}

\msk


Digamos que $r_1,r_2,r,α,β∈\R$, $r_1≠r_2$,

$α,β>0$, e $r_3=x+βi$, $r_4=x-βi$, e:
%
$$\begin{array}{rcl}
  z(x,y) &=& dx^2 + exy + fy^2, \\
  h(x) &=& z(x,1). \\
  \end{array}
$$

Então:

\msk

\begin{tabular}{lll}
se & $h(x) =   (x-r_1)(x-r_2)$ & então $(0,0)$ é um ponto de sela, \\
se & $h(x) =  α(x-r_1)(x-r_2)$ & então $(0,0)$ é um ponto de sela, \\
se & $h(x) = -α(x-r_1)(x-r_2)$ & então $(0,0)$ é um ponto de sela, \\
se & $h(x) =   (x-r)^2$ & então $(0,0)$ é como a figura da p.388, \\
se & $h(x) =  α(x-r)^2$ & então $(0,0)$ é como a figura da p.388, \\
se & $h(x) = -α(x-r)^2$ & então $(0,0)$ é como a figura da p.388, \\
se & $h(x) =   (x-r_3)(x-r_4)$ & então $(0,0)$ ``tem concavidade pra cima'', \\
se & $h(x) =  α(x-r_3)(x-r_4)$ & então $(0,0)$ ``tem concavidade pra cima'', \\
se & $h(x) = -α(x-r_3)(x-r_4)$ & então $(0,0)$ ``tem concavidade pra baixo''. \\
\end{tabular}

}\anothercol{
}}

\newpage

{\bf Exercício}


\scalebox{0.9}{\def\colwidth{12cm}\firstcol{

Digamos que 
%
$$\begin{array}{rcl}
  z(x,y) &=& dx^2 + exy + fy^2, \\
  h(x) &=& z(x,1). \\
  \end{array}
$$

Para cada uma das funções $h(x)$ abaixo diga qual é a função $z(x,y)$

associada a ela e faça o diagrama de sinais dessa função $z(x,y)$.

\msk

a) $h(x) = (x-1)(x+2)$

b) $h(x) = 2(x-1)(x+2)$

c) $h(x) = -3(x-1)(x+2)$

d) $h(x) = (x-1)^2$

e) $h(x) = 2(x-1)^2$

f) $h(x) = -3(x-1)^2$

g) $h(x) = (x-(2+i))(x-(2-i))$

h) $h(x) = 2(x-(2+i))(x-(2-i))$

i) $h(x) = -3(x-(2+i))(x-(2-i))$

}\anothercol{
}}


% (find-bortolossi11page (+ -364 406) "estudo do sinal de f'")

\GenericWarning{Success:}{Success!!!}  % Used by `M-x cv'

\end{document}


% Local Variables:
% coding: utf-8-unix
% ee-tla: "c3mm"
% ee-tla: "c3m222mms"
% End:

