% (find-LATEX "2022-2-C3-intro.tex")
% (defun c  () (interactive) (find-LATEXsh "lualatex -record 2022-2-C3-intro.tex" :end))
% (defun C  () (interactive) (find-LATEXsh "lualatex 2022-2-C3-intro.tex" "Success!!!"))
% (defun D  () (interactive) (find-pdf-page      "~/LATEX/2022-2-C3-intro.pdf"))
% (defun d  () (interactive) (find-pdftools-page "~/LATEX/2022-2-C3-intro.pdf"))
% (defun e  () (interactive) (find-LATEX "2022-2-C3-intro.tex"))
% (defun o  () (interactive) (find-LATEX "2022-1-C3-intro.tex"))
% (defun oo () (interactive) (find-LATEX "2021-2-C3-intro.tex"))
% (defun u  () (interactive) (find-latex-upload-links "2022-2-C3-intro"))
% (defun v  () (interactive) (find-2a '(e) '(d)))
% (defun d0 () (interactive) (find-ebuffer "2022-2-C3-intro.pdf"))
% (defun cv () (interactive) (C) (ee-kill-this-buffer) (v) (g))
%          (code-eec-LATEX "2022-2-C3-intro")
% (find-pdf-page   "~/LATEX/2022-2-C3-intro.pdf")
% (find-sh0 "cp -v  ~/LATEX/2022-2-C3-intro.pdf /tmp/")
% (find-sh0 "cp -v  ~/LATEX/2022-2-C3-intro.pdf /tmp/pen/")
%     (find-xournalpp "/tmp/2022-2-C3-intro.pdf")
%   file:///home/edrx/LATEX/2022-2-C3-intro.pdf
%               file:///tmp/2022-2-C3-intro.pdf
%           file:///tmp/pen/2022-2-C3-intro.pdf
% http://angg.twu.net/LATEX/2022-2-C3-intro.pdf
% (find-LATEX "2019.mk")
% (find-sh0 "cd ~/LUA/; cp -v Pict2e1.lua Pict2e1-1.lua Piecewise1.lua ~/LATEX/")
% (find-sh0 "cd ~/LUA/; cp -v Pict2e1.lua Pict2e1-1.lua Pict3D1.lua ~/LATEX/")
% (find-sh0 "cd ~/LUA/; cp -v C2Subst1.lua C2Formulas1.lua ~/LATEX/")
% (find-CN-aula-links "2022-2-C3-intro" "3" "c3m222intro" "c3i")

% «.defs»	(to "defs")
% «.title»	(to "title")
% «.aula-1»	(to "aula-1")
% «.intro-v»	(to "intro-v")
% «.VT»		(to "VT")
% «.orbita»	(to "orbita")
% «.orbita-2»	(to "orbita-2")
% «.orbita-3»	(to "orbita-3")
% «.orbita-4»	(to "orbita-4")
%
% «.djvuize»	(to "djvuize")



% <videos>
% Video (not yet):
% (find-ssr-links     "c3m222intro" "2022-2-C3-intro")
% (code-eevvideo      "c3m222intro" "2022-2-C3-intro")
% (code-eevlinksvideo "c3m222intro" "2022-2-C3-intro")
% (find-c3m222introvideo "0:00")

\documentclass[oneside,12pt]{article}
\usepackage[colorlinks,citecolor=DarkRed,urlcolor=DarkRed]{hyperref} % (find-es "tex" "hyperref")
\usepackage{amsmath}
\usepackage{amsfonts}
\usepackage{amssymb}
\usepackage{pict2e}
\usepackage[x11names,svgnames]{xcolor} % (find-es "tex" "xcolor")
\usepackage{colorweb}                  % (find-es "tex" "colorweb")
%\usepackage{tikz}
%
% (find-dn6 "preamble6.lua" "preamble0")
%\usepackage{proof}   % For derivation trees ("%:" lines)
%\input diagxy        % For 2D diagrams ("%D" lines)
%\xyoption{curve}     % For the ".curve=" feature in 2D diagrams
%
\usepackage{edrx21}               % (find-LATEX "edrx21.sty")
\input edrxaccents.tex            % (find-LATEX "edrxaccents.tex")
\input edrx21chars.tex            % (find-LATEX "edrx21chars.tex")
\input edrxheadfoot.tex           % (find-LATEX "edrxheadfoot.tex")
\input edrxgac2.tex               % (find-LATEX "edrxgac2.tex")
%\usepackage{emaxima}              % (find-LATEX "emaxima.sty")
%
%\usepackage[backend=biber,
%   style=alphabetic]{biblatex}            % (find-es "tex" "biber")
%\addbibresource{catsem-slides.bib}        % (find-LATEX "catsem-slides.bib")
%
% (find-es "tex" "geometry")
\usepackage[a6paper, landscape,
            top=1.5cm, bottom=.25cm, left=1cm, right=1cm, includefoot
           ]{geometry}
%
\begin{document}

\catcode`\^^J=10
\directlua{dofile "dednat6load.lua"}  % (find-LATEX "dednat6load.lua")
%L dofile "Piecewise1.lua"           -- (find-LATEX "Piecewise1.lua")
%L dofile "QVis1.lua"                -- (find-LATEX "QVis1.lua")
%L dofile "Pict3D1.lua"              -- (find-LATEX "Pict3D1.lua")
%L dofile "C2Formulas1.lua"          -- (find-LATEX "C2Formulas1.lua")
%L Pict2e.__index.suffix = "%"
\pu
\def\pictgridstyle{\color{GrayPale}\linethickness{0.3pt}}
\def\pictaxesstyle{\linethickness{0.5pt}}
\def\pictnaxesstyle{\color{GrayPale}\linethickness{0.5pt}}
\celllower=2.5pt

% «defs»  (to ".defs")
% (find-LATEX "edrx21defs.tex" "colors")
% (find-LATEX "edrx21.sty")

\def\u#1{\par{\footnotesize \url{#1}}}

\def\drafturl{http://angg.twu.net/LATEX/2022-2-C3.pdf}
\def\drafturl{http://angg.twu.net/2022.2-C3.html}
\def\draftfooter{\tiny \href{\drafturl}{\jobname{}} \ColorBrown{\shorttoday{} \hours}}



%  _____ _ _   _                               
% |_   _(_) |_| | ___   _ __   __ _  __ _  ___ 
%   | | | | __| |/ _ \ | '_ \ / _` |/ _` |/ _ \
%   | | | | |_| |  __/ | |_) | (_| | (_| |  __/
%   |_| |_|\__|_|\___| | .__/ \__,_|\__, |\___|
%                      |_|          |___/      
%
% «title»  (to ".title")
% (c3m222introp 1 "title")
% (c3m222introa   "title")

\thispagestyle{empty}

\begin{center}

\vspace*{1.2cm}

{\bf \Large Cálculo 3 - 2022.2}

\bsk

Aulas 1 e 2: introdução ao curso

(e a trajetórias)

\bsk

Eduardo Ochs - RCN/PURO/UFF

\url{http://angg.twu.net/2022.2-C3.html}

\end{center}

\newpage

%     _         _         _ 
%    / \  _   _| | __ _  / |
%   / _ \| | | | |/ _` | | |
%  / ___ \ |_| | | (_| | | |
% /_/   \_\__,_|_|\__,_| |_|
%                           
% «aula-1»  (to ".aula-1")
% (c3m222introp 2 "aula-1")
% (c3m222introa   "aula-1")

{\bf Sobre a aula 1}

Na aula 1 nós usamos as idéias dos 8 primeiros slides daqui,

\ssk

{\footnotesize

% (c3m212introp 1 "title")
% (c3m212introa   "title")
%    http://angg.twu.net/LATEX/2021-2-C3-intro.pdf
\url{http://angg.twu.net/LATEX/2021-2-C3-intro.pdf}

}

\ssk

e do slide 10 daqui,

\ssk

{\footnotesize

% (c3m202planotangp 10 "geral-e-particular")
% (c3m202planotanga    "geral-e-particular")
%    http://angg.twu.net/LATEX/2020-2-C3-plano-tang.pdf#page=10
\url{http://angg.twu.net/LATEX/2020-2-C3-plano-tang.pdf#page=10}

}

\ssk

...pra desenhar casos particulares das figuras das seções 7.4 e 7.5

do ``GA1'' do Felipe Acker:

\ssk

{\footnotesize

% (find-books "__analysis/__analysis.el" "acker")
% (find-ackerGA1page (+ 16 27) "7.4 Soma de vetores")
% (find-ackerGA1page (+ 16 29) "7.5 Somando vetores a pontos")
%    http://angg.twu.net/acker/acker__ga_livro1_2019.pdf#page=43
\url{http://angg.twu.net/acker/acker__ga_livro1_2019.pdf\#page=43}

}


\newpage

% «intro-v»  (to ".intro-v")
% (c3m222introp 3 "intro-v")
% (c3m222introa   "intro-v")

{\bf Introdução ao vetor velocidade}

Em cursos de Cálculo 3 ``pra matemáticos'' a gente normalmente

começa definindo o vetor velocidade como um limite. O Felipe

Acker faz isso muito bem nos capítulos 2 e 3 do ``GA4'',

\ssk

{\footnotesize

% (find-books "__analysis/__analysis.el" "acker")
% (find-ackerGA4page (+ 8 13) "2" "Velocidade")
% (find-ackerGA4text (+ 8 13) "2" "Velocidade")
% (find-ackerGA4page (+ 8 19) "3" "Aceleracao")
% (find-ackerGA4text (+ 8 19) "3" "Aceleracao")
%    http://angg.twu.net/acker/acker__ga_livro4_2019.pdf
\url{http://angg.twu.net/acker/acker__ga_livro4_2019.pdf}

}

\ssk

Eu costumava fazer mais ou menos isso no curso de Cálculo 3,

e a gente gastava uma aula inteira aprendendo a decifrar a

fórmula daquele limite e visualizar o que ela queria dizer.

\msk

Dessa vez vamos tentar fazer algo diferente.

Vamos começar com exemplos e animações.

Assista este vídeo aqui até o 9:00,

\ssk

{\footnotesize

% (c3m212bezierp 1)
% (c3m212bezier 1)
%    http://angg.twu.net/LATEX/2021-2-C3-bezier.pdf
\url{http://angg.twu.net/LATEX/2021-2-C3-bezier.pdf}

%    https://www.youtube.com/watch?v=aVwxzDHniEw
\url{https://www.youtube.com/watch?v=aVwxzDHniEw}

}

\ssk

\standout{mas considere que tudo até o 6:34...}

\newpage

{\bf Introdução ao vetor velocidade (cont.)}

\ssk

...mas considere que tudo no vídeo até o 6:34 são idéias avançadas que
a gente só vai entender nuns exercícios que a gente vai fazer daqui a
algumas aulas. Por enquanto reserve praticamente toda a sua atenção
pro trecho entre 6:34 e 9:00, que é o trecho que a Freya Holmér mostra
os vetores velocidade e aceleração pra algumas curvas de Bézier.

A gente vai fazer o seguinte. Nós vamos acreditar que {\sl em geral}
quando temos uma trajetória $P(t) = (x(t),y(t))$ o vetor velocidade
dessa trajetória é $P'(t) = (x'(t),y'(t))$. Nós vamos ver vários
exemplos disso, e vamos deixar pra entender os detalhes desse ``em
geral'' quando formos entender a definição ``pra matemáticos'' do
vetor velocidade.

\newpage

{\bf Exercício 1: uma trajetória com um bico}

Dê uma olhada no item 1e da VS do semestre passado:

\ssk

{\footnotesize

% (c3m221vsp 2 "questao-1")
% (c3m221vsa   "questao-1")
%    http://angg.twu.net/LATEX/2022-1-C3-VS.pdf#page=2
\url{http://angg.twu.net/LATEX/2022-1-C3-VS.pdf#page=2}

}

\ssk

Faça o que essa questão pede e represente graficamente $Q(t)+Q'(t)$
pra um monte de outros valores de $t$ também --- até você entender
como essa trajetória se comporta. {\sl Dica:} ela é um movimento
retilíneo uniforme até um determinado instante, aí ela muda de vetor
velocidade subitamente e vira um outro movimento retilíneo uniforme.

\msk

{\bf Exercício 2: um trajetória com teleporte}

Represente graficamente a trajetória abaixo. Ela é parecida com a
anterior, mas nessa tem um momento em que a partícula desaparece do
ponto em que em estava e se teleporta pra outro lugar.
%
$$\scalebox{0.9}{$
  R(t) \;=\;
  \begin{cases}
    (t,4)    & \text{quando $t≤6$}, \\
    (5,11-t) & \text{quando $6<t$}. \\
  \end{cases}
  $}
$$


\newpage

% «VT»  (to ".VT")
% (c3m222introp 6 "VT")
% (c3m222introa   "VT")

{\bf ``VT''}

\msk

Vou me referir a esse PDF aqui como ``VT'',

\ssk

{\footnotesize

% (c3m211vtp 3 "exercicio-2")
% (c3m211vta   "exercicio-2")
%    http://angg.twu.net/LATEX/2021-1-C3-vetor-tangente.pdf
\url{http://angg.twu.net/LATEX/2021-1-C3-vetor-tangente.pdf}

}

\ssk

e aos exercícios 1 e 2 dele como ``VTex1'', ``VTex2''.

\bsk

Faça os exercícios VTex1 e VTex2.


\newpage

% «orbita»  (to ".orbita")
% (c3m222introp 7 "orbita")
% (c3m222introa   "orbita")

{\bf Órbita}

Este exercício vai dar uma figura que é a órbita de uma lua.

O resultado vai ser algo como a figura da última página daqui,

\ssk

{\footnotesize

% (c3m221orbitap 2 "links")
% (c3m221orbitaa   "links")
%    http://angg.twu.net/LATEX/2022-1-C3-orbita.pdf
\url{http://angg.twu.net/LATEX/2022-1-C3-orbita.pdf}

}

\ssk

mas olhe pra essa figura durante só uns poucos segundos.

\msk

Neste exercício você vai tentar redescobrir essa figura sozinho, e
você vai tentar descobrir como desenhar uma aproximação bem razoável
pra ela só somando uns vetores no olhômetro e sem fazer nenhuma conta
complicada --- por exemplo, você vai evitar usar uma aproximação
numérica pra $(\cos(\frac{1}{12}·2π), \sen(\frac{1}{12}·2π))$; ao
invés disso você vai usar a representação gráfica deste ponto no
$\R^2$.

\newpage

% «orbita-2»  (to ".orbita-2")
% (c3m222introp 8 "orbita-2")
% (c3m222introa   "orbita-2")
% (find-es "maxima" "plot2d-parametric")
% (find-books "__analysis/__analysis.el" "leithold")
% (find-books "__analysis/__analysis.el" "leithold" "Limaçon")

{\bf Órbita (cont.)}

\scalebox{0.9}{\def\colwidth{14cm}\firstcol{

Seja $h = \frac{1}{12}·2π$.

Esse $h$ vai ser uma ``hora''. Vou explicar isso no quadro.

Sejam:
%
$$\begin{array}{rcl}
  P(t) &=& (\cos t, \sen t), \\
  Q(t) &=& (\cos 4t, \sen 4t), \\
  R(t) &=& \frac{1}{2}(\cos 4t, \sen 4t)  = (\frac{1}{2}\cos 4t, \frac{1}{2}\sen 4t), \\
  S(t) &=& P(t) + R(t). \\
  \end{array}
$$

a) Represente graficamente $P(t)$ para $t=0h, 1h, 2h, \ldots, 12h$.

b) Represente graficamente $P(t) + P'(t)$ para $t=0h, 1h, 2h, \ldots, 12h$.

\ssk

c) Represente graficamente $Q(t)$ para $t=0h, 1h, 2h, \ldots, 12h$.

d) Represente graficamente $Q(t) + Q'(t)$ para $t=0h, 1h, 2h, \ldots, 12h$.

\ssk

e) Represente graficamente $Q(t)$ para $t=0h, 1h, 2h, \ldots, 12h$.

f) Represente graficamente $Q(t) + Q'(t)$ para $t=0h, 1h, 2h, \ldots, 12h$.

\ssk

g) Represente graficamente $S(t)$ para $t=0h, 1h, 2h, \ldots, 12h$.

h) Represente graficamente $S(t) + S'(t)$ para $t=0h, 1h, 2h, \ldots, 12h$.

}\anothercol{
}}


\newpage

% «orbita-3»  (to ".orbita-3")
% (c3m222introp 9 "orbita-3")
% (c3m222introa   "orbita-3")

{\bf Órbita (cont.)}

Nos itens a até f você deve ter obtido pontos sobre círculos e vetores
tangentes aos círculos apoiados nestes pontos. Nos itens g e h você
deve ter obtido algo bem mais complicado: pontos e vetores apoiados
nestes pontos, mas você ainda não sabe direito sobre que curva eles
estão.

Reveja o trecho entre 6:34 e 9:00 do vídeo da Freya Holmér. A
trajetória que ela analisa é bem ``suave'', no sentido de que ela não
bicos ou teleportes, e a derivada da aceleração dela é constante.

No item h você obteve alguns pontos e vetores velocidade {\sl de uma
  trajetória que você não sabe direito qual é}... você só tem uma
lembrança vaga do ``traço'' dessa trajetória, porque você viu a
figura-spoiler durante uns poucos segundos.

\newpage

% «orbita-4»  (to ".orbita-4")
% (c3m222introp 10 "orbita-4")
% (c3m222introa    "orbita-4")

{\bf Órbita (cont.)}

\msk

i) Desenhe uma trajetória bem suave que nos instantes $t=0h$, $1h$,
$\ldots$, $12h$ passe pelos pontos que você obteve no item g. Aqui
você vai conseguir uma aproximação bem tosca pro ``traço'' da
trajetória $S(t)$.

\msk

j) Desenhe uma trajetória bem suave que nos instantes $t=0h$, $1h$,
$\ldots$, $12h$ passe pelos pontos que você obteve no item h, e que
naqueles instantes tenha exatamente os vetores velocidade que você
também desenhou no item h. Aqui você provavelmente vai conseguir uma
aproximação bastante boa pro ``traço'' da trajetória $S(t)$.

\msk

k) Refaça o desenho do item j pra ele ficar mais caprichado e
simétrico e tal. Quando você achar que conseguiu fazer uma versão
caprichada boa olhe de novo a figura-spoiler e compare o seu desenho
com ela.






% (c3m221orbitap 2 "links")
% (c3m221orbitaa   "links")
% (find-LATEXgrep "grep --color=auto -nH --null -e trajet *.tex")








% (c3m212bezierp 1 "title")
% (c3m212beziera   "title")


% (c3m212beziera "title")
% (c3m212beziera "title" "Aula 7: um vídeo sobre curvas de Bézier")

% https://www.youtube.com/watch?v=aVwxzDHniEw


% (find-books "__analysis/__analysis.el" "acker")





\ssk



%\printbibliography

\GenericWarning{Success:}{Success!!!}  % Used by `M-x cv'

\end{document}

%  ____  _             _         
% |  _ \(_)_   ___   _(_)_______ 
% | | | | \ \ / / | | | |_  / _ \
% | |_| | |\ V /| |_| | |/ /  __/
% |____// | \_/  \__,_|_/___\___|
%     |__/                       
%
% «djvuize»  (to ".djvuize")
% (find-LATEXgrep "grep --color -nH --null -e djvuize 2020-1*.tex")




%  __  __       _        
% |  \/  | __ _| | _____ 
% | |\/| |/ _` | |/ / _ \
% | |  | | (_| |   <  __/
% |_|  |_|\__,_|_|\_\___|
%                        
% <make>

 (eepitch-shell)
 (eepitch-kill)
 (eepitch-shell)
# (find-LATEXfile "2019planar-has-1.mk")
make -f 2019.mk STEM=2022-2-C3-intro veryclean
make -f 2019.mk STEM=2022-2-C3-intro pdf

% Local Variables:
% coding: utf-8-unix
% ee-tla: "c3i"
% ee-tla: "c3m222intro"
% End:
