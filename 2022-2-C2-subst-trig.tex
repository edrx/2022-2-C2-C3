% (find-LATEX "2022-2-C2-subst-trig.tex")
% (defun c () (interactive) (find-LATEXsh "lualatex -record 2022-2-C2-subst-trig.tex" :end))
% (defun C () (interactive) (find-LATEXsh "lualatex 2022-2-C2-subst-trig.tex" "Success!!!"))
% (defun D () (interactive) (find-pdf-page      "~/LATEX/2022-2-C2-subst-trig.pdf"))
% (defun d () (interactive) (find-pdftools-page "~/LATEX/2022-2-C2-subst-trig.pdf"))
% (defun e () (interactive) (find-LATEX "2022-2-C2-subst-trig.tex"))
% (defun o () (interactive) (find-LATEX "2022-2-C2-subst-trig.tex"))
% (defun u () (interactive) (find-latex-upload-links "2022-2-C2-subst-trig"))
% (defun v () (interactive) (find-2a '(e) '(d)))
% (defun d0 () (interactive) (find-ebuffer "2022-2-C2-subst-trig.pdf"))
% (defun cv () (interactive) (C) (ee-kill-this-buffer) (v) (g))
%          (code-eec-LATEX "2022-2-C2-subst-trig")
% (find-pdf-page   "~/LATEX/2022-2-C2-subst-trig.pdf")
% (find-sh0 "cp -v  ~/LATEX/2022-2-C2-subst-trig.pdf /tmp/")
% (find-sh0 "cp -v  ~/LATEX/2022-2-C2-subst-trig.pdf /tmp/pen/")
%     (find-xournalpp "/tmp/2022-2-C2-subst-trig.pdf")
%   file:///home/edrx/LATEX/2022-2-C2-subst-trig.pdf
%               file:///tmp/2022-2-C2-subst-trig.pdf
%           file:///tmp/pen/2022-2-C2-subst-trig.pdf
% http://angg.twu.net/LATEX/2022-2-C2-subst-trig.pdf
% (find-LATEX "2019.mk")
% (find-sh0 "cd ~/LUA/; cp -v Pict2e1.lua Pict2e1-1.lua Piecewise1.lua ~/LATEX/")
% (find-sh0 "cd ~/LUA/; cp -v Pict2e1.lua Pict2e1-1.lua Pict3D1.lua ~/LATEX/")
% (find-sh0 "cd ~/LUA/; cp -v C2Subst1.lua C2Formulas1.lua ~/LATEX/")
% (find-CN-aula-links "2022-2-C2-subst-trig" "2" "c2m222strig" "c2st")

% «.defs»	(to "defs")
% «.title»	(to "title")
% «.links»	(to "links")
%
% «.djvuize»	(to "djvuize")



% <videos>
% Video (not yet):
% (find-ssr-links     "c2m222strig" "2022-2-C2-subst-trig")
% (code-eevvideo      "c2m222strig" "2022-2-C2-subst-trig")
% (code-eevlinksvideo "c2m222strig" "2022-2-C2-subst-trig")
% (find-c2m222strigvideo "0:00")

\documentclass[oneside,12pt]{article}
\usepackage[colorlinks,citecolor=DarkRed,urlcolor=DarkRed]{hyperref} % (find-es "tex" "hyperref")
\usepackage{amsmath}
\usepackage{amsfonts}
\usepackage{amssymb}
\usepackage{pict2e}
\usepackage[x11names,svgnames]{xcolor} % (find-es "tex" "xcolor")
\usepackage{colorweb}                  % (find-es "tex" "colorweb")
%\usepackage{tikz}
%
% (find-dn6 "preamble6.lua" "preamble0")
%\usepackage{proof}   % For derivation trees ("%:" lines)
%\input diagxy        % For 2D diagrams ("%D" lines)
%\xyoption{curve}     % For the ".curve=" feature in 2D diagrams
%
\usepackage{edrx21}               % (find-LATEX "edrx21.sty")
\input edrxaccents.tex            % (find-LATEX "edrxaccents.tex")
\input edrx21chars.tex            % (find-LATEX "edrx21chars.tex")
\input edrxheadfoot.tex           % (find-LATEX "edrxheadfoot.tex")
\input edrxgac2.tex               % (find-LATEX "edrxgac2.tex")
%\usepackage{emaxima}              % (find-LATEX "emaxima.sty")
%
%\usepackage[backend=biber,
%   style=alphabetic]{biblatex}            % (find-es "tex" "biber")
%\addbibresource{catsem-slides.bib}        % (find-LATEX "catsem-slides.bib")
%
% (find-es "tex" "geometry")
\usepackage[a6paper, landscape,
            top=1.5cm, bottom=.25cm, left=1cm, right=1cm, includefoot
           ]{geometry}
%
\begin{document}

\catcode`\^^J=10
\directlua{dofile "dednat6load.lua"}  % (find-LATEX "dednat6load.lua")
%L dofile "Piecewise1.lua"           -- (find-LATEX "Piecewise1.lua")
%L dofile "QVis1.lua"                -- (find-LATEX "QVis1.lua")
%L dofile "Pict3D1.lua"              -- (find-LATEX "Pict3D1.lua")
%L dofile "C2Formulas1.lua"          -- (find-LATEX "C2Formulas1.lua")
%L Pict2e.__index.suffix = "%"
\pu
\def\pictgridstyle{\color{GrayPale}\linethickness{0.3pt}}
\def\pictaxesstyle{\linethickness{0.5pt}}
\def\pictnaxesstyle{\color{GrayPale}\linethickness{0.5pt}}
\celllower=2.5pt

% «defs»  (to ".defs")
% (find-LATEX "edrx21defs.tex" "colors")
% (find-LATEX "edrx21.sty")

\def\u#1{\par{\footnotesize \url{#1}}}

\def\drafturl{http://angg.twu.net/LATEX/2022-2-C2.pdf}
\def\drafturl{http://angg.twu.net/2022.2-C2.html}
\def\draftfooter{\tiny \href{\drafturl}{\jobname{}} \ColorBrown{\shorttoday{} \hours}}

\def\P#1{\left(#1\right#1}
\def\Rq{\ColorRed{?}}



%  _____ _ _   _                               
% |_   _(_) |_| | ___   _ __   __ _  __ _  ___ 
%   | | | | __| |/ _ \ | '_ \ / _` |/ _` |/ _ \
%   | | | | |_| |  __/ | |_) | (_| | (_| |  __/
%   |_| |_|\__|_|\___| | .__/ \__,_|\__, |\___|
%                      |_|          |___/      
%
% «title»  (to ".title")
% (c2m222strigp 1 "title")
% (c2m222striga   "title")

\thispagestyle{empty}

\begin{center}

\vspace*{1.2cm}

{\bf \Large Cálculo 2 - 2022.2}

\bsk

Aulas 11 e 12: substituição trigonométrica

(a derivada da função inversa)

\bsk

Eduardo Ochs - RCN/PURO/UFF

\url{http://angg.twu.net/2022.2-C2.html}

\end{center}

\newpage

% «links»  (to ".links")
% (c2m222strigp 2 "links")
% (c2m222striga   "links")

{\bf Links}


Vamos usar a seção 9.4 do Leithold

% (find-books "__analysis/__analysis.el" "leithold")
% (find-books "__analysis/__analysis.el" "leithold" "substituição trigonométrica")
% (find-leitholdptpage (+ 17 545) "9.4. Integração por substituição trigonométrica")

e a seção 8.4 do Miranda:

{\scriptsize

% (find-books "__analysis/__analysis.el" "miranda")
% (find-books "__analysis/__analysis.el" "miranda" "8.4 Substituição Trigonométrica")
% (find-dmirandacalcpage 263 "8.4 Substituição Trigonométrica")
% http://hostel.ufabc.edu.br/~daniel.miranda/calculo/calculo.pdf#263
\url{http://hostel.ufabc.edu.br/~daniel.miranda/calculo/calculo.pdf\#263}

}

\ssk

Alguns links pra PDFs antigos meus:

{\scriptsize

% (c2m221atisp 24 "dfi")
% (c2m221atisa    "dfi")
%    http://angg.twu.net/LATEX/2022-1-C2-algumas-t-ints.pdf#page=24
\url{http://angg.twu.net/LATEX/2022-1-C2-algumas-t-ints.pdf#page=24}

% (c2m202stp 1 "title")
% (c2m202st    "title")
% (c2m202stp 1)
%    http://angg.twu.net/LATEX/2020-2-C2-subst-trig.pdf#page=1
\url{http://angg.twu.net/LATEX/2020-2-C2-subst-trig.pdf#page=1}

% (c2m221p2p 2 "subst-trig")
% (c2m221p2a   "subst-trig")
%    http://angg.twu.net/LATEX/2022-1-C2-P2.pdf#page=2
\url{http://angg.twu.net/LATEX/2022-1-C2-P2.pdf#page=2}

% (c2m221vsap 4 "questao-2")
% (c2m221vsaa   "questao-2")
%    http://angg.twu.net/LATEX/2022-1-C2-VSA.pdf#page=4
\url{http://angg.twu.net/LATEX/2022-1-C2-VSA.pdf#page=4}

}

\ssk

Quadros da primeira aula sobre substituição

trigonométrica (ainda não digitei o conteúdo deles):

{\scriptsize

% (find-angg ".emacs" "c2q222")
% (find-angg ".emacs" "c2q222" "22" "set28: substituição trigonométrica")
% http://angg.twu.net/2022.2-C2/C2-quadros.pdf#page=22
\url{http://angg.twu.net/2022.2-C2/C2-quadros.pdf\#page=22}

}




% (find-angg ".emacs" "c3-2022-1-quadros")
% (find-angg ".emacs" "c3-2022-1-quadros" "diagramas de numerozinhos")



\newpage

{\bf Exercício 1}

\scalebox{0.6}{\def\colwidth{9cm}\firstcol{

{\bf Simplificando raizes quadradas}

Nas últimas aulas você aprendeu -- na prática, não vendo uma
definição formal -- o que é transformar uma integral mais difícil
numa integral mais fácil, que nós sabemos integrar...

\ssk

a) Digamos que você sabe integrar $\ints{\sqrt{1-s^2}}$.

Transforme $\intx{\sqrt{1-(5x)^2}}$ em algo que você sabe integrar.

\ssk

b) Transforme $\intx{\sqrt{1-(ax)^2}}$ em algo que você sabe integrar.

\ssk

c) Digamos que você sabe integrar $\ints{\sqrt{1-s^2}^{\,k}}$ para
qualquer valor de $k$.

Transforme $\intx{{\sqrt{1-(5x)^2}}^{\,42}}$ em algo que você sabe
integrar.

\ssk

d) Transforme $\intx{\sqrt{1-(ax)^2}^{\,42}}$ em algo que você sabe
integrar.

\ssk

e) Transforme $\intx{\sqrt{1-(ax)^2}^{\,k}}$ em algo que você sabe integrar.

\ssk

f) Transforme $\intx{\sqrt{1-(ax)^2}^{\,k}}$ em algo que você sabe integrar.

}\anothercol{

\ssk

g) Entenda este truque aqui:
%
$$\begin{array}{rcl}
  \sqrt{3^2 - x^2} &=& \sqrt{3^2 - 3^2 \frac{1}{3^2} x^2} \\
                   &=& \sqrt{3^2 - 3^2(\frac x3)^2} \\
                   &=& \sqrt{3^2(1 - (\frac x3)^2)} \\
                   &=& \sqrt{3^2}\sqrt{1 - (\frac x3)^2} \\
                   &=& 3\sqrt{1 - (\frac x3)^2} \\
  \end{array}
$$

Use ele -- com adaptações, óbvio -- pra transformar
$\intx{\sqrt{25-x^2}}$ em algo que você sabe integrar.

\ssk

h) Use ele pra transformar
$\intx{\sqrt{25-x^2}^{\,42}}$ em algo que você sabe integrar.

\ssk

i) Use ele pra transformar $\intx{\sqrt{a^2-x^2}}$ em algo que você
sabe integrar.

\ssk

j) Use ele pra transformar
$\intx{\sqrt{a^2-x^2}^{\,k}}$ em algo que você sabe integrar.

\ssk

j) Use ele pra transformar $\intx{x^{20} \sqrt{a^2-x^2}^{\,k}}$ em
algo que você sabe integrar.


}}



\newpage

{\bf Exercício 2}

\sa{[DFI]}{\CFname{DFI}{}}


\scalebox{0.6}{\def\colwidth{9cm}\firstcol{

No final da aula de 28/set -- veja a foto do quadro:

\ssk

{\scriptsize

% (find-angg ".emacs" "c2q222")
% (find-angg ".emacs" "c2q222" "22" "set28: substituição trigonométrica")
% http://angg.twu.net/2022.2-C2/C2-quadros.pdf#page=23
\url{http://angg.twu.net/2022.2-C2/C2-quadros.pdf\#page=23}

}

\ssk

nós vimos que a demonstração de que $\ddx \ln x = \frac1x$ pode ser
generalizada, e aí a gente obtém a ``fórmula da derivada da função
inversa'', que eu chamei de \ga{[DFI]}...

Essa generalização pode ser ``especializada'' pra obter outros casos
particulares diferentes de $\ddx \ln x = \frac1x$.

\msk

a) Faça o primeiro exercício que eu pus no quadro:
%
$$\ga{[DFI]}
  \bmat{
    g(x) := \arcsen x \\
    g'(x) := \arcsen' x \\
    f(x) := \sen x \\
    f'(x) := \cos x \\
  } = \Rq
$$

b) Faça o segundo exercício do quadro:
%
$$\ga{[DFI]}
  \bmat{
    g(x) := \arcsen x \\
    g'(x) := \arcsen' x \\
    f(x) := \sen x \\
    f'(x) := \sqrt{1 - (\sen x)^2} \\
  } = \Rq
$$

}\anothercol{



  c) Use as identidades trigonométricas que vamos ver em sala pra
  encontrar uma fórmula pra derivada do $\arctan$.

\msk

  d) Use as identidades trigonométricas que vamos ver em sala pra
  encontrar uma fórmula pra derivada do $\arcsec$.


}}



\newpage

{\bf Exercício 3}


\scalebox{0.54}{\def\colwidth{10cm}\firstcol{

Slogan:

\begin{quote}

  {\sl Toda integral que pode ser resolvida por uma sequência de mudanças
  de variável pode ser resolvida por uma mudança de variável só.}

\end{quote}

Durante a quarentena eu dei algumas questões de prova sobre este
slogan. Dê uma olhada:

\ssk

{\footnotesize

% (c2m202p1p 4 "questao-2")
% (c2m202p1a   "questao-2")
%    http://angg.twu.net/LATEX/2020-2-C2-P1.pdf#page=4
\url{http://angg.twu.net/LATEX/2020-2-C2-P1.pdf\#page=4}

% (c2m202p1p 9 "gabarito-2")
% (c2m202p1a   "gabarito-2")
%    http://angg.twu.net/LATEX/2020-2-C2-P1.pdf#page=9
\url{http://angg.twu.net/LATEX/2020-2-C2-P1.pdf\#page=9}

% (c2m211p1p 15 "gabarito-2-2020.2")
% (c2m211p1a    "gabarito-2-2020.2")
%    http://angg.twu.net/LATEX/2021-1-C2-P1.pdf#page=15
\url{http://angg.twu.net/LATEX/2021-1-C2-P1.pdf\#page=15}

}

\msk

a) Resolva a integral abaixo usando uma mudança de variável só (dica:
$u=g(h(x))$):
%
$$\intx{f'(g(h(x)))g'(h(x))h'(x)} = \Rq$$

b) Resolva a integral acima usando duas mudanças de variável. Dica:
comece com $u=h(x)$.

\bsk
\bsk

O Miranda e o Leithold preferem fazer em um passo só certas mudanças
de variáveis que eu prefiro fazer em dois ou três passos. Entenda o
exemplo 8.1 do Miranda -- o da seção 8.4, na página 264...

\ssk

{\scriptsize

% (find-books "__analysis/__analysis.el" "miranda")
% (find-books "__analysis/__analysis.el" "miranda" "8.4 Substituição Trigonométrica")
% (find-dmirandacalcpage 263 "8.4 Substituição Trigonométrica")
% http://hostel.ufabc.edu.br/~daniel.miranda/calculo/calculo.pdf#263
\url{http://hostel.ufabc.edu.br/~daniel.miranda/calculo/calculo.pdf\#263}

}

}\anothercol{

% a

  c) ...e descubra como resolver a integral dele fazendo duas mudanças
  de variáveis ao invés de uma só. A segunda mudança de variável vai
  ser $s = \sen θ$, e a primeira eu prefiro não contar qual é -- tente
  usar as idéias do exercício 1 pra descobrir qual ela tem que ser.



}}



\newpage

%\printbibliography

\GenericWarning{Success:}{Success!!!}  % Used by `M-x cv'

\end{document}

%  ____  _             _         
% |  _ \(_)_   ___   _(_)_______ 
% | | | | \ \ / / | | | |_  / _ \
% | |_| | |\ V /| |_| | |/ /  __/
% |____// | \_/  \__,_|_/___\___|
%     |__/                       
%
% «djvuize»  (to ".djvuize")
% (find-LATEXgrep "grep --color -nH --null -e djvuize 2020-1*.tex")

 (eepitch-shell)
 (eepitch-kill)
 (eepitch-shell)
# (find-fline "~/2022.2-C2/")
# (find-fline "~/LATEX/2022-2-C2/")
# (find-fline "~/bin/djvuize")

cd /tmp/
for i in *.jpg; do echo f $(basename $i .jpg); done

f () { rm -v $1.pdf;  textcleaner -f 50 -o  5 $1.jpg $1.png; djvuize $1.pdf; xpdf $1.pdf }
f () { rm -v $1.pdf;  textcleaner -f 50 -o 10 $1.jpg $1.png; djvuize $1.pdf; xpdf $1.pdf }
f () { rm -v $1.pdf;  textcleaner -f 50 -o 20 $1.jpg $1.png; djvuize $1.pdf; xpdf $1.pdf }

f () { rm -fv $1.png $1.pdf; djvuize $1.pdf }
f () { rm -fv $1.png $1.pdf; djvuize WHITEBOARDOPTS="-m 1.0 -f 15" $1.pdf; xpdf $1.pdf }
f () { rm -fv $1.png $1.pdf; djvuize WHITEBOARDOPTS="-m 1.0 -f 30" $1.pdf; xpdf $1.pdf }
f () { rm -fv $1.png $1.pdf; djvuize WHITEBOARDOPTS="-m 1.0 -f 45" $1.pdf; xpdf $1.pdf }
f () { rm -fv $1.png $1.pdf; djvuize WHITEBOARDOPTS="-m 0.5" $1.pdf; xpdf $1.pdf }
f () { rm -fv $1.png $1.pdf; djvuize WHITEBOARDOPTS="-m 0.25" $1.pdf; xpdf $1.pdf }
f () { cp -fv $1.png $1.pdf       ~/2022.2-C2/
       cp -fv        $1.pdf ~/LATEX/2022-2-C2/
       cat <<%%%
% (find-latexscan-links "C2" "$1")
%%%
}

f 20201213_area_em_funcao_de_theta
f 20201213_area_em_funcao_de_x
f 20201213_area_fatias_pizza



%  __  __       _        
% |  \/  | __ _| | _____ 
% | |\/| |/ _` | |/ / _ \
% | |  | | (_| |   <  __/
% |_|  |_|\__,_|_|\_\___|
%                        
% <make>

 (eepitch-shell)
 (eepitch-kill)
 (eepitch-shell)
# (find-LATEXfile "2019planar-has-1.mk")
make -f 2019.mk STEM=2022-2-C2-subst-trig veryclean
make -f 2019.mk STEM=2022-2-C2-subst-trig pdf

% Local Variables:
% coding: utf-8-unix
% ee-tla: "c2st"
% ee-tla: "c2m222strig"
% End:
