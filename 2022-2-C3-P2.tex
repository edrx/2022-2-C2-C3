% (find-LATEX "2022-2-C3-P2.tex")
% (defun c () (interactive) (find-LATEXsh "lualatex -record 2022-2-C3-P2.tex" :end))
% (defun C () (interactive) (find-LATEXsh "lualatex 2022-2-C3-P2.tex" "Success!!!"))
% (defun D () (interactive) (find-pdf-page      "~/LATEX/2022-2-C3-P2.pdf"))
% (defun d () (interactive) (find-pdftools-page "~/LATEX/2022-2-C3-P2.pdf"))
% (defun e () (interactive) (find-LATEX "2022-2-C3-P2.tex"))
% (defun o () (interactive) (find-LATEX "2022-2-C3-P2.tex"))
% (defun u () (interactive) (find-latex-upload-links "2022-2-C3-P2"))
% (defun v () (interactive) (find-2a '(e) '(d)))
% (defun d0 () (interactive) (find-ebuffer "2022-2-C3-P2.pdf"))
% (defun cv () (interactive) (C) (ee-kill-this-buffer) (v) (g))
%          (code-eec-LATEX "2022-2-C3-P2")
% (find-pdf-page   "~/LATEX/2022-2-C3-P2.pdf")
% (find-sh0 "cp -v  ~/LATEX/2022-2-C3-P2.pdf /tmp/")
% (find-sh0 "cp -v  ~/LATEX/2022-2-C3-P2.pdf /tmp/pen/")
%     (find-xournalpp "/tmp/2022-2-C3-P2.pdf")
%   file:///home/edrx/LATEX/2022-2-C3-P2.pdf
%               file:///tmp/2022-2-C3-P2.pdf
%           file:///tmp/pen/2022-2-C3-P2.pdf
% http://angg.twu.net/LATEX/2022-2-C3-P2.pdf
% (find-LATEX "2019.mk")
% (find-sh0 "cd ~/LUA/; cp -v Pict2e1.lua Pict2e1-1.lua Piecewise1.lua ~/LATEX/")
% (find-sh0 "cd ~/LUA/; cp -v Pict2e1.lua Pict2e1-1.lua Pict3D1.lua ~/LATEX/")
% (find-sh0 "cd ~/LUA/; cp -v C2Subst1.lua C2Formulas1.lua ~/LATEX/")
% (find-CN-aula-links "2022-2-C3-P2" "3" "c3m222p2" "c3p2")

% «.defs»		(to "defs")
% «.defs-T-and-B»	(to "defs-T-and-B")
% «.title»		(to "title")
% «.links»		(to "links")
% «.questao-1»		(to "questao-1")
%   «.elipse»		(to "elipse")
% «.questao-2»		(to "questao-2")
% «.questao-3»		(to "questao-3")
% «.questao-4»		(to "questao-4")
% «.grids»		(to "grids")
% «.dicas-diferenciais»	(to "dicas-diferenciais")
%
% «.questao-1-gab»	(to "questao-1-gab")
% «.questao-2-gab»	(to "questao-2-gab")
% «.questao-3-gab»	(to "questao-3-gab")
% «.questao-4-gab»	(to "questao-4-gab")
%
% «.djvuize»		(to "djvuize")



% <videos>
% Video (not yet):
% (find-ssr-links     "c3m222p2" "2022-2-C3-P2")
% (code-eevvideo      "c3m222p2" "2022-2-C3-P2")
% (code-eevlinksvideo "c3m222p2" "2022-2-C3-P2")
% (find-c3m222p2video "0:00")

\documentclass[oneside,12pt]{article}
\usepackage[colorlinks,citecolor=DarkRed,urlcolor=DarkRed]{hyperref} % (find-es "tex" "hyperref")
\usepackage{amsmath}
\usepackage{amsfonts}
\usepackage{amssymb}
\usepackage{pict2e}
\usepackage[x11names,svgnames]{xcolor} % (find-es "tex" "xcolor")
\usepackage{colorweb}                  % (find-es "tex" "colorweb")
%\usepackage{tikz}
%
% (find-dn6 "preamble6.lua" "preamble0")
%\usepackage{proof}   % For derivation trees ("%:" lines)
%\input diagxy        % For 2D diagrams ("%D" lines)
%\xyoption{curve}     % For the ".curve=" feature in 2D diagrams
%
\usepackage{edrx21}               % (find-LATEX "edrx21.sty")
\input edrxaccents.tex            % (find-LATEX "edrxaccents.tex")
\input edrx21chars.tex            % (find-LATEX "edrx21chars.tex")
\input edrxheadfoot.tex           % (find-LATEX "edrxheadfoot.tex")
\input edrxgac2.tex               % (find-LATEX "edrxgac2.tex")
%\usepackage{emaxima}              % (find-LATEX "emaxima.sty")
%
% (find-es "tex" "geometry")
\usepackage[a6paper, landscape,
            top=1.5cm, bottom=.25cm, left=1cm, right=1cm, includefoot
           ]{geometry}
%
\begin{document}

\catcode`\^^J=10
\directlua{dofile "dednat6load.lua"}  % (find-LATEX "dednat6load.lua")
%L dofile "Piecewise1.lua"           -- (find-LATEX "Piecewise1.lua")
% %L dofile "QVis1.lua"                -- (find-LATEX "QVis1.lua")
% %L dofile "Pict3D1.lua"              -- (find-LATEX "Pict3D1.lua")
% %L dofile "C2Formulas1.lua"          -- (find-LATEX "C2Formulas1.lua")
% %L Pict2e.__index.suffix = "%"
\pu
\def\pictgridstyle{\color{GrayPale}\linethickness{0.3pt}}
\def\pictaxesstyle{\linethickness{0.5pt}}
\def\pictnaxesstyle{\color{GrayPale}\linethickness{0.5pt}}
\celllower=2.5pt

% «defs»  (to ".defs")
% (find-LATEX "edrx21defs.tex" "colors")
% (find-LATEX "edrx21.sty")

\def\u#1{\par{\footnotesize \url{#1}}}

\def\drafturl{http://angg.twu.net/LATEX/2022-2-C3.pdf}
\def\drafturl{http://angg.twu.net/2022.2-C3.html}
\def\draftfooter{\tiny \href{\drafturl}{\jobname{}} \ColorBrown{\shorttoday{} \hours}}

% «defs-T-and-B»  (to ".defs-T-and-B")
\long\def\ColorOrange#1{{\color{orange!90!black}#1}}
\def\T(Total: #1 pts){{\bf(Total: #1)}}
\def\T(Total: #1 pts){{\bf(Total: #1 pts)}}
\def\T(Total: #1 pts){\ColorRed{\bf(Total: #1 pts)}}
\def\B       (#1 pts){\ColorOrange{\bf(#1 pts)}}




%  _____ _ _   _                               
% |_   _(_) |_| | ___   _ __   __ _  __ _  ___ 
%   | | | | __| |/ _ \ | '_ \ / _` |/ _` |/ _ \
%   | | | | |_| |  __/ | |_) | (_| | (_| |  __/
%   |_| |_|\__|_|\___| | .__/ \__,_|\__, |\___|
%                      |_|          |___/      
%
% «title»  (to ".title")
% (c3m222p2p 1 "title")
% (c3m222p2a   "title")

\thispagestyle{empty}

\begin{center}

\vspace*{1.2cm}

{\bf \Large Cálculo 3 - 2022.2}

\bsk

P2 (Segunda prova)

\bsk

Eduardo Ochs - RCN/PURO/UFF

\url{http://angg.twu.net/2022.2-C3.html}

\end{center}

\newpage

% «links»  (to ".links")
% (c3m222dicasp2p 6 "abertos-e-fechados")
% (c3m222dicasp2a   "abertos-e-fechados")
% (c3m222dicasp2p 5 "maximos-e-minimos")
% (c3m222dicasp2a   "maximos-e-minimos")
% (c3m222dicasp2p 6 "notacao-de-fisicos")
% (c3m222dicasp2a   "notacao-de-fisicos")
%   (c3m222dpp 3 "um-exemplo")
%   (c3m222dpa   "um-exemplo")
%   (c3m222dpp 2 "links")
%   (c3m222dpa   "links")
%   (find-books "__analysis/__analysis.el" "leithold")
%   (find-books "__analysis/__analysis.el" "leithold"   "reescritas usando")
%   (find-leitholdptpage (+ 17 275)   "reescritas usando notação de Leibniz")


%   ___                  _                _ 
%  / _ \ _   _  ___  ___| |_ __ _  ___   / |
% | | | | | | |/ _ \/ __| __/ _` |/ _ \  | |
% | |_| | |_| |  __/\__ \ || (_| | (_) | | |
%  \__\_\\__,_|\___||___/\__\__,_|\___/  |_|
%                                           
% «questao-1»  (to ".questao-1")
% «elipse»  (to ".elipse")
% (c3m222p2p 2 "elipse")
% (c3m222p2a   "elipse")

%L Pict2e.bounds = PictBounds.new(v(-2,-2), v(2,2))
%L spec = "(0,1)--(1,1)--(2,4)--(3,5)--(4,4)o (4,3)c (4,1)o--(6,3)--(7,3)"
%L spec = ""
%L pws = PwSpec.from(spec)
%L pws:topict():prethickness("1pt"):pgat("pgatc"):sa("grid Q1"):output()
\pu


\scalebox{0.5}{\def\colwidth{10.5cm}\firstcol{

%\vspace*{-0.4cm}

{\Large \bf Questão 1}

\ssk

\T(Total: 3.5 pts)

\msk

Sejam:
%
$$\begin{array}{rcl}
  P(x,y) &=& x^2 + y^2, \\
  H(x,y) &=& xy, \\ 
  E(x,y) &=& x^2 + 4y^2. \\ 
  % A &=& \setofxyst{x,y∈\{-2,-1,0,1,2\}} \\ 
  \end{array}
$$

Represente graficamente:

a) \B (0.1 pts) o diagrama de numerozinhos de $P(x,y)$,

b) \B (0.2 pts) o digrama de numerozinhos de $H(x,y)$,

c) \B (0.2 pts) o diagrama de numerozinhos de $E(x,y)$,

d) \B (0.1 pts) pelo menos 5 curvas de nível de $P(x,y)$,

e) \B (0.2 pts) pelo menos 5 curvas de nível de $H(x,y)$,

f) \B (0.2 pts) pelo menos 5 curvas de nível de $E(x,y)$,

\msk

E os conjuntos abaixo:

g) \B (0.2 pts) $C_1 = E^{-1}(4)$

h) \B (0.2 pts) $C_2 = E^{-1}(1)$

i) \B (0.3 pts) $C_3 = E^{-1}([1,4))$

j) \B (0.3 pts) $C_4 = H^{-1}([-2,1))$

k) \B (0.5 pts) $C_5 = C_3 ∩ C_4$

l) \B (0.5 pts) $C_6 = \Int(C_5)$

m) \B (0.5 pts) $C_7 = \overline{C_5}$



\msk

Use os grids da página 4.

Indique claramente qual desenho é a resposta de cada item e quais
desenhos são rascunhos.

}\anothercol{

% «questao-2»  (to ".questao-2")
% (c3m222p2p 2 "questao-2")
% (c3m222p2a   "questao-2")

{\Large \bf Questão 2}

\ssk

\T(Total: 2.5 pts)

\msk

Sejam:
%
$$\begin{array}{rcl}
  z &=& (x-x_0)^4 (y-y_0)^6, \\
  α &=& x+y, \\
  β &=& x-y, \\
  w &=& (α^3-α)+β^2. \\
  \end{array}
$$

Nesta questão eu vou ver principalmente quais dos truques da ``notação
de físicos'' você sabe usar direito.

\msk

A página 5 tem um monte de dicas de ``notação de físicos'' que você
pode usar como referência. A coluna da esquerda dessa página tem um
exemplo grande que nós vimos em aula; a parte de cima da coluna da
direita tem uma tabela que eu copiei da página 275 do Leithold, na
qual ele mostra como reescrever certas regras de derivação usando
diferenciais; e a parte de baixo da coluna da direita é uma versão
adaptada do primeiro exemplo do capítulo XVI do Silvanus Thompson, em
que ele mostra como fazer contas ficarem menores criando variáveis
dependentes novas.

\msk

Calcule:

a) \B (0.2 pts) $\frac{dz}{dx}$,

b) \B (0.3 pts) $z_{xx}$,

c) \B (0.5 pts) $dz$,

d) \B (1.5 pts) $dw$.

\msk

No item c tente chegar até uma expressão da forma $z_xdx + z_ydy$, e
no item d tente chegar até uma expressão da forma forma
$w_xdx + w_ydy$.



}}

\newpage

% «questao-3»  (to ".questao-3")
% (c3m222p2p 3 "questao-3")
% (c3m222p2a   "questao-3")

\scalebox{0.6}{\def\colwidth{9cm}\firstcol{

{\Large \bf Questão 3}

\ssk

\T(Total: 3.0 pts)

\msk

Sejam
%
$$\begin{array}{rcl}
  z(x,y) &=& dx^2 + exy + fy^2, \\
  h(x) &=& z(x,1). \\
  \end{array}
$$

Vou dizer que a função $h(x,y)$ é a ``função homogênea de grau 2
associada a $h(x)$''.

\msk

a) \B (1.5 pts) Digamos que
%
$$h(x) = -2(x-1)(x+1).$$
%
Faça o diagrama de sinais da $h(x)$ (em $\R$), os numerozinhos da
função $z(x,y)$ nos pontos com $y=1$ e $x∈\{-2,-1,0,1,2\}$ (siiiim, só
5 pontos!) e o diagrama de sinais da função $z(x,y)$ (em $\R^2$), e
diga se o ponto $(0,0)$ é um mínimo, máximo, ponto de sela, etc, etc.

\msk

b) \B (1.5 pts) Agora digamos que
%
$$h(x) = (x-i)(x+i) = x^2+1.$$
%
Faça as mesmas coisas para esta função $h(x)$ e para a função $z(x,y)$
associada a ela.

}\anothercol{

% «questao-4»  (to ".questao-4")
% (c3m222p2p 3 "questao-4")
% (c3m222p2a   "questao-4")

{\Large \bf Questão 4}

\ssk

\T(Total: 3.0 pts)

\msk

Sejam:
%
$$\begin{array}{rcl}
  H(x,y) &=& xy, \\ 
  E(x,y) &=& x^2 + 4y^2, \\ 
  D &=& E^{-1}([0,16]), \\
  F &:& D \to \R \\
    && (x,y) \mapsto H(x,y) \\
  % A &=& \setofxyst{x,y∈\{-2,-1,0,1,2\}} \\ 
  \end{array}
$$

Agora só queremos olhar pro que acontece dentro do ``domínio'' $D$,
que é uma elipse; note que a função $F(x,y)$ só está definida em $D$.

Faça pelo menos 5 curvas de nível de $z=F(x,y)$ (obs: só dentro da
elipse!!!) e mostre no seu gráfico quais dos pontos de $D$ são máximos
locais, mínimos locais ou pontos de sela.




}}



%  (eepitch-maxima)
%  (eepitch-kill)
%  (eepitch-maxima)
% z : (x-x0)^4 * (y-y0^6);
% diff(z,x);
% diff(z,x,2);
% aa : x+y;
% bb : x-y;




\newpage

% «grids»  (to ".grids")
% (c3m222p2p 4 "grids")
% (c3m222p2a   "grids")

\unitlength=10pt

\def\Gr{\scalebox{1.2}{$\ga{grid Q1}$}}

$\begin{matrix}
 \Gr & \Gr & \Gr & \Gr & \Gr \\
 \Gr & \Gr & \Gr & \Gr & \Gr \\
 \Gr & \Gr & \Gr & \Gr & \Gr \\
 \Gr & \Gr & \Gr & \Gr & \Gr \\
 \end{matrix}
$



\newpage

%  ____  _                     _ _  __  __     
% |  _ \(_) ___ __ _ ___    __| (_)/ _|/ _|___ 
% | | | | |/ __/ _` / __|  / _` | | |_| |_/ __|
% | |_| | | (_| (_| \__ \ | (_| | |  _|  _\__ \
% |____/|_|\___\__,_|___/  \__,_|_|_| |_| |___/
%                                              
% «dicas-diferenciais»  (to ".dicas-diferenciais")
% (c3m222p2p 5 "dicas-diferenciais")
% (c3m222p2a   "dicas-diferenciais")

\sa{myexample-body}{
   z &=& (x^3 + y^4)^5 \\
  \\[-7pt]
  \frac{∂z}{∂x} &=& \frac{∂}{∂x}(x^3+y^4)^5 \\
                &=& 5(x^3 + y^4)^4 \frac{∂}{∂x}(x^3+y^4) \\
                &=& 5(x^3 + y^4)^4 (\frac{∂}{∂x}x^3+\frac{∂}{∂x}y^4) \\
                &=& 5(x^3 + y^4)^4 (3x^2) \\
  \\[-7pt]
  \frac{∂z}{∂y} &=& \frac{∂}{∂y}(x^3+y^4)^5 \\
                &=& 5(x^3 + y^4)^4 \frac{∂}{∂y}(x^3+y^4) \\
                &=& 5(x^3 + y^4)^4 (\frac{∂}{∂y}x^3+\frac{∂}{∂y}y^4) \\
                &=& 5(x^3 + y^4)^4 (4y^3) \\
  \\[-7pt]
  dz &=& 5(x^3 + y^4)^4 \, d(x^3+y^4) \\
     &=& 5(x^3 + y^4)^4 (dx^3+dy^4) \\
     &=& 5(x^3 + y^4)^4 (3x^2 \, dx + 4y^3 \, dy) \\
     &=& 5(x^3 + y^4)^4 (3x^2) dx + 5(x^3 + y^4)^4 (4y^3) dy \\
  \\[-7pt]
  dz &=& z_x dx + z_y dy \\
}

\sa{leithold-body}{
  \frac{d(c)}{dx}   &=& 0 \\
  \frac{d(x^n)}{dx} &=& nx^{n-1} \\
  \frac{d(cu)}{dx}  &=& c\frac{du}{dx} \\
  \frac{d(u+v)}{dx} &=& \frac{du}{dx} + \frac{dv}{dx} \\
  \frac{d(uv)}{dx}  &=& u\frac{dv}{dx} + v\frac{du}{dx} \\
  \frac{d(\frac{u}{v})}{dx} &=&
     \frac{v\frac{du}{dx} - u\frac{dv}{dx}}{v^2} \\
  \frac{d(u^n)}{dx} &=& nu^{n-1} \frac{du}{dx} \\
  \\[-7pt]
}

\sa{leithold-body2}{
  d(c)   &=& 0 \\
  d(x^n) &=& nx^{n-1}dx \\
  d(cu)  &=& c\,du \\
  d(u+v) &=& du+dv  \\
  d(uv)  &=& u\,dv + v\,du \\
  d(\frac{u}{v}) &=&
     \frac{v\,du - u\,dv}{v^2} \\
  d(u^n) &=& nu^{n-1} du \\
}

\sa{thompson-body}{
  y &=& (x^2+a^2)^{3/2} \\
  u &=& x^2+a^2 \\
  du &=& 2x\,dx \\
  dy &=& d((x^2+a^2)^{3/2}) \\
     &=& d(u^{3/2}) \\
     &=& u^{1/2}\,du \\
     &=& u^{1/2}·2x\,dx \\
     &=& (x^2+a^2)^{1/2}·2x\,dx \\
}


\scalebox{0.65}{\def\colwidth{9cm}\firstcol{

\msk

$\begin{array}{rcl}
 \ga{myexample-body}
 \end{array}
 %\qquad
 \hspace*{-1cm}
 \begin{array}{c}
   \begin{array}{rcl}
   \ga{leithold-body}
   \end{array}
   \quad
   \begin{array}{rcl}
   \ga{leithold-body2}
   \end{array}
   \\
   \\
   \begin{array}{rcl}
   \ga{thompson-body}
   \end{array}
 \end{array}
$


}\anothercol{
}}


\newpage

% «questao-1-gab»  (to ".questao-1-gab")
% «questao-2-gab»  (to ".questao-2-gab")
% (c3m222p2p 6 "questao-2-gab")
% (c3m222p2a   "questao-2-gab")

\def\dzdx{\frac{dz}{dx}}
\def\und#1#2{\underbrace{#1}_{#2}}


\scalebox{0.55}{\def\colwidth{10.5cm}\firstcol{

{\bf \Large Questão 2: gabarito}

\bsk

$\begin{array}[t]{lrcl}
  \text{Temos:} &
    z &=& (x-x_0)^4 (y-y_0)^6 \\
  \text{Sejam:} &
    u &=& x-x_0, \\
  & v &=& y-y_0. \\
    \\[-5pt]
  \text{Então:}
  & z     &=& u^4 v^6, \\
  & \dzdx &=& \ddx(u^4)v^6 + u^4\ddx(v^6) \\
         &&=& (4u^3\ddx u)v^6 + u^4(6v^5\ddx v) \\
         &&=& (4u^3\ddx(x-x_0))v^6 + u^4(6v^5\ddx(y-y_0)) \\
         &&=& 4u^3v^6 \\
         &&=& 4(x-x_0)^3(y-y_0)^6, \\
    \\[-5pt]
  & z_{xx} &=& \ddx \ddx z \\
          &&=& \ddx (4u^3v^6) \\
          &&=& 4(\ddx(u^3)v^6 + u^3\ddx(v^6)) \\
          &&=& 4(\ddx(u^3)v^6) \\
          &&=& 4(3u^2\ddx(u)v^6) \\
          &&=& 4(3u^2v^6) \\
          &&=& 12u^2v^6 \\
          &&=& 12(x-x_0)^2(y-y_0)^6, \\
    \\[-5pt]
  & dz &=& d(u^4v^6) \\
      &&=& d(u^4)v^6 + u^4d(v^6) \\
      &&=& (4u^3du)v^6 + u^4(6v^5dv) \\
      &&=& (4u^3v^6)dx + (6u^4v^5)dy \\
      &&=& 4(x-x_0)^3(y-y_0)^6dx + 6(x-x_0)^4(y-y_0)^5dy \\
  %\text{Obs:}
  \end{array}
$

}\anothercol{

\vspace*{0.5cm}

$\begin{array}[t]{lrcl}
  \text{Temos:} &
    α &=& x+y, \\
  & β &=& x-y, \\
  & w &=& (α^3-α)+β^2. \\
    \\[-5pt]
  \text{Então:} &
    dw &=& d(α^3+α) + d(β^2) \\
      &&=& (2α+1)dα + 2βdβ \\
      &&=& (2α+1)d(x+y) + 2βd(x-y) \\
      &&=& (2α+1)(dx+dy) + 2β(dx-dy) \\
      &&=& (2α+1+2β)dx + (2α+1-2β)dy \\
      &&=& (2(x+y)+1+2(x-y))dx \\
      &&+& (2(x+y)+1-2(x-y))dy \\
      &&=& (4x+1)dx + (4y+1)dx \\
  \end{array}
$

}}


\newpage



% «questao-3-gab»  (to ".questao-3-gab")
% «questao-4-gab»  (to ".questao-4-gab")




\GenericWarning{Success:}{Success!!!}  % Used by `M-x cv'

\end{document}

%  ____  _             _         
% |  _ \(_)_   ___   _(_)_______ 
% | | | | \ \ / / | | | |_  / _ \
% | |_| | |\ V /| |_| | |/ /  __/
% |____// | \_/  \__,_|_/___\___|
%     |__/                       
%
% «djvuize»  (to ".djvuize")
% (find-LATEXgrep "grep --color -nH --null -e djvuize 2020-1*.tex")

 (eepitch-shell)
 (eepitch-kill)
 (eepitch-shell)
# (find-fline "~/2022.2-C3/")
# (find-fline "~/LATEX/2022-2-C3/")
# (find-fline "~/bin/djvuize")

cd /tmp/
for i in *.jpg; do echo f $(basename $i .jpg); done

f () { rm -v $1.pdf;  textcleaner -f 50 -o  5 $1.jpg $1.png; djvuize $1.pdf; xpdf $1.pdf }
f () { rm -v $1.pdf;  textcleaner -f 50 -o 10 $1.jpg $1.png; djvuize $1.pdf; xpdf $1.pdf }
f () { rm -v $1.pdf;  textcleaner -f 50 -o 20 $1.jpg $1.png; djvuize $1.pdf; xpdf $1.pdf }

f () { rm -fv $1.png $1.pdf; djvuize $1.pdf }
f () { rm -fv $1.png $1.pdf; djvuize WHITEBOARDOPTS="-m 1.0 -f 15" $1.pdf; xpdf $1.pdf }
f () { rm -fv $1.png $1.pdf; djvuize WHITEBOARDOPTS="-m 1.0 -f 30" $1.pdf; xpdf $1.pdf }
f () { rm -fv $1.png $1.pdf; djvuize WHITEBOARDOPTS="-m 1.0 -f 45" $1.pdf; xpdf $1.pdf }
f () { rm -fv $1.png $1.pdf; djvuize WHITEBOARDOPTS="-m 0.5" $1.pdf; xpdf $1.pdf }
f () { rm -fv $1.png $1.pdf; djvuize WHITEBOARDOPTS="-m 0.25" $1.pdf; xpdf $1.pdf }
f () { cp -fv $1.png $1.pdf       ~/2022.2-C3/
       cp -fv        $1.pdf ~/LATEX/2022-2-C3/
       cat <<%%%
% (find-latexscan-links "C3" "$1")
%%%
}

f 20201213_area_em_funcao_de_theta
f 20201213_area_em_funcao_de_x
f 20201213_area_fatias_pizza



%  __  __       _        
% |  \/  | __ _| | _____ 
% | |\/| |/ _` | |/ / _ \
% | |  | | (_| |   <  __/
% |_|  |_|\__,_|_|\_\___|
%                        
% <make>

 (eepitch-shell)
 (eepitch-kill)
 (eepitch-shell)
# (find-LATEXfile "2019planar-has-1.mk")
make -f 2019.mk STEM=2022-2-C3-P2 veryclean
make -f 2019.mk STEM=2022-2-C3-P2 pdf

% Local Variables:
% coding: utf-8-unix
% ee-tla: "c3p2"
% ee-tla: "c3m222p2"
% End:
