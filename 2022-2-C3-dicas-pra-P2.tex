% (find-LATEX "2022-2-C3-dicas-pra-P2.tex")
% (defun c () (interactive) (find-LATEXsh "lualatex -record 2022-2-C3-dicas-pra-P2.tex" :end))
% (defun C () (interactive) (find-LATEXsh "lualatex 2022-2-C3-dicas-pra-P2.tex" "Success!!!"))
% (defun D () (interactive) (find-pdf-page      "~/LATEX/2022-2-C3-dicas-pra-P2.pdf"))
% (defun d () (interactive) (find-pdftools-page "~/LATEX/2022-2-C3-dicas-pra-P2.pdf"))
% (defun e () (interactive) (find-LATEX "2022-2-C3-dicas-pra-P2.tex"))
% (defun o () (interactive) (find-LATEX "2022-2-C3-dicas-pra-P2.tex"))
% (defun u () (interactive) (find-latex-upload-links "2022-2-C3-dicas-pra-P2"))
% (defun v () (interactive) (find-2a '(e) '(d)))
% (defun d0 () (interactive) (find-ebuffer "2022-2-C3-dicas-pra-P2.pdf"))
% (defun cv () (interactive) (C) (ee-kill-this-buffer) (v) (g))
%          (code-eec-LATEX "2022-2-C3-dicas-pra-P2")
% (find-pdf-page   "~/LATEX/2022-2-C3-dicas-pra-P2.pdf")
% (find-sh0 "cp -v  ~/LATEX/2022-2-C3-dicas-pra-P2.pdf /tmp/")
% (find-sh0 "cp -v  ~/LATEX/2022-2-C3-dicas-pra-P2.pdf /tmp/pen/")
%     (find-xournalpp "/tmp/2022-2-C3-dicas-pra-P2.pdf")
%   file:///home/edrx/LATEX/2022-2-C3-dicas-pra-P2.pdf
%               file:///tmp/2022-2-C3-dicas-pra-P2.pdf
%           file:///tmp/pen/2022-2-C3-dicas-pra-P2.pdf
% http://angg.twu.net/LATEX/2022-2-C3-dicas-pra-P2.pdf
% (find-LATEX "2019.mk")
% (find-sh0 "cd ~/LUA/; cp -v Pict2e1.lua Pict2e1-1.lua Piecewise1.lua ~/LATEX/")
% (find-sh0 "cd ~/LUA/; cp -v Pict2e1.lua Pict2e1-1.lua Pict3D1.lua ~/LATEX/")
% (find-sh0 "cd ~/LUA/; cp -v C2Subst1.lua C2Formulas1.lua ~/LATEX/")
% (find-CN-aula-links "2022-2-C3-dicas-pra-P2" "3" "c3m222dicasp2" "c3d2")

% «.defs»	(to "defs")
% «.title»	(to "title")
% «.links»	(to "links")



% <videos>
% Video (not yet):
% (find-ssr-links     "c3m222dicasp2" "2022-2-C3-dicas-pra-P2")
% (code-eevvideo      "c3m222dicasp2" "2022-2-C3-dicas-pra-P2")
% (code-eevlinksvideo "c3m222dicasp2" "2022-2-C3-dicas-pra-P2")
% (find-c3m222dicasp2video "0:00")

\documentclass[oneside,12pt]{article}
\usepackage[colorlinks,citecolor=DarkRed,urlcolor=DarkRed]{hyperref} % (find-es "tex" "hyperref")
\usepackage{amsmath}
\usepackage{amsfonts}
\usepackage{amssymb}
\usepackage{pict2e}
\usepackage[x11names,svgnames]{xcolor} % (find-es "tex" "xcolor")
\usepackage{colorweb}                  % (find-es "tex" "colorweb")
%\usepackage{tikz}
%
% (find-dn6 "preamble6.lua" "preamble0")
%\usepackage{proof}   % For derivation trees ("%:" lines)
%\input diagxy        % For 2D diagrams ("%D" lines)
%\xyoption{curve}     % For the ".curve=" feature in 2D diagrams
%
\usepackage{edrx21}               % (find-LATEX "edrx21.sty")
\input edrxaccents.tex            % (find-LATEX "edrxaccents.tex")
\input edrx21chars.tex            % (find-LATEX "edrx21chars.tex")
\input edrxheadfoot.tex           % (find-LATEX "edrxheadfoot.tex")
\input edrxgac2.tex               % (find-LATEX "edrxgac2.tex")
%\usepackage{emaxima}              % (find-LATEX "emaxima.sty")
%
%\usepackage[backend=biber,
%   style=alphabetic]{biblatex}            % (find-es "tex" "biber")
%\addbibresource{catsem-slides.bib}        % (find-LATEX "catsem-slides.bib")
%
% (find-es "tex" "geometry")
\usepackage[a6paper, landscape,
            top=1.5cm, bottom=.25cm, left=1cm, right=1cm, includefoot
           ]{geometry}
%
\begin{document}

\catcode`\^^J=10
\directlua{dofile "dednat6load.lua"}  % (find-LATEX "dednat6load.lua")
%L dofile "Piecewise1.lua"           -- (find-LATEX "Piecewise1.lua")
%L dofile "QVis1.lua"                -- (find-LATEX "QVis1.lua")
%L dofile "Pict3D1.lua"              -- (find-LATEX "Pict3D1.lua")
%L dofile "C2Formulas1.lua"          -- (find-LATEX "C2Formulas1.lua")
%L Pict2e.__index.suffix = "%"
\pu
\def\pictgridstyle{\color{GrayPale}\linethickness{0.3pt}}
\def\pictaxesstyle{\linethickness{0.5pt}}
\def\pictnaxesstyle{\color{GrayPale}\linethickness{0.5pt}}
\celllower=2.5pt

% «defs»  (to ".defs")
% (find-LATEX "edrx21defs.tex" "colors")
% (find-LATEX "edrx21.sty")

\def\u#1{\par{\footnotesize \url{#1}}}

\def\drafturl{http://angg.twu.net/LATEX/2022-2-C3.pdf}
\def\drafturl{http://angg.twu.net/2022.2-C3.html}
\def\draftfooter{\tiny \href{\drafturl}{\jobname{}} \ColorBrown{\shorttoday{} \hours}}



%  _____ _ _   _                               
% |_   _(_) |_| | ___   _ __   __ _  __ _  ___ 
%   | | | | __| |/ _ \ | '_ \ / _` |/ _` |/ _ \
%   | | | | |_| |  __/ | |_) | (_| | (_| |  __/
%   |_| |_|\__|_|\___| | .__/ \__,_|\__, |\___|
%                      |_|          |___/      
%
% «title»  (to ".title")
% (c3m222dicasp2p 1 "title")
% (c3m222dicasp2a   "title")

\thispagestyle{empty}

\begin{center}

\vspace*{1.2cm}

{\bf \Large Cálculo 3 - 2022.2}

\bsk

Dicas pra P2

\bsk

Eduardo Ochs - RCN/PURO/UFF

\url{http://angg.twu.net/2022.2-C3.html}

\end{center}

\newpage

% «links»  (to ".links")
% (c3m222dicasp2p 2 "links")
% (c3m222dicasp2a   "links")
% (c3m222plcp 1 "conteudo")
% (c3m222plca   "conteudo")

{\bf Assuntos da prova}

\scalebox{0.8}{\def\colwidth{12cm}\firstcol{

As principais questões da P2 vão ser sobre estes itens do
programa do curso:

\msk

\begin{tabular}{lp{9cm}}
2.3. & Noções de conjuntos abertos e fechados no $\R^n$. \\
2.4. & Limite e continuidade. Definição e propriedades. \\
3.1. & Derivadas parciais. \\
3.7. & Derivadas parciais de ordens superiores. \\
3.8. & Fórmula de Taylor. \\
4.   & Máximos e mínimos. \\
4.1. & Extremos relativos. Condição necessária para a existência de extremos relativos. \\
4.2. & Ponto crítico. Teste da derivada segunda. \\
4.3. & Máximos e mínimos sobre um compacto. \\
\end{tabular}

\msk

As questões vão ser principalmente sobre os itens 4, 4.1, 4.2 e 4.3,
mas você vai precisar dos itens 2.3, 2.4, 3.1, 3.7 e 3.8 pra fazer
algumas partes delas.

}\anothercol{
}}

\newpage

{\bf Como estudar em casa}

\scalebox{0.8}{\def\colwidth{13.5cm}\firstcol{

    Comecem entendendo os conceitos dos itens 4, 4.1, 4.2 e 4.3. O
    Bortolossi tem boas explicações pra eles nos capítulos 10, 11 e
    12:

\msk

{\footnotesize

% (find-bortolossi10page (+ -350 351) "10. Máximos e mínimos de funções de várias variáveis")
% (find-bortolossi10page (+ -350 351) "10.1. Definições e exemplos")
% (find-bortolossi10page (+ -350 357) "10.2. Exercícios")
\url{http://angg.twu.net/2019.2-C3/Bortolossi/bortolossi-cap-10.pdf}

% (find-bortolossi11page (+ -364 365) "11. Otimização sem restrições")
% (find-bortolossi11page (+ -364 365)   "Teorema 11.1: A regra de Fermat")
% (find-bortolossi11page (+ -364 371) "11.2. Polinômios de Taylor")
% (find-bortolossi11page (+ -364 375)   "Teorema 11.3: Condições de segunda ordem")
% (find-bortolossi11page (+ -364 376)   "Polinômios de Taylor de ordem k")
% (find-bortolossi11page (+ -364 383) "11.3. Formas quadráticas e matrizes definidas")
% (find-bortolossi11page (+ -364 391)   "Teorema 11.7: Condições de segunda ordem")
% (find-bortolossi11page (+ -364 395) "11.4. Menores de uma matriz")
% (find-bortolossi11page (+ -364 405) "11.5. Questões de globalidade e convexidade")
% (find-bortolossi11page (+ -364 420) "11.6. O método dos mínimos quadrados")
% (find-bortolossi11page (+ -364 426) "11.7. Exercícios")
\url{http://angg.twu.net/2019.2-C3/Bortolossi/bortolossi-cap-11.pdf}

% (find-bortolossi12page (+ -456 457) "12. Otimização com restrições")
% (find-bortolossi12page (+ -456 457)   "multiplicadores de Lagrange")
% (find-bortolossi12page (+ -456 459) "12.1. Otimização com uma restrição em igualdade")
% (find-bortolossi12page (+ -456 462)   "Teorema 12.1")
% (find-bortolossi12page (+ -456 463)     "Lagrangeano")
% (find-bortolossi12page (+ -456 472)   "Teorema 12.3")
% (find-bortolossi12page (+ -456 473) "12.2. Otimização com várias restrições em igualdade")
% (find-bortolossi12page (+ -456 495) "12.4")
% (find-bortolossi12page (+ -456 506) "12.5")
% (find-bortolossi12page (+ -456 512) "12.6")
% (find-bortolossi12page (+ -456 514) "12.7")
% (find-bortolossi12page (+ -456 515) "12.8")
% (find-bortolossi12page (+ -456 520) "12.9")
% (find-bortolossi12page (+ -456 521) "12.10 Exercícios")
\url{http://angg.twu.net/2019.2-C3/Bortolossi/bortolossi-cap-12.pdf}

}

\msk

Comece lendo as partes mais legíveis desses capítulos e entendendo as
figu\-ras. Depois tente entender os enunciados dos exercícios.

\msk

Nas próximas aulas nós vamos fazer muitos exercícios que são versões
``desmontadas'' de exercícios do Bortolossi. Por exemplo, os
exercícios 1 a 4 daqui

\ssk

{\footnotesize

% (c3m222topp 4 "exercicios-1-e-2")
% (c3m222topa   "exercicios-1-e-2")
%    http://angg.twu.net/LATEX/2022-2-C3-topologia.pdf#page=4
\url{http://angg.twu.net/LATEX/2022-2-C3-topologia.pdf\#page=4}

}

\ssk

são parecidos com passos do exercício 8 da página 359 do Bortolossi:

\ssk

{\footnotesize

% (find-bortolossi10page (+ -350 359)   "[08] Use o Teorema de Weierstrass")
\url{http://angg.twu.net/2019.2-C3/Bortolossi/bortolossi-cap-10.pdf\#page=9}

}


}\anothercol{
}}






% (find-books "__analysis/__analysis.el" "bortolossi" "10. Máximos e mínimos")
% (find-books "__analysis/__analysis.el" "bortolossi" "11. Otimização sem restrições")
% (find-books "__analysis/__analysis.el" "bortolossi" "12. Otimização com restrições")
% (find-angg ".emacs" "c3q192" "e Teorema de Weierstrass")


\GenericWarning{Success:}{Success!!!}  % Used by `M-x cv'

\end{document}



% Local Variables:
% coding: utf-8-unix
% ee-tla: "c3d2"
% ee-tla: "c3m222dicasp2"
% End:
