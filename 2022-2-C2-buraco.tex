% (find-LATEX "2022-2-C2-buraco.tex")
% (defun c () (interactive) (find-LATEXsh "lualatex -record 2022-2-C2-buraco.tex" :end))
% (defun C () (interactive) (find-LATEXsh "lualatex 2022-2-C2-buraco.tex" "Success!!!"))
% (defun D () (interactive) (find-pdf-page      "~/LATEX/2022-2-C2-buraco.pdf"))
% (defun d () (interactive) (find-pdftools-page "~/LATEX/2022-2-C2-buraco.pdf"))
% (defun e () (interactive) (find-LATEX "2022-2-C2-buraco.tex"))
% (defun o () (interactive) (find-LATEX "2022-2-C2-buraco.tex"))
% (defun u () (interactive) (find-latex-upload-links "2022-2-C2-buraco"))
% (defun v () (interactive) (find-2a '(e) '(d)))
% (defun d0 () (interactive) (find-ebuffer "2022-2-C2-buraco.pdf"))
% (defun cv () (interactive) (C) (ee-kill-this-buffer) (v) (g))
%          (code-eec-LATEX "2022-2-C2-buraco")
% (find-pdf-page   "~/LATEX/2022-2-C2-buraco.pdf")
% (find-sh0 "cp -v  ~/LATEX/2022-2-C2-buraco.pdf /tmp/")
% (find-sh0 "cp -v  ~/LATEX/2022-2-C2-buraco.pdf /tmp/pen/")
%     (find-xournalpp "/tmp/2022-2-C2-buraco.pdf")
%   file:///home/edrx/LATEX/2022-2-C2-buraco.pdf
%               file:///tmp/2022-2-C2-buraco.pdf
%           file:///tmp/pen/2022-2-C2-buraco.pdf
% http://angg.twu.net/LATEX/2022-2-C2-buraco.pdf
% (find-LATEX "2019.mk")
% (find-sh0 "cd ~/LUA/; cp -v Pict2e1.lua Pict2e1-1.lua Piecewise1.lua ~/LATEX/")
% (find-sh0 "cd ~/LUA/; cp -v Pict2e1.lua Pict2e1-1.lua Pict3D1.lua ~/LATEX/")
% (find-sh0 "cd ~/LUA/; cp -v C2Subst1.lua C2Formulas1.lua ~/LATEX/")
% (find-CN-aula-links "2022-2-C2-buraco" "2" "c2m222buraco" "c2bu")

% «.defs»	(to "defs")
% «.title»	(to "title")
%
% «.djvuize»	(to "djvuize")



% <videos>
% Video (not yet):
% (find-ssr-links     "c2m222buraco" "2022-2-C2-buraco")
% (code-eevvideo      "c2m222buraco" "2022-2-C2-buraco")
% (code-eevlinksvideo "c2m222buraco" "2022-2-C2-buraco")
% (find-c2m222buracovideo "0:00")

\documentclass[oneside,12pt]{article}
\usepackage[colorlinks,citecolor=DarkRed,urlcolor=DarkRed]{hyperref} % (find-es "tex" "hyperref")
\usepackage{amsmath}
\usepackage{amsfonts}
\usepackage{amssymb}
\usepackage{pict2e}
\usepackage[x11names,svgnames]{xcolor} % (find-es "tex" "xcolor")
\usepackage{colorweb}                  % (find-es "tex" "colorweb")
%\usepackage{tikz}
%
% (find-dn6 "preamble6.lua" "preamble0")
%\usepackage{proof}   % For derivation trees ("%:" lines)
%\input diagxy        % For 2D diagrams ("%D" lines)
%\xyoption{curve}     % For the ".curve=" feature in 2D diagrams
%
\usepackage{edrx21}               % (find-LATEX "edrx21.sty")
\input edrxaccents.tex            % (find-LATEX "edrxaccents.tex")
\input edrx21chars.tex            % (find-LATEX "edrx21chars.tex")
\input edrxheadfoot.tex           % (find-LATEX "edrxheadfoot.tex")
\input edrxgac2.tex               % (find-LATEX "edrxgac2.tex")
%\usepackage{emaxima}              % (find-LATEX "emaxima.sty")
%
%\usepackage[backend=biber,
%   style=alphabetic]{biblatex}            % (find-es "tex" "biber")
%\addbibresource{catsem-slides.bib}        % (find-LATEX "catsem-slides.bib")
%
% (find-es "tex" "geometry")
\usepackage[a6paper, landscape,
            top=1.5cm, bottom=.25cm, left=1cm, right=1cm, includefoot
           ]{geometry}
%
\begin{document}

\catcode`\^^J=10
\directlua{dofile "dednat6load.lua"}  % (find-LATEX "dednat6load.lua")
%L dofile "Piecewise1.lua"           -- (find-LATEX "Piecewise1.lua")
%L dofile "QVis1.lua"                -- (find-LATEX "QVis1.lua")
%L dofile "Pict3D1.lua"              -- (find-LATEX "Pict3D1.lua")
%L dofile "C2Formulas1.lua"          -- (find-LATEX "C2Formulas1.lua")
%L Pict2e.__index.suffix = "%"
\pu
\def\pictgridstyle{\color{GrayPale}\linethickness{0.3pt}}
\def\pictaxesstyle{\linethickness{0.5pt}}
\def\pictnaxesstyle{\color{GrayPale}\linethickness{0.5pt}}
\celllower=2.5pt

% «defs»  (to ".defs")
% (find-LATEX "edrx21defs.tex" "colors")
% (find-LATEX "edrx21.sty")

\def\u#1{\par{\footnotesize \url{#1}}}

\def\drafturl{http://angg.twu.net/LATEX/2022-2-C2.pdf}
\def\drafturl{http://angg.twu.net/2022.2-C2.html}
\def\draftfooter{\tiny \href{\drafturl}{\jobname{}} \ColorBrown{\shorttoday{} \hours}}



%  _____ _ _   _                               
% |_   _(_) |_| | ___   _ __   __ _  __ _  ___ 
%   | | | | __| |/ _ \ | '_ \ / _` |/ _` |/ _ \
%   | | | | |_| |  __/ | |_) | (_| | (_| |  __/
%   |_| |_|\__|_|\___| | .__/ \__,_|\__, |\___|
%                      |_|          |___/      
%
% «title»  (to ".title")
% (c2m222buracop 1 "title")
% (c2m222buracoa   "title")

\thispagestyle{empty}

\begin{center}

\vspace*{1.2cm}

{\bf \Large Cálculo 2 - 2022.2}

\bsk

Aula 1: a operação `$[:=]$', ou:

aqui o curso tem um buraco

\bsk

Eduardo Ochs - RCN/PURO/UFF

\url{http://angg.twu.net/2022.2-C2.html}

\end{center}

\newpage

{\bf O macaco}

Você já deve ter assistido o vídeo do Mathologer sobre o ``Calculus
Made Easy'':

\ssk

\url{http://angg.twu.net/mathologer-calculus-easy.html}

\ssk

Eu vou usar esse vídeo como uma espécie de mapa pra um monte de idéias
importantes de curso de Cálculo 2, e o Mathologer --- obs: às vezes eu
vou chamar ele de Burkard, que é o nome dele... ele respondeu um
e-mail meu, então vou fingir ele é meu amigo, tá {\smile} --- mas,
bom, voltando: o Mathologer diz várias vezes que a gente pode treinar
um macaco pra calcular derivadas, e isso vai ser uma das coisas mais
importantes do meu curso de Cálculo 2. Deixa eu explicar.

Num curso tradicional de Cálculo 1 a gente faz centenas de horas de
contas na mão. Aí a gente adquire muita prática nisso e a gente passa
a poder fazer o papel do macaco muito bem. Depois que a gente tem essa
prática toda a gente {\sl começa} a poder fazer também um outro papel,
que é o papel na pessoa que programa o macaco e diz quais regras ele
tem que seguir...

Deixa eu dar um exemplo. No trecho do vídeo que começa no 17:00 a
gente vê como o macaco calcula a derivada
$\left(\frac{5+\sin x}{x^3 \ln x}\right)'$ ``fazendo a álgebra no
piloto automático''. Se a gente seguir o vídeo com bastante atenção a
gente vê que o macaco não está usando só as 5 regras pra derivada que
aparecem na coluna esquerda no vídeo no 16:15, que o Burkard escreve
como $(f+g)' = f'+g'$, $(f-g)' = f'-g'$, $(fg)' = f'g + fg'$,
$(\frac{f}{g})' = \frac{f'g-fg'}{g^2}$ e $(f(g))' = f'(g)g'$... o
macaco também usa as regras que o Burkard põe na tabela na parte de
cima da tela no 13:00, e no 17:35 ele calcula em separado o resultado
de $(5 + \sin x)'$ --- que dá $\cos x$ --- e no 17:50 o macaco
substitui o $(5 + \sin x)'$ na expressão original por $\cos x$.

% (find-calceasyvideo "16:15")
% (find-calceasyvideo "17:00")
% (find-calceasyvideo "17:30")
% (find-calceasyvideo "17:35")
% (find-calceasyvideo "18:15")

\newpage

{\bf Exercício 1.}

Acesse o PDF do capítulo 2 do Leithold.

Entre as páginas 68 e 70 ele tem várias contas de limites feitas passo
a passo, com os sinais de `=' alinhados e com justificativas à
direita. Esse formato está explicado aqui:

\ssk

{\footnotesize

% (c2m212introp 7 "linguagem")
% (c2m212introa   "linguagem")
%    http://angg.twu.net/LATEX/2021-2-C2-intro.pdf#page=7
\url{http://angg.twu.net/LATEX/2021-2-C2-intro.pdf#page=7}

}

\ssk

Faça as contas que o Mathologer faz entre o 17:00 e o 18:15 nesse
formato, com os `='s alinhados e as justificativas à direita, e com as
restrições que eu vou pôr no quadro (vou PDFizar elas depois)...


% (find-leitholdptpage (+ 17  68)   "se ... então")



\GenericWarning{Success:}{Success!!!}  % Used by `M-x cv'

\end{document}

%  ____  _             _         
% |  _ \(_)_   ___   _(_)_______ 
% | | | | \ \ / / | | | |_  / _ \
% | |_| | |\ V /| |_| | |/ /  __/
% |____// | \_/  \__,_|_/___\___|
%     |__/                       
%
% «djvuize»  (to ".djvuize")
% (find-LATEXgrep "grep --color -nH --null -e djvuize 2020-1*.tex")



%  __  __       _        
% |  \/  | __ _| | _____ 
% | |\/| |/ _` | |/ / _ \
% | |  | | (_| |   <  __/
% |_|  |_|\__,_|_|\_\___|
%                        
% <make>

 (eepitch-shell)
 (eepitch-kill)
 (eepitch-shell)
# (find-LATEXfile "2019planar-has-1.mk")
make -f 2019.mk STEM=2022-2-C2-buraco veryclean
make -f 2019.mk STEM=2022-2-C2-buraco pdf

% Local Variables:
% coding: utf-8-unix
% ee-tla: "c2bu"
% ee-tla: "c2m222buraco"
% End:
