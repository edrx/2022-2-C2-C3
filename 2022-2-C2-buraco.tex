% (find-LATEX "2022-2-C2-buraco.tex")
% (defun c () (interactive) (find-LATEXsh "lualatex -record 2022-2-C2-buraco.tex" :end))
% (defun C () (interactive) (find-LATEXsh "lualatex 2022-2-C2-buraco.tex" "Success!!!"))
% (defun D () (interactive) (find-pdf-page      "~/LATEX/2022-2-C2-buraco.pdf"))
% (defun d () (interactive) (find-pdftools-page "~/LATEX/2022-2-C2-buraco.pdf"))
% (defun e () (interactive) (find-LATEX "2022-2-C2-buraco.tex"))
% (defun o () (interactive) (find-LATEX "2022-2-C2-buraco.tex"))
% (defun u () (interactive) (find-latex-upload-links "2022-2-C2-buraco"))
% (defun v () (interactive) (find-2a '(e) '(d)))
% (defun d0 () (interactive) (find-ebuffer "2022-2-C2-buraco.pdf"))
% (defun cv () (interactive) (C) (ee-kill-this-buffer) (v) (g))
%          (code-eec-LATEX "2022-2-C2-buraco")
% (find-pdf-page   "~/LATEX/2022-2-C2-buraco.pdf")
% (find-sh0 "cp -v  ~/LATEX/2022-2-C2-buraco.pdf /tmp/")
% (find-sh0 "cp -v  ~/LATEX/2022-2-C2-buraco.pdf /tmp/pen/")
%     (find-xournalpp "/tmp/2022-2-C2-buraco.pdf")
%   file:///home/edrx/LATEX/2022-2-C2-buraco.pdf
%               file:///tmp/2022-2-C2-buraco.pdf
%           file:///tmp/pen/2022-2-C2-buraco.pdf
% http://angg.twu.net/LATEX/2022-2-C2-buraco.pdf
% (find-LATEX "2019.mk")
% (find-sh0 "cd ~/LUA/; cp -v Pict2e1.lua Pict2e1-1.lua Piecewise1.lua ~/LATEX/")
% (find-sh0 "cd ~/LUA/; cp -v Pict2e1.lua Pict2e1-1.lua Pict3D1.lua ~/LATEX/")
% (find-sh0 "cd ~/LUA/; cp -v C2Subst1.lua C2Formulas1.lua ~/LATEX/")
% (find-CN-aula-links "2022-2-C2-buraco" "2" "c2m222buraco" "c2bu")

% «.defs»			(to "defs")
% «.title»			(to "title")
% «.macaco»			(to "macaco")
% «.exercicio-1»		(to "exercicio-1")
% «.derivadas-formais»		(to "derivadas-formais")
%   «.maxima-workbook»		(to "maxima-workbook")
%   «.dx-quando»		(to "dx-quando")
% «.exercicio-2»		(to "exercicio-2")
%   «.exercicio-arvores»	(to "exercicio-arvores")
% «.exercicio-2-dicas»		(to "exercicio-2-dicas")
%   «.dx-fecha-parentese»	(to "dx-fecha-parentese")
% «.menos-elementares»		(to "menos-elementares")
% «.chutar-e-testar»		(to "chutar-e-testar")
% «.leibniz-links»		(to "leibniz-links")
%
% «.djvuize»			(to "djvuize")



% <videos>
% Video (not yet):
% (find-ssr-links     "c2m222buraco" "2022-2-C2-buraco")
% (code-eevvideo      "c2m222buraco" "2022-2-C2-buraco")
% (code-eevlinksvideo "c2m222buraco" "2022-2-C2-buraco")
% (find-c2m222buracovideo "0:00")

\documentclass[oneside,12pt]{article}
\usepackage[colorlinks,citecolor=DarkRed,urlcolor=DarkRed]{hyperref} % (find-es "tex" "hyperref")
\usepackage{amsmath}
\usepackage{amsfonts}
\usepackage{amssymb}
\usepackage{pict2e}
\usepackage[x11names,svgnames]{xcolor} % (find-es "tex" "xcolor")
\usepackage{colorweb}                  % (find-es "tex" "colorweb")
%\usepackage{tikz}
%
% (find-dn6 "preamble6.lua" "preamble0")
%\usepackage{proof}   % For derivation trees ("%:" lines)
%\input diagxy        % For 2D diagrams ("%D" lines)
%\xyoption{curve}     % For the ".curve=" feature in 2D diagrams
%
\usepackage{edrx21}               % (find-LATEX "edrx21.sty")
\input edrxaccents.tex            % (find-LATEX "edrxaccents.tex")
\input edrx21chars.tex            % (find-LATEX "edrx21chars.tex")
\input edrxheadfoot.tex           % (find-LATEX "edrxheadfoot.tex")
\input edrxgac2.tex               % (find-LATEX "edrxgac2.tex")
%\usepackage{emaxima}              % (find-LATEX "emaxima.sty")
%
%\usepackage[backend=biber,
%   style=alphabetic]{biblatex}            % (find-es "tex" "biber")
%\addbibresource{catsem-slides.bib}        % (find-LATEX "catsem-slides.bib")
%
% (find-es "tex" "geometry")
\usepackage[a6paper, landscape,
            top=1.5cm, bottom=.25cm, left=1cm, right=1cm, includefoot
           ]{geometry}
%
\begin{document}

\catcode`\^^J=10
\directlua{dofile "dednat6load.lua"}  % (find-LATEX "dednat6load.lua")
%L dofile "Piecewise1.lua"           -- (find-LATEX "Piecewise1.lua")
%L dofile "QVis1.lua"                -- (find-LATEX "QVis1.lua")
%L dofile "Pict3D1.lua"              -- (find-LATEX "Pict3D1.lua")
%L dofile "C2Formulas1.lua"          -- (find-LATEX "C2Formulas1.lua")
%L Pict2e.__index.suffix = "%"
\pu
\def\pictgridstyle{\color{GrayPale}\linethickness{0.3pt}}
\def\pictaxesstyle{\linethickness{0.5pt}}
\def\pictnaxesstyle{\color{GrayPale}\linethickness{0.5pt}}
\celllower=2.5pt

% «defs»  (to ".defs")
% (find-LATEX "edrx21defs.tex" "colors")
% (find-LATEX "edrx21.sty")

\def\u#1{\par{\footnotesize \url{#1}}}

\def\drafturl{http://angg.twu.net/LATEX/2022-2-C2.pdf}
\def\drafturl{http://angg.twu.net/2022.2-C2.html}
\def\draftfooter{\tiny \href{\drafturl}{\jobname{}} \ColorBrown{\shorttoday{} \hours}}



%  _____ _ _   _                               
% |_   _(_) |_| | ___   _ __   __ _  __ _  ___ 
%   | | | | __| |/ _ \ | '_ \ / _` |/ _` |/ _ \
%   | | | | |_| |  __/ | |_) | (_| | (_| |  __/
%   |_| |_|\__|_|\___| | .__/ \__,_|\__, |\___|
%                      |_|          |___/      
%
% «title»  (to ".title")
% (c2m222buracop 1 "title")
% (c2m222buracoa   "title")

\thispagestyle{empty}

\begin{center}

\vspace*{1.2cm}

{\bf \Large Cálculo 2 - 2022.2}

\bsk

Aulas 1 e 3: a operação `$[:=]$', ou:

aqui o curso tem um buraco

\bsk

Eduardo Ochs - RCN/PURO/UFF

\url{http://angg.twu.net/2022.2-C2.html}

\end{center}

\newpage

% «macaco»  (to ".macaco")
% (c2m222buracop 2 "macaco")
% (c2m222buracoa   "macaco")

{\bf O macaco}

Você já deve ter assistido o vídeo do Mathologer sobre o ``Calculus
Made Easy'':

\ssk

\url{http://angg.twu.net/mathologer-calculus-easy.html}

\ssk

Eu vou usar esse vídeo como uma espécie de mapa pra um monte de idéias
importantes de curso de Cálculo 2, e o Mathologer --- obs: às vezes eu
vou chamar ele de Burkard, que é o nome dele... ele respondeu um
e-mail meu, então vou fingir ele é meu amigo, tá {\smile} --- mas,
bom, voltando: o Mathologer diz várias vezes que a gente pode treinar
um macaco pra calcular derivadas, e isso vai ser uma das coisas mais
importantes do meu curso de Cálculo 2. Deixa eu explicar.

Num curso tradicional de Cálculo 1 a gente faz centenas de horas de
contas na mão. Aí a gente adquire muita prática nisso e a gente passa
a poder fazer o papel do macaco muito bem. Depois que a gente tem essa
prática toda a gente {\sl começa} a poder fazer também um outro papel,
que é o papel na pessoa que programa o macaco e diz quais regras ele
tem que seguir...

Deixa eu dar um exemplo. No trecho do vídeo que começa no 17:00 a
gente vê como o macaco calcula a derivada
$\left(\frac{5+\sin x}{x^3 \ln x}\right)'$ ``fazendo a álgebra no
piloto automático''. Se a gente seguir o vídeo com bastante atenção a
gente vê que o macaco não está usando só as 5 regras pra derivada que
aparecem na coluna esquerda no vídeo no 16:15, que o Burkard escreve
como $(f+g)' = f'+g'$, $(f-g)' = f'-g'$, $(fg)' = f'g + fg'$,
$(\frac{f}{g})' = \frac{f'g-fg'}{g^2}$ e $(f(g))' = f'(g)g'$... o
macaco também usa as regras que o Burkard põe na tabela na parte de
cima da tela no 13:00, e no 17:35 ele calcula em separado o resultado
de $(5 + \sin x)'$ --- que dá $\cos x$ --- e no 17:50 o macaco
substitui o $(5 + \sin x)'$ na expressão original por $\cos x$.

% (find-calceasyvideo "16:15")
% (find-calceasyvideo "17:00")
% (find-calceasyvideo "17:30")
% (find-calceasyvideo "17:35")
% (find-calceasyvideo "18:15")

\newpage

% «exercicio-1»  (to ".exercicio-1")
% (c2m222buracop 4 "exercicio-1")
% (c2m222buracoa   "exercicio-1")

{\bf Exercício 1.}

Acesse o PDF do capítulo 2 do Leithold.

Entre as páginas 68 e 70 ele tem várias contas de limites feitas passo
a passo, com os sinais de `=' alinhados e com justificativas à
direita. Esse formato está explicado aqui:

\ssk

{\footnotesize

% (c2m212introp 7 "linguagem")
% (c2m212introa   "linguagem")
%    http://angg.twu.net/LATEX/2021-2-C2-intro.pdf#page=7
\url{http://angg.twu.net/LATEX/2021-2-C2-intro.pdf#page=7}

}

\ssk

Faça as contas que o Mathologer faz entre o 17:00 e o 18:15 nesse
formato, com os `='s alinhados e as justificativas à direita, e com as
restrições que eu vou pôr no quadro (vou PDFizar elas depois)...


\newpage

% «derivadas-formais»  (to ".derivadas-formais")
% (c2m222buracop 5 "derivadas-formais")
% (c2m222buracoa   "derivadas-formais")

{\bf Aula 3: Derivadas formais (e algumas coisas relacionadas)}

\ssk

% (find-books "__analysis/__analysis.el" "leithold")
% (find-leitholdptpage (+ 17 156) "3.3. Teoremas sobre derivação de funções algébricas")

Dê uma olhada na seção 3.3 do Leithold (no capítulo 3). Os teoremas
3.3.1 até 3.3.7 dele correspondem exatamente às operações de derivação
que o Mathologer usa no vídeo (obs: a regra da cadeia é o teorema
3.6.1), mas cada um desses teoremas tem uma {\sl fórmula} e as {\sl
  hipóteses} necessárias pra fórmula valer. O macaco que calcula
derivadas no vídeo do Methologer no trecho entre 17:00 e 18:15 usa só
as fórmulas sem checar que as hipóteses valem. O que o macaco faz é
chamado de {\sl derivação formal}, e está explicado aqui:

\ssk

\url{https://en.wikipedia.org/wiki/Formal_derivative}

\msk

Em Cálculo 2 nós vamos ver várias operações que são tão difíceis que é
o melhor jeito de aprendê-las, e de debugar erros nelas, é dividindo
elas em várias partes. Nós acabamos de ver que ``calcular derivadas
(complicadas)'' pode ser dividido em ``aplicar fórmulas'' e ``checar
hipóteses'', e você deve lembrar que você passou a maior parte do seu
curso de Cálculo 1 só ``aplicando fórmulas'' sem ``checar hipóteses'',
ou checando as hipóteses rapidinho no olho, mas sem escrever nada tipo
``aqui as hipóteses fulana e beltrana valem, então blá''.

Nos últimos semestres {\sl mooooooontes} de alunos chegaram em Cálculo
2 sem saber ``aplicar fórmulas'' direito. Eu também vou dividir a
operação de ``aplicar uma fórmula'' em várias partes, até chegar a uma
parte --- a operação `$[:=]$' --- que é fácil de implementar num
computador e que corresponde exatamente ao que o macaco que do
Mathologer faz, mas que o Mathologer não explica explicitamente.

\msk

{\sl Se você tiver qualquer dificuldade com essa operação `[:=]' o
  melhor modo de estudá-la é estudando a matéria de Cálculo 1 e
  Cálculo 2 pelo Leithold e vendo como essa operação aparece
  implicitamente em todo lugar, mas sempre acompanhada do ``checar
  hipóteses''.}

% «maxima-workbook»  (to ".maxima-workbook")
% (c2m222buracop 7 "maxima-workbook")
% (c2m222buracoa   "maxima-workbook")

{\sl Eu costumo dividir essa operação de ``aplicar fórmula'' em várias
  operações separadas e dar um nome ``padrão'' pra cada uma dessas
  operações. Os programas de computação simbólica também fazem essa
  divisão em várias suboperações, porque cada suboperação corresponde
  a uma função diferente, e essas funções podem ser chamadas em
  separado. Eu vou usar a terminologia do Maxima, que é o programa de
  computação simbólica que eu tenho usado... o Maxima considera
  ``substituição'' e ``simplificação'' como operações separadas. O
  ``Maxima Workbook'' explica substituição na seção 11.4 e
  simplificação no capítulo 12; ele também explica ``formas
  canônicas'' nas seções 9.2 e 9.3, e eu vou considerar que ``pôr uma
  expressão na sua forma canônica'' é uma forma de simplificação.
  Link:}

\url{http://roland-salz.de/Maxima_Workbook.pdf}

\msk

Deixa eu dar um exemplo mais básico disso. Nós conhecemos a fórmula
$2a = a+a$. Se nós {\sl só} substituirmos todos os `$a$' nesta fórmula
por 10 o resultado é $2·10 = 10+10$; trocar $2·10$ por 20, ou $10+10$
por 20, são consideradas ``simplificações''.

A substituição sempre substitui {\sl variáveis} por {\sl expressões}.

\msk

% «dx-quando»  (to ".dx-quando")
% (c2m222buracop 8 "dx-quando")
% (c2m222buracoa   "dx-quando")

No vídeo do Mathologer ele às vezes abrevia $f(x)$ como $f$, e
desabrevia isso depois, e ele às vezes escreve `$df$' pra
``diferencial'' (veja a seção 4.9 do Leithold). É muito mais difícil
formalizar contas de Cálculo --- ou seja, justificar formalmente cada
passo delas --- quando a gente permite esses truques (que são parte da
``notação de Leibniz''). Na maior parte do meu curso esses truques vão
ser proibidos; ele às vezes vão ser permitidos temporariamente, mas a
gente sempre vai ver como tratar as contas em que eles são permitidos
como ``versões abreviadas'' de contas em que eles são proibidos.
Depois eu vou disponibilizar um monte de links sobre porque é que os
matemáticos pararam de usar a notação de Leibniz.

% (find-books "__comp/__comp.el" "maxima-workbook")

\newpage

%     _                                  
%    / \   _ ____   _____  _ __ ___  ___ 
%   / _ \ | '__\ \ / / _ \| '__/ _ \/ __|
%  / ___ \| |   \ V / (_) | | |  __/\__ \
% /_/   \_\_|    \_/ \___/|_|  \___||___/
%                                        
% «exercicio-2»  (to ".exercicio-2")
% (c2m222buracop 10 "exercicio-2")
% (c2m222buracoa    "exercicio-2")
% «exercicio-arvores»  (to ".exercicio-arvores")
% (c2m222buracop 10 "exercicio-arvores")
% (c2m222buracoa    "exercicio-arvores")

{\bf Exercício 2: árvores}

\ssk

Vai ser muito mais fácil a gente entender como o macaco derivador
funciona se a gente souber tratar expressões matemáticas como árvores.

Veja os dois screenshots abaixo --- eles mostram como Maxima e o Sympy
representam certas expressões como árvores. {\sl Note que as
  representações são diferentes!}

\ssk

% (find-TH "eev-maxima" "luatree")

%         (find-fline "~/IMAGES/luatree.png")
%    http://angg.twu.net/IMAGES/luatree.png
\url{http://angg.twu.net/IMAGES/luatree.png}

%         (find-fline "~/IMAGES/luatree-sympy.png")
%    http://angg.twu.net/IMAGES/luatree-sympy.png
\url{http://angg.twu.net/IMAGES/luatree-sympy.png}

\ssk

Pegue algumas expressões que você obteve no exercício 1 e represente
elas como árvores no formato do Maxima.


\newpage

% «exercicio-2-dicas»  (to ".exercicio-2-dicas")
% (c2m222buracop 11 "exercicio-2-dicas")
% (c2m222buracoa    "exercicio-2-dicas")
% «dx-fecha-parentese»  (to ".dx-fecha-parentese")
% (c2m222buracop 11 "dx-fecha-parentese")
% (c2m222buracoa    "dx-fecha-parentese")

{\bf Exercício 2: dicas}

\ssk

Lembre que nas árvores não aparecem parênteses.

Lembre que $a-b-c = (a-b)-c$.

Lembre que $a-(b-c) ≠ (a-b)-c$.

Lembre que $a-(b-c)$ e $(a-b)-c$ dão árvores diferentes.

Lembre que $a^{b^c} = a^{(b^c)}$.

\bsk

\standout{MUITO IMPORTANTE:} lembre que na maior parte do curso a
expressão `$dx$' não vai poder aparecer sozinha... eu vou até me
referir a ela a toda hora como ``uma espécie de fecha parêntese'' pra
fazer as pessoas lembrarem disso.


\newpage

% «menos-elementares»  (to ".menos-elementares")
% (c2m222buracop 12 "menos-elementares")
% (c2m222buracoa    "menos-elementares")

{\bf Funções menos elementares}

No vídeo o Mathologer primeiro define ``funções elementares'' de um
jeito, e depois, a partir do 27:26, ele diz que poderíamos ter
incluído nas nossas operação que criam novas funções elementares a
operação que obtém a inversa de uma função... ele não dá detalhes, mas
repara que se a gente puder obter a inversa de toda função a gente vai
ter que poder obter a inversa de funções constantes, como por exemplo
$f(x)=1$... então provavelmente o que ele quer dizer é que a gente
quer considerar como ``elementares'' coisas como $\sqrt{x}$,
$\sqrt[3]{x}$, $\arcsen x$, $\arccos x$, etc...

\bsk

Obs: falta \LaTeX ar as coisas que a gente viu no final da aula de
2022aug31! As fotos dos quadros desse dia estão aqui:

% (find-angg ".emacs" "c2-2022-2-quadros")
% http://angg.twu.net/2022.2-C2/C2-quadros.pdf#page=3
\url{http://angg.twu.net/2022.2-C2/C2-quadros.pdf\#page=3}


\newpage

% «chutar-e-testar»  (to ".chutar-e-testar")
% (c2m222buracop 13 "chutar-e-testar")
% (c2m222buracoa    "chutar-e-testar")
% (find-angg ".emacs" "c2q221")

{\bf EDOs por chutar-e-testar}

Links pros exercícios de hoje:

\ssk

{\footnotesize

% (c2m212introp 12 "EDOs-chutar-testar")
% (c2m212introa    "EDOs-chutar-testar")
%    http://angg.twu.net/LATEX/2021-2-C2-intro.pdf#page=12
\url{http://angg.twu.net/LATEX/2021-2-C2-intro.pdf#page=12}

% (c2q221  4 "abr06 E1" "EDOs por chutar e testar; [S1]")
% http://angg.twu.net/2022.1-C2/C2-quadros.pdf#page=4
\url{http://angg.twu.net/2022.1-C2/C2-quadros.pdf\#page=4}

% (c2m221vsbp 9 "questao-4-gab")
% (c2m221vsba   "questao-4-gab")
%    http://angg.twu.net/LATEX/2022-1-C2-VSB.pdf#page=9
\url{http://angg.twu.net/LATEX/2022-1-C2-VSB.pdf\#page=9}

% (find-books "__analysis/__analysis.el" "miranda")
% (find-books "__analysis/__analysis.el" "miranda" " 185 ")
% (find-dmirandacalcpage 185   "Exercícios")
%    http://hostel.ufabc.edu.br/~daniel.miranda/calculo/calculo.pdf
\url{http://hostel.ufabc.edu.br/~daniel.miranda/calculo/calculo.pdf\#page=185}

}


\msk

...e os exercícios da seção 5.1 do Leithold.

% (find-leitholdptpage (+ 17 294) "Exercícios 5.1")



\ssk



% (c2plp 2 "plano-de-curso")
% (c2pla   "plano-de-curso")

% Definição de solução de EDO. Integração como EDO. Integral indefinida.



% http://roland-salz.de/Maxima_Workbook.pdf#page=101

% Integrais como parênteses

% Integral dupla: bolo

% Funções menos elementares: adicionar arcsen, adicionar f e f'





% «leibniz-links»  (to ".leibniz-links")
% https://math.stackexchange.com/questions/726950/how-is-it-that-treating-leibniz-notation-as-a-fraction-is-fundamentally-incorrec
% https://math.stackexchange.com/questions/23902/what-is-the-practical-difference-between-a-differential-and-a-derivative
% https://math.stackexchange.com/questions/21199/is-frac-textrmdy-textrmdx-not-a-ratio/21209#21209
% https://math.stackexchange.com/questions/21869/if-fracdydtdt-doesnt-cancel-then-what-do-you-call-it
% https://math.stackexchange.com/questions/21199/dy-dx-is-not-a-ratio/21209#21209
% https://news.ycombinator.com/item?id=12196295 Mathematical Notation Is Awful (blackhole12.blogspot.com)
% https://blackhole12.blogspot.com/2016/07/mathematical-notation-is-awful.html
% https://math.stackexchange.com/questions/1966777/newton-vs-leibniz-notation


% (find-leitholdptpage (+ 17  68)   "se ... então")
% (find-fline "~/TH/2022.1-C2.blogme" "harper")
% forall, bound variables, lambda, 
% (find-books "__logic/__logic.el" "nederpelt-geuvers")
% https://en.wikipedia.org/wiki/Formal_calculation
% https://en.wikipedia.org/wiki/Formal_derivative
% (find-books "__analysis/__analysis.el" "livshits")


\GenericWarning{Success:}{Success!!!}  % Used by `M-x cv'

\end{document}

%  ____  _             _         
% |  _ \(_)_   ___   _(_)_______ 
% | | | | \ \ / / | | | |_  / _ \
% | |_| | |\ V /| |_| | |/ /  __/
% |____// | \_/  \__,_|_/___\___|
%     |__/                       
%
% «djvuize»  (to ".djvuize")
% (find-LATEXgrep "grep --color -nH --null -e djvuize 2020-1*.tex")



%  __  __       _        
% |  \/  | __ _| | _____ 
% | |\/| |/ _` | |/ / _ \
% | |  | | (_| |   <  __/
% |_|  |_|\__,_|_|\_\___|
%                        
% <make>

 (eepitch-shell)
 (eepitch-kill)
 (eepitch-shell)
# (find-LATEXfile "2019planar-has-1.mk")
make -f 2019.mk STEM=2022-2-C2-buraco veryclean
make -f 2019.mk STEM=2022-2-C2-buraco pdf

% Local Variables:
% coding: utf-8-unix
% ee-tla: "c2bu"
% ee-tla: "c2m222buraco"
% End:
