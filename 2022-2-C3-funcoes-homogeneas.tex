% (find-LATEX "2022-2-C3-funcoes-homogeneas.tex")
% (defun c () (interactive) (find-LATEXsh "lualatex -record 2022-2-C3-funcoes-homogeneas.tex" :end))
% (defun C () (interactive) (find-LATEXsh "lualatex 2022-2-C3-funcoes-homogeneas.tex" "Success!!!"))
% (defun D () (interactive) (find-pdf-page      "~/LATEX/2022-2-C3-funcoes-homogeneas.pdf"))
% (defun d () (interactive) (find-pdftools-page "~/LATEX/2022-2-C3-funcoes-homogeneas.pdf"))
% (defun e () (interactive) (find-LATEX "2022-2-C3-funcoes-homogeneas.tex"))
% (defun o () (interactive) (find-LATEX "2022-2-C3-funcoes-homogeneas.tex"))
% (defun u () (interactive) (find-latex-upload-links "2022-2-C3-funcoes-homogeneas"))
% (defun v () (interactive) (find-2a '(e) '(d)))
% (defun d0 () (interactive) (find-ebuffer "2022-2-C3-funcoes-homogeneas.pdf"))
% (defun cv () (interactive) (C) (ee-kill-this-buffer) (v) (g))
%          (code-eec-LATEX "2022-2-C3-funcoes-homogeneas")
% (find-pdf-page   "~/LATEX/2022-2-C3-funcoes-homogeneas.pdf")
% (find-sh0 "cp -v  ~/LATEX/2022-2-C3-funcoes-homogeneas.pdf /tmp/")
% (find-sh0 "cp -v  ~/LATEX/2022-2-C3-funcoes-homogeneas.pdf /tmp/pen/")
%     (find-xournalpp "/tmp/2022-2-C3-funcoes-homogeneas.pdf")
%   file:///home/edrx/LATEX/2022-2-C3-funcoes-homogeneas.pdf
%               file:///tmp/2022-2-C3-funcoes-homogeneas.pdf
%           file:///tmp/pen/2022-2-C3-funcoes-homogeneas.pdf
% http://angg.twu.net/LATEX/2022-2-C3-funcoes-homogeneas.pdf
% (find-LATEX "2019.mk")
% (find-sh0 "cd ~/LUA/; cp -v Pict2e1.lua Pict2e1-1.lua Piecewise1.lua ~/LATEX/")
% (find-sh0 "cd ~/LUA/; cp -v Pict2e1.lua Pict2e1-1.lua Pict3D1.lua ~/LATEX/")
% (find-sh0 "cd ~/LUA/; cp -v C2Subst1.lua C2Formulas1.lua ~/LATEX/")
% (find-CN-aula-links "2022-2-C3-funcoes-homogeneas" "3" "c3m222fh" "c3fh")

% «.defs»	(to "defs")
% «.title»	(to "title")
% «.links»	(to "links")
%
% «.djvuize»	(to "djvuize")



% <videos>
% Video (not yet):
% (find-ssr-links     "c3m222fh" "2022-2-C3-funcoes-homogeneas")
% (code-eevvideo      "c3m222fh" "2022-2-C3-funcoes-homogeneas")
% (code-eevlinksvideo "c3m222fh" "2022-2-C3-funcoes-homogeneas")
% (find-c3m222fhvideo "0:00")

\documentclass[oneside,12pt]{article}
\usepackage[colorlinks,citecolor=DarkRed,urlcolor=DarkRed]{hyperref} % (find-es "tex" "hyperref")
\usepackage{amsmath}
\usepackage{amsfonts}
\usepackage{amssymb}
\usepackage{pict2e}
\usepackage[x11names,svgnames]{xcolor} % (find-es "tex" "xcolor")
\usepackage{colorweb}                  % (find-es "tex" "colorweb")
%\usepackage{tikz}
%
% (find-dn6 "preamble6.lua" "preamble0")
%\usepackage{proof}   % For derivation trees ("%:" lines)
%\input diagxy        % For 2D diagrams ("%D" lines)
%\xyoption{curve}     % For the ".curve=" feature in 2D diagrams
%
\usepackage{edrx21}               % (find-LATEX "edrx21.sty")
\input edrxaccents.tex            % (find-LATEX "edrxaccents.tex")
\input edrx21chars.tex            % (find-LATEX "edrx21chars.tex")
\input edrxheadfoot.tex           % (find-LATEX "edrxheadfoot.tex")
\input edrxgac2.tex               % (find-LATEX "edrxgac2.tex")
%\usepackage{emaxima}              % (find-LATEX "emaxima.sty")
%
%\usepackage[backend=biber,
%   style=alphabetic]{biblatex}            % (find-es "tex" "biber")
%\addbibresource{catsem-slides.bib}        % (find-LATEX "catsem-slides.bib")
%
% (find-es "tex" "geometry")
\usepackage[a6paper, landscape,
            top=1.5cm, bottom=.25cm, left=1cm, right=1cm, includefoot
           ]{geometry}
%
\begin{document}

\catcode`\^^J=10
\directlua{dofile "dednat6load.lua"}  % (find-LATEX "dednat6load.lua")
%L dofile "Piecewise1.lua"           -- (find-LATEX "Piecewise1.lua")
%L dofile "QVis1.lua"                -- (find-LATEX "QVis1.lua")
%L dofile "Pict3D1.lua"              -- (find-LATEX "Pict3D1.lua")
%L dofile "C2Formulas1.lua"          -- (find-LATEX "C2Formulas1.lua")
%L Pict2e.__index.suffix = "%"
\pu
\def\pictgridstyle{\color{GrayPale}\linethickness{0.3pt}}
\def\pictaxesstyle{\linethickness{0.5pt}}
\def\pictnaxesstyle{\color{GrayPale}\linethickness{0.5pt}}
\celllower=2.5pt

% «defs»  (to ".defs")
% (find-LATEX "edrx21defs.tex" "colors")
% (find-LATEX "edrx21.sty")

\def\u#1{\par{\footnotesize \url{#1}}}

\def\drafturl{http://angg.twu.net/LATEX/2022-2-C3.pdf}
\def\drafturl{http://angg.twu.net/2022.2-C3.html}
\def\draftfooter{\tiny \href{\drafturl}{\jobname{}} \ColorBrown{\shorttoday{} \hours}}



%  _____ _ _   _                               
% |_   _(_) |_| | ___   _ __   __ _  __ _  ___ 
%   | | | | __| |/ _ \ | '_ \ / _` |/ _` |/ _ \
%   | | | | |_| |  __/ | |_) | (_| | (_| |  __/
%   |_| |_|\__|_|\___| | .__/ \__,_|\__, |\___|
%                      |_|          |___/      
%
% «title»  (to ".title")
% (c3m222fhp 1 "title")
% (c3m222fha   "title")

\thispagestyle{empty}

\begin{center}

\vspace*{1.2cm}

{\bf \Large Cálculo 3 - 2022.2}

\bsk

Aula 22: Funções homogêneas e

derivadas parciais de ordens superiores.

\bsk

Eduardo Ochs - RCN/PURO/UFF

\url{http://angg.twu.net/2022.2-C3.html}

\end{center}

\newpage

% «links»  (to ".links")

{\bf Links}

\ssk

{\scriptsize

% (c3m221fhp 2 "exercicio-1")
% (c3m221fha   "exercicio-1")
%    http://angg.twu.net/LATEX/2021-2-C3-funcoes-homogeneas.pdf#page=2
\url{http://angg.twu.net/LATEX/2021-2-C3-funcoes-homogeneas.pdf#page=2}

}



\bsk

Nessa aula eu escrevi muuuuitas coisas no

quadro, e ainda não tive tempo de digitá-las.

Você pode acessar os quadros dessa aula aqui:

\ssk

{\scriptsize

% (find-angg ".emacs" "c3q222")
% (find-angg ".emacs" "c3q222" "Funções homogêneas")
%    http://angg.twu.net/2022.2-C3/C3-quadros.pdf#page=17
\url{http://angg.twu.net/2022.2-C3/C3-quadros.pdf\#page=17}

}

\ssk




% Pôr polinômios homogêneos de duas variáveis numa página.
% Fazer exercícios de graus de monômios e de decompor 
% polis em partes homogêneas de vários graus.

% Definição: a reta que passa pela origem gerada pelo ponto (a,b).
% Ao invés de t a gente vai usar lambda.
% Fazer exercícios de pontos na mesma reta que passa pela origem.
% Link pra retas degeneradas que são um ponto só.

% Relação com Taylor. As retas 

%\printbibliography

\GenericWarning{Success:}{Success!!!}  % Used by `M-x cv'

\end{document}

%  ____  _             _         
% |  _ \(_)_   ___   _(_)_______ 
% | | | | \ \ / / | | | |_  / _ \
% | |_| | |\ V /| |_| | |/ /  __/
% |____// | \_/  \__,_|_/___\___|
%     |__/                       
%
% «djvuize»  (to ".djvuize")
% (find-LATEXgrep "grep --color -nH --null -e djvuize 2020-1*.tex")

 (eepitch-shell)
 (eepitch-kill)
 (eepitch-shell)
# (find-fline "~/2022.2-C3/")
# (find-fline "~/LATEX/2022-2-C3/")
# (find-fline "~/bin/djvuize")

cd /tmp/
for i in *.jpg; do echo f $(basename $i .jpg); done

f () { rm -v $1.pdf;  textcleaner -f 50 -o  5 $1.jpg $1.png; djvuize $1.pdf; xpdf $1.pdf }
f () { rm -v $1.pdf;  textcleaner -f 50 -o 10 $1.jpg $1.png; djvuize $1.pdf; xpdf $1.pdf }
f () { rm -v $1.pdf;  textcleaner -f 50 -o 20 $1.jpg $1.png; djvuize $1.pdf; xpdf $1.pdf }

f () { rm -fv $1.png $1.pdf; djvuize $1.pdf }
f () { rm -fv $1.png $1.pdf; djvuize WHITEBOARDOPTS="-m 1.0 -f 15" $1.pdf; xpdf $1.pdf }
f () { rm -fv $1.png $1.pdf; djvuize WHITEBOARDOPTS="-m 1.0 -f 30" $1.pdf; xpdf $1.pdf }
f () { rm -fv $1.png $1.pdf; djvuize WHITEBOARDOPTS="-m 1.0 -f 45" $1.pdf; xpdf $1.pdf }
f () { rm -fv $1.png $1.pdf; djvuize WHITEBOARDOPTS="-m 0.5" $1.pdf; xpdf $1.pdf }
f () { rm -fv $1.png $1.pdf; djvuize WHITEBOARDOPTS="-m 0.25" $1.pdf; xpdf $1.pdf }
f () { cp -fv $1.png $1.pdf       ~/2022.2-C3/
       cp -fv        $1.pdf ~/LATEX/2022-2-C3/
       cat <<%%%
% (find-latexscan-links "C3" "$1")
%%%
}

f 20201213_area_em_funcao_de_theta
f 20201213_area_em_funcao_de_x
f 20201213_area_fatias_pizza



%  __  __       _        
% |  \/  | __ _| | _____ 
% | |\/| |/ _` | |/ / _ \
% | |  | | (_| |   <  __/
% |_|  |_|\__,_|_|\_\___|
%                        
% <make>

 (eepitch-shell)
 (eepitch-kill)
 (eepitch-shell)
# (find-LATEXfile "2019planar-has-1.mk")
make -f 2019.mk STEM=2022-2-C3-funcoes-homogeneas veryclean
make -f 2019.mk STEM=2022-2-C3-funcoes-homogeneas pdf

% Local Variables:
% coding: utf-8-unix
% ee-tla: "c3fh"
% ee-tla: "c3m222fh"
% End:
