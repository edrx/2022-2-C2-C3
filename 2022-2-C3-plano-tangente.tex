% (find-LATEX "2022-2-C3-plano-tangente.tex")
% (defun c () (interactive) (find-LATEXsh "lualatex -record 2022-2-C3-plano-tangente.tex" :end))
% (defun C () (interactive) (find-LATEXsh "lualatex 2022-2-C3-plano-tangente.tex" "Success!!!"))
% (defun D () (interactive) (find-pdf-page      "~/LATEX/2022-2-C3-plano-tangente.pdf"))
% (defun d () (interactive) (find-pdftools-page "~/LATEX/2022-2-C3-plano-tangente.pdf"))
% (defun e () (interactive) (find-LATEX "2022-2-C3-plano-tangente.tex"))
% (defun o () (interactive) (find-LATEX "2022-2-C3-plano-tangente.tex"))
% (defun u () (interactive) (find-latex-upload-links "2022-2-C3-plano-tangente"))
% (defun v () (interactive) (find-2a '(e) '(d)))
% (defun d0 () (interactive) (find-ebuffer "2022-2-C3-plano-tangente.pdf"))
% (defun cv () (interactive) (C) (ee-kill-this-buffer) (v) (g))
%          (code-eec-LATEX "2022-2-C3-plano-tangente")
% (find-pdf-page   "~/LATEX/2022-2-C3-plano-tangente.pdf")
% (find-sh0 "cp -v  ~/LATEX/2022-2-C3-plano-tangente.pdf /tmp/")
% (find-sh0 "cp -v  ~/LATEX/2022-2-C3-plano-tangente.pdf /tmp/pen/")
%     (find-xournalpp "/tmp/2022-2-C3-plano-tangente.pdf")
%   file:///home/edrx/LATEX/2022-2-C3-plano-tangente.pdf
%               file:///tmp/2022-2-C3-plano-tangente.pdf
%           file:///tmp/pen/2022-2-C3-plano-tangente.pdf
% http://angg.twu.net/LATEX/2022-2-C3-plano-tangente.pdf
% (find-LATEX "2019.mk")
% (find-sh0 "cd ~/LUA/; cp -v Pict2e1.lua Pict2e1-1.lua Piecewise1.lua ~/LATEX/")
% (find-sh0 "cd ~/LUA/; cp -v Pict2e1.lua Pict2e1-1.lua Pict3D1.lua ~/LATEX/")
% (find-sh0 "cd ~/LUA/; cp -v C2Subst1.lua C2Formulas1.lua ~/LATEX/")
% (find-CN-aula-links "2022-2-C3-plano-tangente" "3" "c3m222pt" "c3pt")

% «.defs»			(to "defs")
% «.title»			(to "title")
% «.links»			(to "links")
% «.introducao»			(to "introducao")
% «.diagramas-de-chaves»	(to "diagramas-de-chaves")
% «.primeiros-pltans»		(to "primeiros-pltans")
% «.exercicio-1»		(to "exercicio-1")
% «.curvas-de-nivel»		(to "curvas-de-nivel")
% «.exercicio-2»		(to "exercicio-2")
% «.exercicio-3»		(to "exercicio-3")
% «.sela-5x5-maxima»		(to "sela-5x5-maxima")
% «.sela-5x5»			(to "sela-5x5")
% «.retas-normais»		(to "retas-normais")
% «.exercicio-4»		(to "exercicio-4")
% «.derivada-direcional»	(to "derivada-direcional")
%
% «.djvuize»	(to "djvuize")



% <videos>
% Video (not yet):
% (find-ssr-links     "c3m222pt" "2022-2-C3-plano-tangente")
% (code-eevvideo      "c3m222pt" "2022-2-C3-plano-tangente")
% (code-eevlinksvideo "c3m222pt" "2022-2-C3-plano-tangente")
% (find-c3m222ptvideo "0:00")

\documentclass[oneside,12pt]{article}
\usepackage[colorlinks,citecolor=DarkRed,urlcolor=DarkRed]{hyperref} % (find-es "tex" "hyperref")
\usepackage{amsmath}
\usepackage{amsfonts}
\usepackage{amssymb}
\usepackage{pict2e}
\usepackage[x11names,svgnames]{xcolor} % (find-es "tex" "xcolor")
\usepackage{colorweb}                  % (find-es "tex" "colorweb")
%\usepackage{tikz}
%
% (find-dn6 "preamble6.lua" "preamble0")
%\usepackage{proof}   % For derivation trees ("%:" lines)
%\input diagxy        % For 2D diagrams ("%D" lines)
%\xyoption{curve}     % For the ".curve=" feature in 2D diagrams
%
\usepackage{edrx21}               % (find-LATEX "edrx21.sty")
\input edrxaccents.tex            % (find-LATEX "edrxaccents.tex")
\input edrx21chars.tex            % (find-LATEX "edrx21chars.tex")
\input edrxheadfoot.tex           % (find-LATEX "edrxheadfoot.tex")
\input edrxgac2.tex               % (find-LATEX "edrxgac2.tex")
%\usepackage{emaxima}              % (find-LATEX "emaxima.sty")
%
%\usepackage[backend=biber,
%   style=alphabetic]{biblatex}            % (find-es "tex" "biber")
%\addbibresource{catsem-slides.bib}        % (find-LATEX "catsem-slides.bib")
%
% (find-es "tex" "geometry")
\usepackage[a6paper, landscape,
            top=1.5cm, bottom=.25cm, left=1cm, right=1cm, includefoot
           ]{geometry}
%
\begin{document}

\catcode`\^^J=10
\directlua{dofile "dednat6load.lua"}  % (find-LATEX "dednat6load.lua")
%L dofile "Piecewise1.lua"           -- (find-LATEX "Piecewise1.lua")
%L dofile "QVis1.lua"                -- (find-LATEX "QVis1.lua")
%L dofile "Pict3D1.lua"              -- (find-LATEX "Pict3D1.lua")
%L dofile "C2Formulas1.lua"          -- (find-LATEX "C2Formulas1.lua")
%L Pict2e.__index.suffix = "%"
\pu
\def\pictgridstyle{\color{GrayPale}\linethickness{0.3pt}}
\def\pictaxesstyle{\linethickness{0.5pt}}
\def\pictnaxesstyle{\color{GrayPale}\linethickness{0.5pt}}
\celllower=2.5pt

% «defs»  (to ".defs")
% (find-LATEX "edrx21defs.tex" "colors")
% (find-LATEX "edrx21.sty")

\def\u#1{\par{\footnotesize \url{#1}}}

\def\drafturl{http://angg.twu.net/LATEX/2022-2-C3.pdf}
\def\drafturl{http://angg.twu.net/2022.2-C3.html}
\def\draftfooter{\tiny \href{\drafturl}{\jobname{}} \ColorBrown{\shorttoday{} \hours}}



%  _____ _ _   _                               
% |_   _(_) |_| | ___   _ __   __ _  __ _  ___ 
%   | | | | __| |/ _ \ | '_ \ / _` |/ _` |/ _ \
%   | | | | |_| |  __/ | |_) | (_| | (_| |  __/
%   |_| |_|\__|_|\___| | .__/ \__,_|\__, |\___|
%                      |_|          |___/      
%
% «title»  (to ".title")
% (c3m222ptp 1 "title")
% (c3m222pta   "title")

\thispagestyle{empty}

\begin{center}

\vspace*{1.2cm}

{\bf \Large Cálculo 3 - 2022.2}

\bsk

Aulas 13 e 14: Plano tangente,

reta normal e derivada direcional.

\bsk

Eduardo Ochs - RCN/PURO/UFF

\url{http://angg.twu.net/2022.2-C3.html}

\end{center}

\newpage

% «links»  (to ".links")
% (c3m222ptp 2 "links")
% (c3m222pta   "links")

{\bf Links}

% (find-books "__analysis/__analysis.el" "apex-calculus")
% (find-books "__analysis/__analysis.el" "apex-calculus" "Tangent Planes")
% (find-books "__analysis/__analysis.el" "apex-calculus" "11.4 Unit Tangent and Normal Vectors")
% (find-books "__analysis/__analysis.el" "bortolossi" "plano tangente")
% (find-LATEX "2020-1-C3-tudo.tex" "parts")
% (find-LATEX "2020-2-C3-tudo.tex" "parts")
% (find-LATEX "2021-1-C3-tudo.tex" "parts")
% (find-LATEX "2021-2-C3-tudo.tex" "parts")
% (c3m211dpp 11 "3D-fig")
% (c3m211dpa    "3D-fig")
% (c3m212mt2p 7 "figura-homogeneas")
% (c3m212mt2a   "figura-homogeneas")
% (c3m201sups1p 5 "sombrero")
% (c3m201sups1a   "sombrero")
% (find-es "maxima" "gradef")
% (find-es "maxima" "op-and-args")

(Depois)

% No início do curso nós vimos que em funções feitas de segmentos a reta
% tangente...
% 
% Revisão de inclinações como triângulos
% 
% Introdução a derivadas parciais
% 
% Derivadas parciais de funções definidas por casos

\newpage

% «introducao»  (to ".introducao")
% (c3m222ptp 3 "introducao")
% (c3m222pta    "introducao")

{\bf Introdução}

\scalebox{0.5}{\def\colwidth{11.5cm}\firstcol{

Na aula passada nós vimos que num plano com esta equação
%
$$z \;\;=\;\; F(x,y) \;\;=\;\; a + bx + cy
$$
%
dá pra encontrar os coeficientes da equação desse plano só olhando pro
diagrama de numerozinhos, fazendo isto aqui:
%
$$\begin{array}{rcl}
  a &=& F(0,0) \\
  b &=& F(x+1,y) - F(x,y) \\
  c &=& F(x,y+1) - F(x,y) \\
  \end{array}
$$

A interpretação geométrica de $b = F(x+1,y) - F(x,y)$ é a seguinte.
Digamos que a gente escolheu um ponto $(x,y)$ de $\R^2$. A gente vai
considerar que esse é o nosso ``ponto original'', e a gente desloca
ele uma unidade pra direita. O melhor modo de entender deslocamentos é
pensando em termos de ``antes'' e ``depois''; ``antes'' nós estávamos
em $(x,y)$ e ``depois'' nós andamos pra $(x+1,y)$. A gente geralmente
vai usar o subscrito `$·_0$' pra indicar ``antes'' e o subscrito
`$·_1$' pra indicar ``depois''; então $(x_0,y_0) = (x,y)$ e
$(x_1,y_1)=(x+1,y)$, e $(Δx,Δy)=(1,0)$. Além disso temos
%
$$\begin{array}{rclcl}
  z   &=& F(x,y) \\
  z_0 &=& F(x_0,y_0) &=& F(x,y)   \\
  z_1 &=& F(x_1,y_1) &=& F(x+1,y), \\
  \end{array}
$$
%
e então:
%
$$b \;\;=\;\; F(x+1,y) - F(x,y) \;\;=\;\; z_1 - z_0 \;\;=\;\; Δz
$$

}\anothercol{

Também dá pra interpretar o $c$ de uma forma parecida, só que no
caso do $c$ o deslocamento é diferente:
$(x_1,y_1) - (x_0,y_0) = (0,1)$.

Se a gente souber usar direito esses truques notacionais a gente vai
conseguir formalizar mais ou menos facilmente idéias como essa aqui:
%
\begin{quote}
  Num plano $z=F(x,y)=a+bx+cy$ o valor de $b$ é o $Δz$ quando
  $(Δx,Δy)=(1,0)$.
\end{quote}

Pra formalizar o que essa frase quer dizer a gente vai precisar de um
monte de regras que dizem como certas abreviações devem funcionar. Eu
chamo essas regras, e a notação com elas, de ``notação de físicos'', e
eu chamo a notação que não permite essas abreviações de ``notação de
matemáticos''. Nós vimos que o Bortolossi fala sobre as abreviações da
``notação de físicos'' que ele vai evitar nas páginas 171 a 173 do
capítulo 5 dele:

\ssk

{\scriptsize

% (find-books "__analysis/__analysis.el" "bortolossi")
% (find-books "__analysis/__analysis.el" "bortolossi" "a notação D_1 f")
% (find-bortolossi5page (+ -162 171)   "a notação D_1 f é a mais clara")
% (find-bortolossi5page (+ -162 172)   "omitir os pontos onde as parciais são calculadas")
% (find-bortolossi5page (+ -162 173)   "podem causar confusão")
% http://angg.twu.net/2019.2-C3/Bortolossi/bortolossi-cap-5.pdf#page=9
\url{http://angg.twu.net/2019.2-C3/Bortolossi/bortolossi-cap-5.pdf\#page=9}

}

\ssk

A ``notação de físicos'' que a gente vai usar é praticamente a do
``Calculus Made Easy'' do Silvanus P.~Thompson, mas ele 1) usa muito
pouco a convenção de que `$·_0$' e `$·_1$' indicam ``antes'' e
``depois'', 2) ele geralmente não distingue `$d$' e `$Δ$', 3) ele
muitas vezes escreve `$=$' em lugares em que seria mais correto usar
`$≈$'...



}}

\newpage

% «diagramas-de-chaves»  (to ".diagramas-de-chaves")
% (c3m222ptp 4 "diagramas-de-chaves")
% (c3m222pta   "diagramas-de-chaves")

{\bf Diagramas de chaves}

Nós às vezes vamos usar diagramas com chaves

parecidos com o do PDF sobre tipos --

\ssk

{\footnotesize

% (c3m222typesp 2 "C")
% (c3m222typesa   "C")
%    http://angg.twu.net/LATEX/2022-2-C3-tipos.pdf#page=2
\url{http://angg.twu.net/LATEX/2022-2-C3-tipos.pdf#page=2}

% (c3m222typesp 5 "exercicio-1")
% (c3m222typesa   "exercicio-1")
%    http://angg.twu.net/LATEX/2022-2-C3-tipos.pdf#page=5
\url{http://angg.twu.net/LATEX/2022-2-C3-tipos.pdf#page=5}

}

\ssk

pra indicar traduções passo a passo

ou contas passo a passo. Por exemplo:
%
$$\def\und#1#2{\underbrace{#1}_{#2}}
  \und{
  \und{F(\und{\und{x}{x_0}+\und{1}{Δx}}{x_1},
       \und{\und{y}{y_0}}{y_1})
      }{z_1}
  - \und{F(\und{x}{x_0},\und{y}{y_0})}{z_0}
  }{Δz}
$$


\newpage

% «primeiros-pltans»  (to ".primeiros-pltans")
% (c3m222ptp 5 "primeiros-pltans")
% (c3m222pta   "primeiros-pltans")

{\bf Primeiros planos tangentes}

\scalebox{0.47}{\def\colwidth{12cm}\firstcol{

Considere a seguinte construção:
%
$$\begin{array}{rcl}
  z &=& F(x,y) \\
  S &=& \setofxyzst{z=F(x,y)} \\
  (x_0,y_0) &∈& \R^2 \\
  f(Δx) &=& F(x_0+Δx,y_0) \\
  g(Δy) &=& F(x_0,y_0+Δy) \\
  \vv &=& \VEC{1,0,f'(0)} \\
  \ww &=& \VEC{0,1,g'(0)} \\
  r &=& \setofst{(x_0,y_0,z_0)+t\vv}{t∈\R} \\
  s &=& \setofst{(x_0,y_0,z_0)+u\ww}{u∈\R} \\
  π &=& \setofst{(x_0,y_0,z_0)+t\vv+u\ww}{t,u∈\R} \\
  \end{array}
$$

Ela corresponde à figura da página 739 do capítulo 12

do APEX Calculus. Dê uma olhada:

\ssk

{\scriptsize

% (find-books "__analysis/__analysis.el" "apex-calculus" "12.7" "Tangent Planes")
% (find-apexcalculuspage (+ 10 739) "12.7 Tangent Lines, Normal Lines, and Tangent Planes")
%    http://angg.twu.net/2022.2-C3/APEX_Calculus_Version_4_cap_12.pdf#page=62
\url{http://angg.twu.net/2022.2-C3/APEX_Calculus_Version_4_cap_12.pdf\#page=62}

}

\msk

As retas $r$ e $s$ são retas tangentes à superfície $S$ no ponto
$(x_0,y_0)$ e o plano $π$ é o plano que contém as retas $r$ e $s$. A
definição no APEX Calculus usa derivadas parciais, a que eu pus acima
não.

\msk

% «exercicio-1»  (to ".exercicio-1")
% (c3m222ptp 5 "exercicio-1")
% (c3m222pta   "exercicio-1")

{\bf Exercício 1.}

a) Sejam:
%
$$\begin{array}{rcl}
  F(x,y) &=& 3 + 2x + 1y \\
  A &=& \setofxyst{x,y∈\{0,1,2,3,4\}} \\
  (x_0,y_0) &=& (2,0) \\
  \end{array}
$$

Visualize todos os objetos da construção à esquerda e desenhe alguns
deles usando numerozinhos. Mais precisamente...

}\anothercol{

...mais precisamente: 1) pra visualizar a superfície $S$ você vai
desenhar um numerozinho para cada ponto do conjunto $A$; 2) pra
entender as funções $f$ e $g$ você vai fazer uma tabela com os valores
de $f(-2)$, $f(-1)$, $f(0)$, $f(1)$ e $f(2)$ e depois uma tabela
parecida para a função $g$; 3) pra visualizar as retas $r$ e $s$ você
vai desenhar como numerozinhos os pontos de $r$ e $s$ que estão sobre
o conjunto $A$; e 4) pra visualizar $π$ você vai usar o que você já
sabe sobre planos pra desenhar o diagrama de numerozinhos que contém
os pontos que você desenhos no passo 3.

\msk

b) Faça a mesma coisa, mas agora mudando o ponto $(x_0,y_0)$ para
$(3,0)$.

\msk

c) Faça a mesma coisa, mas agora mudando o ponto $(x_0,y_0)$ para
$(3,2)$.

\bsk

c) Faça a mesma coisa, mas agora para:
%
$$\begin{array}{rcl}
  F(x,y) &=& x^2+y^2 \\
  A &=& \setofxyst{x,y∈\{0,1,2,3,4\}} \\
  (x_0,y_0) &=& (2,0) \\
  \end{array}
$$

Nos itens a, b e c a superfície $S$ era um plano. Agora ela passou a
ser um parabolóide, e tudo vai passar a ser bem mais complicado.

\msk

d) Faça a mesma coisa que no item c, mas agora mudando o ponto
$(x_0,y_0)$ para $(3,0)$.

\msk

e) Faça a mesma coisa que nos dois últimos itens, mas agora mudando o
ponto $(x_0,y_0)$ para $(3,2)$.


}}


\newpage

% «curvas-de-nivel»  (to ".curvas-de-nivel")
% (c3m222ptp 6 "curvas-de-nivel")
% (c3m222pta   "curvas-de-nivel")

{\bf Curvas de nível}

\scalebox{0.55}{\def\colwidth{10cm}\firstcol{

    Dê uma olhada em como o Bortolossi define curvas de nível --- ele
    faz isso nas páginas 97 e 98 do capítulo 3 ---

% (find-books "__analysis/__analysis.el" "bortolossi")
% (find-books "__analysis/__analysis.el" "bortolossi" "3.3. Curvas de nível")
% (find-bortolossi3page (+ -78  97) "3.3. Curvas de nível")
% (find-bortolossi3page (+ -78  98)   "O desenho da curva de nível deve ser feito no plano")

\ssk

{\scriptsize

%    http://angg.twu.net/2019.2-C3/Bortolossi/bortolossi-cap-3.pdf#page=19
\url{http://angg.twu.net/2019.2-C3/Bortolossi/bortolossi-cap-3.pdf\#page=19}

}

e dê uma olhada nas figuras de curvas de nível (``level curves'') das
páginas 684 a 687 do APEX Calculus:

\ssk

{\scriptsize

% (find-books "__analysis/__analysis.el" "apex-calculus")
% (find-books "__analysis/__analysis.el" "apex-calculus" "Level curves")
% (find-apexcalculuspage (+ 10 684)      "Level curves")
%    http://angg.twu.net/2022.2-C3/APEX_Calculus_Version_4_cap_12.pdf#page=7
\url{http://angg.twu.net/2022.2-C3/APEX_Calculus_Version_4_cap_12.pdf\#page=7}

}



\bsk

% «exercicio-2»  (to ".exercicio-2")
% (c3m222ptp 6 "exercicio-2")
% (c3m222pta   "exercicio-2")

{\bf Exercício 2.}

Sejam:
%
$$\begin{array}{rcl}
  F(x,y) &=& y-2 \\
  G(x,y) &=& (x-2)+(y-2) \\
  H(x,y) &=& F(x,y)·G(x,y) \\
  \end{array}
$$

a) Faça os diagramas de numerozinhos de $F(x,y)$, $G(x,y)$ e $H(x,y)$.
Desenhe numerozinhos nos pontos com $x,y∈\{0,1,2,3,4\}$.

\msk

b) Faça uma cópia beeeem grande do seu diagrama pra $H(x,y)$ e tente
desenhar sobre ela as curvas de nível de $H(x,y)$ em $z=-1$, $z=0$,
$\ldots$, $z=8$. Discuta com os seus colegas e tente descobrir que
técnicas você pode usar pra desenhar aproximações razoáveis pra essas
curvas na mão e no olhômetro.


}\anothercol{

Nas próximas aulas nós vamos aprender truques com derivadas que vão
nos permitir desenhar aproximações bem boas pra essas curvas de nível
fazendo poucas contas. Se você conseguir visualizar bem o truque do
plano tangente do próximo exercício você vai conseguir entender bem
esses truques com derivadas.

\msk

% «exercicio-3»  (to ".exercicio-3")
% (c3m222ptp 6 "exercicio-3")
% (c3m222pta   "exercicio-3")

{\bf Exercício 3.}

Use a construção do exercício 1 pra fazer o diagrama de numerozinhos
do plano tangente à superfície $H(x,y)$ em $(x_0,y_0)=3$. Chame esse
plano tangente de $π$. Descubra qual é a interseção desse plano $π$
com o plano $z=z_0=H(x_0,y_0)$. Essa interseção vai ser uma reta;
vamos chamá-la de $r'$. {\sl O vetor diretor dessa reta vai ser
  tangente à curva de nível} --- faça todas as figuras e depois tente
entender isto.

}}


\newpage

% «sela-5x5-maxima»  (to ".sela-5x5-maxima")
% (c3m222ptp 7 "sela-5x5-maxima")
% (c3m222pta   "sela-5x5-maxima")
% (find-anggfile "MAXIMA/matrixify.mac")
% (find-esgrep "grep --color=auto -nH --null -e plot3d maxima.e")
% (setq eepitch-preprocess-regexp "^")
% (setq eepitch-preprocess-regexp "^%T ")
%
%T  (eepitch-maxima)
%T  (eepitch-kill)
%T  (eepitch-maxima)
%T load       ("~/MAXIMA/matrixify.mac");
%T [Dx,Dy] : [x-x0,y-y0];
%T [x0,y0] : [2,2];
%T F : Dy;
%T G : Dx+Dy;
%T H : F*G;
%T matrixify (0,0, 4,4, H);
%T plot3d (H, [x, 0, 4], [y, 0, 4]);


% «sela-5x5»  (to ".sela-5x5")
% (c3m222ptp 7 "sela-5x5")
% (c3m222pta   "sela-5x5")
% (c3m221vsbp 2 "questao-1")
% (c3m221vsba   "questao-1")
%
%L Pict2e.bounds = PictBounds.new(v(0,0), v(4,4))
%L sela_5x5 = Numerozinhos.from(0, 0, 
%L     [[  0  2  4  6  8
%L        -1  0  1  2  3 
%L         0  0  0  0  0 
%L         3  2  1  0 -1 
%L         8  6  4  2  0 ]])
%L sela_5x5:topictu("25pt"):sa("sela 5x5"):output()
\pu

\def\SELA{\ga{sela 5x5} \;\;\;\;\;\;}

\hspace{-0.5cm}
$\scalebox{0.85}{$
 \begin{array}{ccc}
 \SELA & \SELA & \SELA \\ \\[5pt]
 \SELA & \SELA & \SELA \\
 \end{array}
 $}
$

\newpage

% «retas-normais»  (to ".retas-normais")
% (c3m222ptp 8 "retas-normais")
% (c3m222pta   "retas-normais")

{\bf Retas normais}

Na página 741 do capítulo 12 ---

\ssk

{\scriptsize

% (find-books "__analysis/__analysis.el" "apex-calculus" "741" "Normal lines")
% (find-apexcalculuspage (+ 10 741)      "Normal lines")
%    http://angg.twu.net/2022.2-C3/APEX_Calculus_Version_4_cap_12.pdf#page=64
\url{http://angg.twu.net/2022.2-C3/APEX_Calculus_Version_4_cap_12.pdf\#page=64}

}

\ssk

o APEX Calculus define a ``reta normal'' ao plano tangente e mostra
que ela pode ser calculada por uma fórmula bem curta. Nós vamos usar
essa fórmula algumas vezes nas próximas aulas, mas agora é melhor a
gente rever como o ``produto vetorial'', ou ``produto cruzado'', é
``um pedaço da conta do determinante''.

\bsk

% «exercicio-4»  (to ".exercicio-4")
% (c3m222ptp 8 "exercicio-4")
% (c3m222pta   "exercicio-4")

{\bf Exercício 4.}

Faça os exercícios das páginas 47, 48 e 49 daqui:

% (mpgp 47 "determinantes-em-R3")
% (mpga    "determinantes-em-R3")
% (mpgp 49 "cross-prod")
% (mpga    "cross-prod")
% (find-books "__analysis/__analysis.el" "acker")
% (find-books "__analysis/__analysis.el" "acker" "produto vetorial")

\ssk

{\footnotesize

% (mpgp 47)
%    http://angg.twu.net/LATEX/material-para-GA.pdf#page=47
\url{http://angg.twu.net/LATEX/material-para-GA.pdf#page=47}

}

\newpage

% «derivada-direcional»  (to ".derivada-direcional")
% (c3m222ptp 9 "derivada-direcional")
% (c3m222pta   "derivada-direcional")
% (c3m211qp 25 "derivada-direcional")
% (c3m211qa    "derivada-direcional")
% (c3m221nfp 30 "derivada-direcional")
% (c3m221nfa    "derivada-direcional")

{\bf A derivada direcional}


\scalebox{0.9}{\def\colwidth{12cm}\firstcol{

O Bortolossi define derivada direcional na p.296 (cap.8)

e o APEX Calculus na página 729 (cap.12)... links:

\ssk

{\scriptsize

% (find-books "__analysis/__analysis.el" "bortolossi")
% (find-books "__analysis/__analysis.el" "bortolossi" "Derivada direcional")
% (find-bortolossi8page (+ -290 296)   "Definição 8.1. (Derivada direcional)")
%    http://angg.twu.net/2019.2-C3/Bortolossi/bortolossi-cap-8.pdf#page=6
\url{http://angg.twu.net/2019.2-C3/Bortolossi/bortolossi-cap-8.pdf\#page=6}

% (find-books "__analysis/__analysis.el" "apex-calculus")
% (find-books "__analysis/__analysis.el" "apex-calculus" "Directional Derivatives")
% (find-apexcalculuspage (+ 10 729) "12.6 Directional Derivatives")
%    http://angg.twu.net/2022.2-C3/APEX_Calculus_Version_4_cap_12.pdf#page=52
\url{http://angg.twu.net/2022.2-C3/APEX_Calculus_Version_4_cap_12.pdf\#page=52}

}

\bsk

{\bf Exercício 5.}

O Bortolossi usa esta notação:
%
$$\frac{∂f}{∂𝐛v}(𝐛p) =
  \lim_{t→0} \frac{ f(𝐛p + t·𝐛v) - f(𝐛p) }{t}
$$

a) Descubra como traduzir ela -- passo a passo! -- pra

``notação de físicos'', com $𝐛p=(x_0,y_0)$ e $𝐛v=\VEC{α,β}$.

\msk

b) Faça os exercícios 17, 18 e 19 daqui:

\ssk

{\scriptsize

% (c3m221nfp 30 "derivada-direcional")
% (c3m221nfa    "derivada-direcional")
%    http://angg.twu.net/LATEX/2022-1-C3-notacao-de-fisicos.pdf#page=30
\url{http://angg.twu.net/LATEX/2022-1-C3-notacao-de-fisicos.pdf\#page=30}

}

}\anothercol{
}}







%\printbibliography

\GenericWarning{Success:}{Success!!!}  % Used by `M-x cv'

\end{document}

%  ____  _             _         
% |  _ \(_)_   ___   _(_)_______ 
% | | | | \ \ / / | | | |_  / _ \
% | |_| | |\ V /| |_| | |/ /  __/
% |____// | \_/  \__,_|_/___\___|
%     |__/                       
%
% «djvuize»  (to ".djvuize")
% (find-LATEXgrep "grep --color -nH --null -e djvuize 2020-1*.tex")

 (eepitch-shell)
 (eepitch-kill)
 (eepitch-shell)
# (find-fline "~/2022.2-C3/")
# (find-fline "~/LATEX/2022-2-C3/")
# (find-fline "~/bin/djvuize")

cd /tmp/
for i in *.jpg; do echo f $(basename $i .jpg); done

f () { rm -v $1.pdf;  textcleaner -f 50 -o  5 $1.jpg $1.png; djvuize $1.pdf; xpdf $1.pdf }
f () { rm -v $1.pdf;  textcleaner -f 50 -o 10 $1.jpg $1.png; djvuize $1.pdf; xpdf $1.pdf }
f () { rm -v $1.pdf;  textcleaner -f 50 -o 20 $1.jpg $1.png; djvuize $1.pdf; xpdf $1.pdf }

f () { rm -fv $1.png $1.pdf; djvuize $1.pdf }
f () { rm -fv $1.png $1.pdf; djvuize WHITEBOARDOPTS="-m 1.0 -f 15" $1.pdf; xpdf $1.pdf }
f () { rm -fv $1.png $1.pdf; djvuize WHITEBOARDOPTS="-m 1.0 -f 30" $1.pdf; xpdf $1.pdf }
f () { rm -fv $1.png $1.pdf; djvuize WHITEBOARDOPTS="-m 1.0 -f 45" $1.pdf; xpdf $1.pdf }
f () { rm -fv $1.png $1.pdf; djvuize WHITEBOARDOPTS="-m 0.5" $1.pdf; xpdf $1.pdf }
f () { rm -fv $1.png $1.pdf; djvuize WHITEBOARDOPTS="-m 0.25" $1.pdf; xpdf $1.pdf }
f () { cp -fv $1.png $1.pdf       ~/2022.2-C3/
       cp -fv        $1.pdf ~/LATEX/2022-2-C3/
       cat <<%%%
% (find-latexscan-links "C3" "$1")
%%%
}

f 20201213_area_em_funcao_de_theta
f 20201213_area_em_funcao_de_x
f 20201213_area_fatias_pizza



%  __  __       _        
% |  \/  | __ _| | _____ 
% | |\/| |/ _` | |/ / _ \
% | |  | | (_| |   <  __/
% |_|  |_|\__,_|_|\_\___|
%                        
% <make>

 (eepitch-shell)
 (eepitch-kill)
 (eepitch-shell)
# (find-LATEXfile "2019planar-has-1.mk")
make -f 2019.mk STEM=2022-2-C3-plano-tangente veryclean
make -f 2019.mk STEM=2022-2-C3-plano-tangente pdf

% Local Variables:
% coding: utf-8-unix
% ee-tla: "c3pt"
% ee-tla: "c3m222pt"
% End:
